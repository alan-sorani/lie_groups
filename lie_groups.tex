\documentclass[10pt, twoside]{book}

\usepackage{hyperref}
\usepackage{stmaryrd}

%%%%%%%%%%%%
% Geometry %
%%%%%%%%%%%%

\RequirePackage{geometry}
\geometry{
	 a4paper,
	 inner=10mm,
	 outer=10mm,
	 marginparwidth = 20mm,
	 top = 20mm,
	 bottom = 20mm
	}

%%%%%%%%%
% Babel %
%%%%%%%%%

\usepackage[nil,bidi=basic-r]{babel}
\babelprovide[import=he,main]{hebrew}
\babelprovide[import=en-GB]{english}

% For some reason Babel’s `\babelfont` doesn’t work
\babelfont[hebrew]{rm}{Open Sans Hebrew}

\newcommand{\texthebrew}[1]{\foreignlanguage{hebrew}{#1}}
\newcommand{\nikud}[1]{$\mbox{\H{#1}}$}
\newcommand{\textenglish}[1]{\foreignlanguage{english}{#1}}
\newcommand{\LR}[1]{{‏\textdir TLT #1}}

%%%%%%%%%
% Maths %
%%%%%%%%%

\usepackage{math-fonts}
\usepackage{math-graphics}
\usepackage{math-symbols}
\usepackage{math-theorems-heb}

\newcommand{\Lie}{\mrm{Lie}}

%%%%%%%%%
% Title %
%%%%%%%%%

\title{רשימות הרצאה לחבורות לי \\ \large{חורף 2020, הטכניון}}
\author{הרצאותיו של מקס גורביץ' \\ \large \small{הוקלדו על ידי אלעד צורני}}
\date{\today}

\begin{document}

\maketitle
\tableofcontents

\chapter{מבוא}

\section{מבוא היסטורי}

\subsection{חבורות לי}

בנורבגיה, סביב שנת 1870, מתמטיקאי בשם סופוס לי, שחקר משוואות דיפרנציאליות, שם לב להופעה של מה שנקרא
\textenglish{\emph{``transformation groups''}}.

\begin{description}
\item[באופן לוקלי,]
נניח כי נתון שדה וקטורי חלק
\[\text{.} x \colon \mbb{R}^n \to \mbb{R}^n\]
נחפש פתרון למשוואה
\begin{align*}
y'\prs{t} &= x\prs{y\prs{t}} \\
y\prs{0} &= y_0
\end{align*}
עבור
$y \colon \mbb{R} \to \mbb{R}^n$.
לפי משפט קיום ויחידות של מד"ר, קיים פתרון יחיד
$y\prs{t}$
שמוגדר עבור
$t \in \prs{-\eps, \eps}$
עבור
$\eps > 0$
כלשהו.

נגדיר
$\phi_x\prs{t} \prs{y_0} \ceq y\prs{t}$
לכל
$y_0 \in \mbb{R}^n$.
אז
\[\phi_X\prs{t} \colon \mbb{R}^n \to \mbb{R}^n\]
אוטומורפיזם של
$\mbb{R}^n$.
זאת משפחה של אופרטורים שמשתנה באופן חלק כתלות ב־%
$t$.

מתקיים
\[\phi_x\prs{0} = \id\]
ומיחידות הפתרונות,
\[ \text{.} \phi_x\prs{t} \circ \phi_x\prs{s} = \phi_x\prs{t+s}\]
התמונה
$\im\phi_x$
נקראת
\emph{חבורה חד־פרמטרית}.
אז
\[\phi_x \colon \mbb{R} \to \aut\prs{\mbb{R}^n}\]
הומומורפיזם של חבורות.
זה לא מדויק, $\phi_x$ אולי לא מוגדר תמיד.

\item[באופן גלובלי,]
קיימות משוואות שהפתרונות שלהן אינווריאנטיים לפעולה של חבורה כלשהי. למשל
\[\text{.} \mcal{O}_n\prs{\mbb{R}} \ceq \set{A \in M_n\prs{\mbb{R}}}{A^t A = I}\]

נסתכל על הלפלסיאן
\[\text{.} \Delta = \sum_{i \in [n]} \del_{x_i, x_i}\]
אם
$y \colon \mbb{R}^n \to \mbb{R}$
מקיימת
$\Delta\prs{y} = 0$
אז
$y \circ g = y$
לכל
$g \in \mcal{O}_n\prs{\mbb{R}}$.
\end{description}

סופוס לי חקר חבורות חלקות כאלו, ולמעשה הגדיר חשבון דיפרנציאלי של חבורות.
כדי לחקור פונקציה חלקה
$f \colon \mbb{R}^n \to \mbb{R}^m$
אנו מסתכלים על הנגזרת של
$f$
בנקודה, שהיא העתקה לינארית
\[\text{.} T \colon \mbb{R}^n \to \mbb{R}^m\]
האנלוג לחבורות הוא
\emph{חבורת לי}
שהיא חבורה שאותה אפשר
.``לגזור''
פורמלית, זאת חבורה עם מבנה של יריעה חלקה, או במילים אחרות, אוביקט חבורה בקטגוריה של יריעות חלקות.

``נגזרת''
של חבורת לי נקראת
\emph{אלגברת לי}
$\mrm{Lie}\prs{G}$.
זהו מרחב וקטורי עם מבנה מסוים.
תורת לי בסיסית היא הבנת ההתאמה בין חבורות לי לאלגבראות לי שלהן.

\begin{examples*}
\enumthm
\begin{enumerate}
\item $\mrm{Lie}\prs{\mrm{GL}_n\prs{\mbb{R}}} = M_n\prs{\mbb{R}}$.
\item $\mrm{Lie}\prs{\mrm{GL}_n\prs{\mbb{C}}} = M_n\prs{\mbb{C}}$.
\item $\mrm{Lie}\prs{\mcal{O}_n\prs{k}} = \mfrak{so}_n\prs{k} \ceq \set{A \in M_n\prs{k}}{A^t = -A}$.
\item $\mrm{Lie}\prs{\mrm{SL}_n\prs{k}} = \mfrak{sl}_n\prs{k} \ceq \set{A \in M_n\prs{k}}{\tr\prs{A} = 0}$.
\item נגדיר
\[\mrm{Sp}_{2n}\prs{k} = \set{A \in M_{2n}\prs{k}}{A^t J A = J}\]
עבור
\[J = \pmat{0 & I_n \\ -I_n & 0} \text{.}\]
זאת נקראת
\emph{החבורה הסימפלקטית}.
אז
\[\mrm{Lie}\prs{\mrm{Sp}_{2n}\prs{k}} = \mfrak{sp}_{2n}\prs{k} \ceq \set{A \in M_{2n}\prs{k}}{A^tJ = -J A} \text{.}\]
\end{enumerate}
\end{examples*}

באופן כללי, אם ניקח חבורת לי
$G \leq \mrm{GL}_n\prs{k}$
אז
$\mrm{Lie}\prs{G} \leq M_n\prs{k}$
וזאת תהיה אלגברה יחד עם הפעולה של הקומוטטור
\[\text{.} \brs{A,B} = AB - BA\]

בקורס זה נעשה "קירוב" של התאמת לי על ידי לקיחת
\emph{כל}
תת־חבורה
$G \leq \mrm{GL}_n\prs{\mbb{R}}$
ובניית
$\mrm{Lie}\prs{G} \leq M_n\prs{\mbb{R}}$.

\subsection{סיווג של חבורות לי}

חבורות לי מופיעות במגוון מקומות בטבע ובמתמטיקה.
הרבה חבורות סימטריה בטבע הינן חבורות לי. אנו רוצים להבין בין השאר מסיבות אלו את המבנה של חבורות לי ולנסות לסווג אותן.

תחילה נבין את הקשר בין
$G_1, G_2$
כך ש־%
$\mrm{Lie}\prs{G_1} \cong \mrm{Lie}\prs{G_2}$.
התשובה לשאלה זאת תיעזר בחבורות כיסוי.
נקבל מכך שמספיק להשיג סיווג של אלגבראות לי. למעשה, מספיק לסווג אלגבראות לי מעל
$\mbb{C}$.

נצמצם לפעמים את הדיון לחבורות לי
\emph{פשוטות}
שהן חבורות לי קשירות ללא תת־חבורות נורמליות קשירות לא טריוויאליות.
דוגמאות לחבורות לי פשוטות הן
$\mrm{SO}_n = \mrm{SL}_n \cap \mcal{O}_n, \mrm{Sp}_{2n}, \mrm{SL}_n$.
מהן ניתן, במובן מסוים, לבנות כל חבורת לי.
עבור
$G$
חבורת לי פשוטה,
$\mrm{Lie}\prs{G}$
היא אלגברת לי פשוטה.
לכן, מספיק לסווג אלגבראות לי פשוטות.

סביב שנת 1890, \textenglish{Cartan} ו־%
\textenglish{Killing}
סיווגו באופן מלא אלגבראות לי פשוטות מעל
$\mbb{C}$,
שהן "אבני הבניין של הסימטריה של הטבע".

%date 26.10.2020

יש משפחות של אלגבראות לי פשוטות, שנקראות אלגבראות קלאסיות. הן
$\mfrak{sl}_n\prs{\mbb{C}}, \mfrak{so}_n\prs{\mbb{C}}, \mfrak{sp}_{2n}\prs{\mbb{C}}$
שהגדרנו מקודם.
חבורות לי שאלו הן אלגבראות לי שלהן נקראות חבורות קלאסיות.
התברר שלמעט המשפחות הקלאסיות יש עוד בדיוק 5 אלגבראות לי פשוטות, שנקראות מיוחדות
\textenglish{exceptional}.
הן מסומנות
$G_2, F_4, E_6, E_7, E_8$.
הכי קטנה מהן היא
$G_2$
בעלת מימד
$14$,
והכי גדולה היא
$E_8$
בעלת מימד
$248$.
לצורך הבנת התוצאה, נדרשת התעמקות אלגברית במבנה של אלגבראות לי, מה שדורש זמן.

\subsection{חבורות קומפקטיות}

חבורות לי הן חבורות שהן יריעות חלקות.
מתברר שאם הן קומפקטיות, יש הרבה מה להגיד עליהן.
חבורות לי קומפקטיות הן למשל
\begin{align*}
\mcal{O}\prs{n} &\equiv \mcal{O}_n\prs{\mbb{R}} \ceq \set{A \in \mrm{GL}_n\prs{\mbb{R}}}{A^t A = I} \\
\text{.} U\prs{n} &\ceq \set{A \in \mrm{GL}_n\prs{\mbb{C}}}{\bar{A}^t A = I}
\end{align*}
כאשר המבנה שלהן מושרה מזה של
$M_n\prs{k} \cong k^{n^2}$.

\subsection{תורת ההצגות}

בתורת לי הגיעו להבנה שניתן ללמוד הרבה על חבורה ע"י הבנת הפעולה שלה על מרחבים וקטוריים. זוהי תורת ההצגות. עבור חבורות קומפקטיות יש מבנה נחמד לתורת ההצגות.

\subsection{חבורות רדוקטיביות ואלגבריות}

משפט שנחתור אליו הוא שכל חבורת לי קומפקטית קשירה היא חבורה אלגברית של מטריצות ממשיות.
מסקנה ממשפט זה תהיה שלכל חבורת לי קומקפטית ניתן להגדיר קומפלקסיפיקציה על ידי הרחבת סקלרים. זאת מוגדרת על ידי השורשים המרוכבים של אותם פולינומים.
חבורות שנקבל מקומפלקסיפיקציה נקראות
\emph{רדוקטיביות}
ומהוות משפחה גדולה.

בשנת 1920 פיתח הרמן וייל
\textenglish{(Weyl)}
את המבנה וההצגות של חבורות לי רדוקטיביות על סמך חבורות קומפקטיות.

בשנת 1940 קלוד שיבליי
\textenglish{(Chevalley)}
הראה שניתן לתאר כל חבורת לי פשוטה כחבורה אלגברית.
מסקנה של כך היא ש"לא צריך" אנליזה כדי לחקור תורת לי.
אפשר גם להגדיר אנלוגית של חבורות לי מעל כל שדה.

\subsection{התפתחות מודרנית של התחום}

חבורות לי התפתחו לנושאים מאוחרים שמופיעים בהרבה תחומים.

\begin{itemize}
\item
אחד התחומים הוא חבורות אלגבריות. תחום אחר הוא חבורות סופיות מסוג לי
\textenglish{(Lie Type)}
שהן חבורות אלגבריות מעל שדה סופי. אלו אבני יסוד בסיווג של חבורות סופיות פשוטות.

\item
נושא אחר הוא מרחבים סימטריים. אלו מרחבים טבעיים עם פעולה של חבורת לי. למשל
$S^3$
הוא מרחב סימטרי עם הפעולה של
$\mcal{O}\prs{3}$.

\item
חבורות לי גם הפכו לכלי מרכזי בחקר של תבניות אוטומורפיות ותורת המספרים. הן מתקשרות למשפט
\textenglish{Harish-Chandra}%
ולתוכנית
\textenglish{Laglands}.

\item
חבורות לי מתקשרות גם לאנליזה הרמונית.

\item נחקרת היום גם תורת הצגות אינסוף־מימדיות של חבורות לי ופעולות של חבורות לי על מרחבי פונקציות.
בין השאר נחקרות פעולות על מרחבי פונקציות שמקודדים תוכן של תורת מספרים.

\item
נחקרות גם היום, משנת 1985, חבורות קוונטיות, שהן דפורמציות של חבורות לי.

\item נחקרות גם חבורות ואלגבראות אינסוף־מימדיות שנקראות
אלגבראות
\textenglish{Kac-Moody}.

\item ישנן תורת לי קטגורית ותורת לי גיאומטרית. הראשונה עוסקת למשל בקטגוריות שמוגדרות באופן דומה לחבורות לי.

\end{itemize}

\section{חבורות לי}

\subsection{אקספוננט של מטריצות}

נעבוד מעל שדה
$k \in \set{\mbb{R}, \mbb{C}}$.

נסמן ב־%
\[M \equiv M_n \equiv M_n\prs{k}\]
מטריצות
$n \times n$
מעל
$k$
עם הנורמה האוקלידית המושרית מ־%
$k^{n^2}$.

\begin{proposition}[עקרון ההצבה]
יהי
\[F\prs{z} = \sum_{n=0}^{\infty} a_n z^n\]
עבור
$a_n \in k$
טור חזקות עם רדיוס התכנסות
$r$.

לכל
$A \in M_n$
עבורה
$\norm{A} < r$
הטור
$F\prs{A} = \sum_{n=0}^{\infty} a_n A^n$
מגדיר טור מטריצות מתכנס בנורמה.
\end{proposition}

\begin{proof}
הנורמה מקיימת
\begin{align*}
\norm{X+Y} &\leq \norm{X} + \norm{Y} \\
\text{.} \norm{X \cdot Y} &\leq \norm{X} \cdot \norm{Y}
\end{align*}
לכן
\begin{align*}
\norm{\sum_{n=k}^\ell a_n A^n} &\leq \sum_{n=k}^\ell \abs{a_n}\norm{A^n}
\\&\leq
\sum_{n=k}^\ell \abs{a_n}\norm{A}^n
\\&<
\sum_{n=k}^\ell \abs{a_n}\prs{r-\eps}^n
\\&\xrightarrow[k,\ell \to \infty]{} 0
\end{align*}
ולפי קריטריון קושי נקבל שיש לסדרה
$\norm{\sum_{n=1}^N a_n A^n}$
גבול.
\end{proof}

אם
$F\prs{z}$
טור חזקות עם רדיוס התכנסות
$\rho$
ו־%
$G\prs{z}$
עם רדיוס התכנסות
$\sigma$,
אז
\begin{align*}
\prs{F+G}\prs{A} = F\prs{A} + G\prs{A} \\
\prs{FG}\prs{A} = F\prs{A} \cdot G\prs{A}
\end{align*}
עבור
$A$
עם
$\norm{A} < \min\set{\rho,\sigma}$.
אם
$G\prs{0} = 0$
אז
$F \circ G$
הוא טור חזקות, ועבור מטריצה
$A$
כך ש־%
$\norm{A} < \sigma$
וגם
$\norm{G\prs{A}} < \rho$
הטור
\[\prs{F \circ G}\prs{A} = F\prs{G\prs{A}}\]
מתכנס.



\begin{definition}[אקספוננט של מטריצה]
עבור
$z \in k$
נגדיר
\[\exp\prs{z} \ceq \sum_{n = 0}^{\infty} \frac{1}{n!} \cdot z^n\]
ועבור
$z \in k$
עם
$\abs{z} < 1$
נגדיר
\[\text{.} \log\prs{1+z} = \sum_{n=0}^{\infty} \frac{\prs{-1}^{n-1}}{n} \cdot z^n\]

מכאן, לכל מטריצה
$X \in M$
ניתן להגדיר
\[\exp\prs{X} = \sum_{n=0}^{\infty} \frac{1}{n!} X\]
ועבור
$X \in M$
עבורה
$\norm{X - I} < 1$
נגדיר
\[\text{.} \log\prs{x} \ceq \log\prs{1+\prs{X-I}}\]
\end{definition}

\begin{corollary}
\begin{enumerate}
\item לכל מטריצה
$X \in M$
מתקיים
\[\exp\prs{X} \cdot \exp\prs{-X} = I\]
וגם
$\exp\prs{X} \in \mrm{GL}_n\prs{k}$.

\item כאשר
$\norm{X - I} < 1$
מתקיים
\[\text{.} \exp\prs{\log\prs{X}} = X\]

\item עבור
$X \in M$
המקיימת
$\norm{X} < \log 2$
מתקיים
\[\text{.} \log\prs{\exp\prs{X}} = X\]

אכן
מתקיים
\begin{align*}
\norm{\exp\prs{X} - I} &= \norm{\sum_{n=1}^{\infty} \frac{1}{n!} X^n}
\\&\leq
\sum_{n=1}^{\infty} \frac{1}{n!} \norm{X}^n
\\&= \exp\prs{\norm{X}} - 1
\\&< 2 - 1
\\\text{.} \hphantom{\norm{\exp\prs{X} - I}} &= 1
\end{align*}
\end{enumerate}
\end{corollary}

\begin{exercise}
כאשר
$XY = YX$
מתקיים
\begin{align*}
\text{.} \exp\prs{X} \cdot \exp\prs{Y} = \sum_{n=0}^{\infty} \frac{1}{n!} \prs{X+Y}^n = \exp\prs{X+Y}
\end{align*}
בפרט, עבור
$t,s \in k$
מתקיים
\[\text{.} \exp\prs{tX} \cdot \exp\prs{sX} = \exp\prs{\prs{t+s}X}\]
לכן
\begin{align*}
a_X \colon k &\to \mrm{GL}_n\prs{k} \\
t &\mapsto \exp\prs{tX}
\end{align*}
הומומורפיזם של חבורות.
\end{exercise}

\begin{proposition}
\begin{enumerate}
\item $a_x$ הוא הפתרון היחיד למשוואה
\begin{align*}
a'\prs{t} = a\prs{t} \cdot X \\
a\prs{0} = I
\end{align*}
או למשוואה
\begin{align*}
a'\prs{t} = X \cdot a\prs{t} \\
a\prs{0} = I
\end{align*}
עבור
\[ \text{.} a \colon k \to \mrm{GL}_n\prs{k}\]

\item 
$a_X$
הוא ההומומורפיזם החלק היחיד
\[a \colon k \to \mrm{GL}_n\prs{k}\]
שמקיים
$a'\prs{0} = X$.
\end{enumerate}
\end{proposition}

\begin{proof}
\begin{enumerate}
\item $a_X$
פתרון למשוואה. מתקיים
\begin{align*}
\frac{\diff}{\diff t}\prs{a_X\prs{t}} &= \sum_{n=1}^\infty \frac{\diff}{\diff t} \prs{\frac{1}{n!} \prs{tX}^n}
\\&=
\sum_{n=1}^{\infty} \frac{t^{n-1}}{\prs{n-1}!} X^n
\\ &=
\exp\prs{tX} \cdot X
\\ \text{.} \hphantom{\frac{\diff}{\diff t}\prs{a_X\prs{t}}} &=
X \cdot \exp\prs{tX}
\end{align*}
יחידות נובעת ממשפט קיום ויחידות למד"ר.

\item אם
$a$
הוא הומומורפיזם כלשהו, כמו שנתון, מתקיים
\begin{align*}
a'\prs{t} &= \left. \frac{\diff}{\diff s} \prs{a\prs{t+s}} \right|_{s=0}
\\&=
\left. \frac{\diff}{\diff s}\prs{a \prs{t} \cdot a\prs{s}} \right|_{s=0}
\\&=
a\prs{t} \cdot a'\prs{0}
\\&=
a\prs{t} \cdot X
\end{align*}
ומ־%
$1$
נובע
$a = a_X$.
\end{enumerate}
\end{proof}

%TODO fill in last part of 2nd lecture

%29.10.2020

\begin{example}
נחשב אקספוננט של מטריצה אלכסונית. נקבל
\begin{otherlanguage}{english}
\begin{align*}
\exp\pmat{\lambda_1 & & \\ & \ddots & \\ & & \lambda_n}
&=
\sum_{k = 0}^{\infty} \frac{1}{k!} \pmat{\lambda_1 & & \\ & \ddots & \\ & & \lambda_n}^k
\\&=
\sum_{k=0}^{\infty} \frac{1}{k!} \pmat{\lambda_1^k & & \\ & \ddots & \\ & & \lambda_n^k}
\\\text{.}\hphantom{\exp\pmat{\lambda_1 & & \\ & \ddots & \\ & & \lambda_n}}&=
\pmat{\exp\prs{\lambda_1} & & \\ & \ddots & \\ & & \exp\prs{\lambda_n}}
\end{align*}
\end{otherlanguage}
\end{example}

\begin{definition}[אופרטור ההצמדה]
בהינתן
$a \in \mrm{GL}_n\prs{k}$
נגדיר את
\emph{אופרטור ההצמדה ב־%
$a$}
על ידי
\begin{align*}
\mrm{Ad}\prs{a} \colon M &\to M \\
\text{.} \hphantom{\mrm{Ad}\prs{a} \colon} X &\mapsto aXa^{-1}
\end{align*}
\end{definition}

\begin{notation}
עבור מרחב וקטורי
$V$
נסמן ב־%
$\endo\prs{V}$
את מרחב ההעתקות הלינאריות מ־%
$V$
לעצמו.
זהו מרחב וקטורי איזומורפי ל־%
$M_{n^2}$.
\end{notation}

\begin{notation}
עבור מרחב וקטורי
$V$
נסמן ב־%
$\mrm{GL}\prs{V} \subseteq \endo\prs{V}$
את האנדומורפיזמים ההפיכים של
$V$.
לעתים מסמנים זאת
$\mrm{Aut}\prs{V}$.
\end{notation}

\begin{remark}
מתקיים
$\mrm{Ad}\prs{a} \in \mrm{GL}\prs{M}$
כיוון ש־%
\[\text{.} \mrm{Ad}\prs{a^{-1}} \prs{\mrm{Ad}\prs{a}\prs{X}} = X\]
כמו כן,
$\mrm{Ad}$
הוא הומומורפיזם של חבורות
$\mrm{GL}_n\prs{k} \to \mrm{GL}\prs{M}$
כיוון שמתקיים
\[\text{.} \mrm{Ad}\prs{ab}\prs{X} =  ab X \prs{ab}^{-1} = a \prs{b X b^{-1}} a^{-1} = \mrm{Ad}\prs{a} \circ \mrm{Ad}\prs{b} \prs{X}\]
\end{remark}

\begin{exercise}
אם
$f\prs{z} = \sum_{k=0}^{\infty} a_k z^k$
טור חזקות, ו־%
$X \in M_n$
עבורה
$f\prs{X}$
מוגדרת, מתקיים
\[\text{.} \mrm{Ad}\prs{a} \prs{f\prs{X}} = f\prs{\mrm{Ad}\prs{a}\prs{X}}\]
בפרט, מתקיים
\[\text{.} \mrm{Ad}\prs{a} \prs{\exp\prs{X}} = \exp\prs{\mrm{Ad}\prs{a}\prs{X}}\]
\end{exercise}

\begin{corollary}
נניח כי
$V$
מרחב וקטורי עם בסיס
$B$.
יש זיהוי
\begin{align*}
\endo\prs{V} &\cong M_n \\
\text{.} \hphantom{\endo\prs{.}} T &\leftrightarrow \brs{T}_B
\end{align*}
לכן, עבור
$X \in \endo\prs{V}$
אפשר להגדיר
$\exp\prs{X} \in \mrm{GL}\prs{V}$
כאופרטור שמקיים
\[\brs{\exp\prs{X}}_B = \exp\prs{\brs{X}_B}\]
ובנייה זאת לא תלויה בבחירת הבסיס
$B$.
\end{corollary}

\begin{definition}[הקומוטטור]
עבור
$X,Y \in M$
נגדיר
\[\text{.} \brs{X,Y} = XY - YX\]
\end{definition}

\begin{definition}
עבור
$X \in M$
נגדיר
\begin{align*}
\mrm{ad}\prs{X} \colon M &\to M \\
\text{.} \hphantom{\mrm{ad}\prs{.} aaa} Y &\mapsto \brs{X,Y}
\end{align*}
\end{definition}

\begin{proposition}
עבור
$X \in M$
מתקיים
\[\text{.} \mrm{Ad}\prs{\exp\prs{X}} = \exp\prs{\mrm{ad}\prs{x}}\]
\end{proposition}

\begin{proof}
מתקיים
\[\mrm{Ad}\prs{\exp\prs{X}}\prs{Y} = \exp\prs{X} \cdot Y \cdot \exp\prs{-X}\]
ומצד שני
\begin{align*}
\text{.} \exp\prs{\mrm{ad}\prs{X}}\prs{Y} &= \sum_{k=0}^{\infty} \frac{1}{k!} \prs{\mrm{ad}\prs{X}}^k \prs{Y}
\end{align*}
נגדיר
\[\text{.} A\prs{t} \ceq \mrm{Ad}\prs{\exp\prs{tX}}\]
נחפש את
$A\prs{t}$.
מתקיים
\[A\prs{0} = \mrm{Ad}\prs{I} = \id\]
וגם
\begin{align*}
\frac{\diff}{\diff t}\prs{A\prs{t}\prs{Y}}
&=
\frac{\diff}{\diff t} \prs{\exp\prs{tX} \cdot Y \cdot \exp\prs{-tX}}
\\&=
\frac{\diff}{\diff t}\prs{\exp\prs{tX}} Y \exp\prs{-tX} + \exp\prs{tX}Y \frac{\diff}{\diff t} \prs{\exp\prs{-tX}}
\\&=
X \cdot A\prs{t}\prs{Y} + A\prs{t}\prs{Y} \cdot \prs{-X}
\\ \text{.} \hphantom{\frac{\diff}{\diff t}\prs{A\prs{t}\prs{Y}}} &= \mrm{ad}\prs{X} \prs{A\prs{t} \prs{Y}}
\end{align*}
לכן
\[\text{.} A'\prs{t} = \mrm{ad}\prs{X} \cdot A\prs{t}\]
ראינו שזה מחייב
\[\text{.} A\prs{t} = \mrm{exp}\prs{t \mrm{ad}\prs{X}}\]
לכן
$A\prs{1}$
מה שרצינו.
\end{proof}

\subsection{נוסחת
\textenglish{Campbell-Baker-Hausdorff}}

ראינו שקיימות סביבה
$U \subseteq M$
של
$0$
וסביבה
$V \subseteq \mrm{GL}_n\prs{k}$
של
$I$
כך ש־%
$\exp\colon U \to V$
הוא הומאומורפיזם.

עבור
$A,B \in V$
מספיק קרובות ל־%
$I$
מתקיים
$AB \in V$.
כלומר, אם
$A = \exp\prs{X}$
ו־%
$B = \exp\prs{Y}$
יש
$Z \in V$
עבורה
\[\text{.} \exp\prs{Z} = \exp\prs{X} \cdot \exp\prs{Y}\]
נחשוב על ההתאמה הזאת כפונקציה,
$Z = C\prs{X,Y}$,
שמוגדרת לפחות עבור
$X,Y$
קרובים מספיק ל־%
$0$.

נרצה נוסחה עבור
$C\prs{X,Y}$,
או לפחות תיאור כלשהו של
$C$.
ראינו שאם
$XY = YX$
אז
$C\prs{X,Y} = X+Y$.

\begin{proposition}[נוסחת
\textenglish{Campbell-Baker-Hausdorff}]
באופן כללי מתקיים
\[C\prs{X,Y} = x + Y + \frac{1}{2} \brs{X,Y} + \frac{1}{12} \prs{\brs{X, \brs{X,Y}} + \brs{Y,\brs{Y,X}}} + \ldots\]
כאשר שאר הביטויים הם קומוטטורים ארוכים יותר ב־%
$X,Y$.
\end{proposition}

\begin{remark}
מתקיים
\[\text{.} C\prs{tx, ty} = t\prs{x+Y} + \frac{t^2}{2} \brs{X,Y} + o\prs{t^2}\]
\end{remark}

\begin{definition}[אלגברת לי של מטריצות]
תת מרחב
$\mfrak{g} \leq M$
יקרא
\emph{אלגברת לי של מטריצות}
אם הוא סגור תחת הקומוטטור.
\end{definition}

\begin{corollary}
אם
$X,Y \in \mfrak{g}$
ו־%
$C\prs{X,Y}$
מוגדרת, אז
$C\prs{X,Y} \in \mfrak{g}$.
\end{corollary}

%2.11.2020

\begin{exercise}
לכל מטריצה
$X \in M_n$
מתקיים
\[\text{.} \det\prs{\exp\prs{X}} = e^{\tr\prs{X}}\]
\end{exercise}

נביט בטורי החזקות הבאים.
\begin{align*}
F\prs{z} &= \sum_{n \in \mbb{N}} \frac{\prs{-1}^n}{\prs{n+1}!} z^n = \left\{ \mat{ \frac{1-\exp\prs{-z}}{z} & z \neq 0 \\ 1 & z = 0 } \right. \\
G\prs{z} &= \sum_{n \in \mbb{N}} \frac{\prs{-1}^n}{n+1} z^n = \left\{ \mat{\frac{\log\prs{1+z}}{z} & z \neq 0 \\ 1 & z = 0} \right.
\end{align*}
כאשר
$G$
מוגדרת כשמתקיים
$\abs{z} < 1$.
אז
\begin{align*}
G\prs{\exp\prs{z} - 1} \cdot \exp\prs{z} \cdot F\prs{z} &=
\frac{z}{\exp\prs{z} - 1} \cdot \exp\prs{z} \cdot \frac{1 - \exp\prs{-z}}{z}
= 1
\end{align*}
למשל כאשר
$\abs{z} < \log\prs{2}$.

\begin{proposition}[נוסחת \textenglish{Duhamel}]
תהי
$x \colon \mbb{R} \to M_n$
מסילה חלקה.
מתקיים
\[\text{.} \frac{\diff}{\diff t} \exp\prs{x\prs{t}} = \exp\prs{x\prs{t}} F\prs{\ad\prs{x\prs{t}} \prs{x'\prs{t}}}\]
\end{proposition}

\begin{proof}
נגדיר
\[\text{.} Y\prs{s,t} \ceq \exp\prs{-s x\prs{t}} \cdot \frac{\diff}{\diff t} \prs{\exp\prs{s x\prs{t}}}\]
אז
\[\frac{\diff}{\diff t} \prs{\exp\prs{x\prs{t}}} = \exp\prs{x\prs{t}} Y\prs{1,t}\]
וגם
\[\text{.} Y\prs{0,t} = 0\]
אז
\begin{align*}
\frac{\del Y}{\del s} &= \exp\prs{-s x\prs{t}} \cdot \prs{-x\prs{t}} \cdot \frac{\diff}{\diff t}\prs{\exp\prs{sx\prs{t}}} + \exp\prs{-s x\prs{t}} \cdot \frac{\diff}{\diff t} \prs{x\prs{t} \exp\prs{s x\prs{t}}}
\\&=
\exp\prs{-sx\prs{t}} \cdot x'\prs{t} \cdot \exp\prs{sx\prs{t}}
\\&=
\Ad\prs{\exp\prs{-sx\prs{t}}}\prs{x'\prs{t}}
\\&=
\exp\prs{\ad\prs{-sx\prs{t}}} \prs{x'\prs{t}}
\\&=
\exp\prs{-s \ad\prs{x\prs{t}}} \prs{x'\prs{t}}
\end{align*}
ולכן
\begin{align*}
Y\prs{1,t} &=
\int_0^1 \frac{\del Y}{\del s}\prs{s,t} \diff s
\\&=
\int_0^1 \sum_{n \in \mbb{N}} \frac{\prs{-1}^n s^n}{n!} \prs{\ad\prs{x\prs{t}}}^n \prs{x'\prs{t}} \diff s
\\&=
\sum_{n \in \mbb{N}} \int_0^1 \frac{\prs{-1}^n s^n}{n!} \prs{\ad\prs{x\prs{t}}}^n \prs{x'\prs{t}} \diff s
\\&=
\sum_{n \in \mbb{N}} \frac{\prs{-1}^n}{\prs{n+1}!} \prs{\ad \prs{x\prs{t}}}^n \prs{x'\prs{t}}
\\&=
F\prs{\ad\prs{x\prs{t}}}\prs{x'\prs{t}}
\end{align*}
כנדרש.
\end{proof}

\begin{notation}
נסמן
\[\text{.} \brs{X_1, X_2, \ldots, X_k} \ceq \brs{X_1, \ldots , \brs{X_{k+2}, \brs{X_{k+1, X_k}}}}\]
\end{notation}

\begin{exercise}
אם
$C\prs{X,Y}$
מוגדר, גם
$C\prs{tX, tY}$
מוגדר לכל
$t \in \prs{0,1}$.
\end{exercise}

\begin{proposition}[נוסחת
\textenglish{Dynkin} (1947)]
אם
$C\prs{X,Y}$
מוגדר אז
\[\text{.} C\prs{X,Y} = \sum_{k \in \mbb{N}_+} \frac{\prs{-1}^{k-1}}{k} \sum_{\substack{i_1, \ldots, i_k \in \mbb{N} \\ j_1, \ldots, j_k \in \mbb{N} \\ \forall n \in \mbb{N} \colon i_n + j_n > 0}} \frac{1}{\prs{i_1 + j_1} \cdot \ldots \cdot \prs{i_k + j_k} \cdot i_1! j_1! \cdot \ldots \cdot i_k! j_k!} [\underset{i_1}{\underbrace{X, \ldots, X}}, \underset{j_1}{\underbrace{Y, \ldots, Y}}, \ldots, \underset{i_k}{\underbrace{X, \ldots, X}}, \underset{j_k}{\underbrace{Y, \ldots, Y}}]\]
\end{proposition}

\begin{proof}
נגדיר
\[Z\prs{t} = C\prs{tX, tY}\]
עבור
$t \in \prs{0,1}$,
ונגדיר
$Z\prs{0} = 0$.
אז
\[\exp Z\prs{t} = \exp\prs{tX} \cdot \exp\prs{tY}\]
ולכן
\begin{align*}
\Ad\prs{\exp Z\prs{t}} &= \Ad\prs{\exp\prs{tX}} \Ad\prs{\exp\prs{tY}} \\
\exp\prs{\ad Z\prs{t}} &= \exp\prs{\ad \prs{tX}} \exp\prs{\ad\prs{tY}}
\\
\text{.} \hphantom{\exp\prs{\ad Z\prs{t}}} &= \exp\prs{t \ad\prs{X}} \exp\prs{t \ad\prs{Y}}
\end{align*}
לכן מתקיים
\begin{align*}
\frac{\diff}{\diff t} \prs{\exp Z\prs{t}} &=
X \exp \prs{Z\prs{t}} + \exp\prs{Z\prs{t}} \cdot Y
\end{align*}
ומצד שני
\begin{align*}
\text{.} \frac{\diff}{\diff t} \prs{\exp Z\prs{t}} &=
\exp\prs{Z\prs{t}} \cdot F\prs{\ad Z\prs{t}} \prs{Z'\prs{t}}
\end{align*}
לכן
\[X \exp \prs{Z\prs{t}} + \exp\prs{Z\prs{t}} \cdot Y = \exp\prs{Z\prs{t}} \cdot F\prs{\ad Z\prs{t}} \prs{Z'\prs{t}}\]
 ולאחר פישוט נקבל
 \begin{align*}
 F\prs{\ad Z\prs{t}} \prs{Z'\prs{t}} &= \Ad\prs{\exp\prs{-Z\prs{t}}}\prs{X} + Y
 \\&=
 \exp\prs{-ad\prs{Z\prs{t}}}\prs{X} + Y
\end{align*}
לכן
\begin{align*}
Z'\prs{t} &= \prs{G\prs{\exp\prs{\ad \prs{Z\prs{t}}} - I} \exp\prs{\ad \prs{Z\prs{t}}}} \prs{\exp\prs{-ad\prs{Z\prs{t}}}\prs{X} + Y}
\\&=
G\prs{\exp\prs{\ad\prs{Z\prs{t}}} - I}\prs{X + \exp\prs{\ad Z\prs{t}}\prs{Y}}
\\&=
\sum_{k \in \mbb{N}} \frac{\prs{-1}^k}{k+1} \prs{\exp\prs{t \ad X} \exp\prs{t \ad Y} - I}^k \prs{X + \exp\prs{\ad \prs{Z\prs{t}}} \prs{Y}}
\end{align*}
כאשר
\begin{align*}
\exp\prs{t \ad \prs{Y}} \prs{Y} &= \sum_{j \in \mbb{N}} \frac{1}{j!} t^j \prs{\ad \prs{Y}}^j \prs{Y} = Y
\end{align*}
ולכן
\begin{align*}
Z'\prs{t} &= \sum_{k \in \mbb{N}} \frac{\prs{-1}^k}{k+1} \prs{\exp\prs{t \ad X} \exp\prs{t \ad Y} - I}^k \prs{X + \exp\prs{t \ad\prs{X}} \prs{Y}}
\\\text{.} \hphantom{Z'\prs{t}} &=
\sum_{k \in \mbb{N}} \frac{\prs{-1}^k}{k+1} \sum_{n \in \mbb{N}} t^n \brs{\text{sums of commutators in $X$ and $Y$}}
\end{align*}
אז
\begin{align*}
\text{.} C\prs{X,Y} & Z\prs{1}
\\&=
\int_0^1 Z'\prs{t} \diff t
\end{align*}
על ידי אינטגרציה איבר איבר בטור של
$Z'$
נגיע לנוסחה עבור
$C\prs{X,Y}$
ונקבל את הנדרש.
\end{proof}

\begin{remark}
מתקיים
\[\text{.} C\prs{X,Y} = \log\prs{\exp\prs{X} \cdot \exp\prs{Y}}\]
\end{remark}

\chapter{אלגבראות לי}

\section{חבורות לי ואלגבראות לי}

\subsection{מאלגבראות לי לחבורות לי}

לכל
$V \leq M_n\prs{\mbb{R}}$
נגדיר חבורת מטריצות
\[\Gamma\prs{V} \ceq \set{\exp\prs{X_1} \cdot \ldots \cdot \exp\prs{X_m}}{x_1, \ldots, x_m \in V} \leq \mrm{GL}_n\prs{k}\]
שהיא תת־חבורת מטריצות של
$\mrm{GL}_n\prs{k}$.

\subsection{מחבורות לי לאלגבראות לי}

בכיוון ההפוך, עבור כל תת־חבורה
$G \leq \mrm{GL}_n\prs{k}$
נגדיר את
\emph{המרחב המשיק ל־%
$G$}
על ידי
\[\text{.} \mfrak{g} \ceq \mrm{Lie}\prs{G} = \set{X \in M_n}{\exists \gamma \colon \prs{-\eps, \eps} \to G \colon \substack{ \gamma \in \mscr{C}^1 \\ \gamma\prs{0} = I \\ \gamma'\prs{0} = X}}\]

\begin{proposition}
\begin{enumerate}
\item $\mfrak{g}$
מרחב וקטורי.
\item עבור
$X \in \mfrak{g}$
ו־%
$a \in G$
מתקיים
\[\text{.} \Ad\prs{a}\prs{X} \in \mfrak{g}\]
\item $\mfrak{g}$
אלגברת לי.
\end{enumerate}
\end{proposition}

\begin{proof}
\begin{enumerate}
\item תהיינה
$X,Y \in \mfrak{g}$
ויהיו
$\alpha,\beta \in \mbb{R}$.
נניח כי
$a,b$
מסילות ב־%
$G$
עבורן
\begin{align*}
a\prs{0} &= b\prs{0} = I \\
a'\prs{0} &= X \\
\text{.} b'\prs{0} &= Y
\end{align*}
נגדיר
\begin{align*}
c\prs{t} \ceq a\prs{\alpha t} \cdot b\prs{\beta\prs{t}} \in G \\
\text{.} c\prs{0} = I
\end{align*}
אז
\[\text{.} c'\prs{0} = \alpha a'\prs{0} b\prs{0} + \beta a\prs{0} b'\prs{0} = \alpha X + \beta Y \in \mfrak{g} \]
\item תהי
$\gamma$
מסילה ב־%
$G$
עם
$\gamma\prs{0} = I$
ו־%
$\gamma'\prs{0} = X$.
לכל
$t$
בתחום ההגדרה של
$\gamma$
מתקיים
\[\text{.} \delta\prs{t} \Ad\prs{a} \prs{\gamma\prs{t}} \in G\]
אז
\begin{align*}
\delta\prs{0} &= \Ad\prs{a}\prs{I} = I \\
\delta'\prs{0} &= \left.\frac{\diff}{\diff t} \prs{a \gamma\prs{t} a^{-1}}\right|_{t=0} 
\\&=
\Ad\prs{a} \prs{\gamma'\prs{0}}
\\&=
\Ad\prs{a} \prs{X}
\end{align*}
כנדרש.

\item
יהיו
$X,Y \in \mfrak{g}$
ו־%
$a,b$
מסילות ב־%
$G$
עבורן
$a'\prs{0} = X, b'\prs{0} = Y$.
יהי
\[I = a\prs{t} a\prs{t}^{-1}\]
ואז
\[\text{.} 0 = a'\prs{0} \cdot a\prs{0}^{-1} + \left. a\prs{0} \frac{\diff}{\diff t}\prs{a\prs{t}^{-1}} \right|_{t=0}\]
לכן
\[\text{.} \left. \frac{\diff}{\diff t}\prs{a\prs{t}^{-1}}\right|_{t=0} = -a'\prs{0} = -X\]
אז
\[\gamma\prs{t} = \Ad\prs{a\prs{t}}\prs{Y} \in \mfrak{g}\]
ומתקיים
\begin{align*}
\gamma'\prs{0} &= a'\prs{0} \cdot Y \cdot a\prs{0}^{-1} + a\prs{0} Y \left. \frac{\diff}{\diff t}\prs{a\prs{t}^{-1}} \right|_{t=0}
\\&=
X \cdot Y \cdot Y + I \cdot Y \cdot \prs{-X}
\\&= \brs{X,Y}
\end{align*}
לכן
\begin{otherlanguage}{english}
\[\text{.} \brs{X,Y} = \gamma'\prs{0} \in \mfrak{g}\]
\end{otherlanguage}
\end{enumerate}
\end{proof}

נרצה להראות שתחת תנאים מתאימים, ההעתקות
$\mrm{Lie}, \Gamma$
הופכיות אחת לשנייה.
כלומר,
נרצה ליצור התאמה חח"ע בין חבורות מטריצות מסוימות לבין אלגבראות לי.

ליתר דיוק, נוכיח שלושה דברים מרכזיים.

\begin{enumerate}
\item לכל אלגברת לי
$\mfrak{g} \leq M_n$
ממשית מתקיים
\[\text{.} \mfrak{g} = \mrm{Lie}\prs{\Gamma\prs{\mfrak{g}}}\]
\item עבור כל תת־חבורה
$G \leq \mrm{GL}_n\prs{k}$
מתקיים
\[\text{.} \Gamma\prs{\mrm{Lie}\prs{G}} \leq G\]
\item נחפש תנאים על
$G$
עבורם
\[\text{.} \Gamma\prs{\mrm{Lie}\prs{G}} = G\]
חבורה
$G$
כזאת תיקרא בהמשך
\emph{קשירה}.
\\
נוכיח את משפט התת־חבורה הסגורה:
אם
$G \leq \mrm{GL}_n\prs{k}$
תת־חבורה סגורה וקשירה, מתקיים
\[\text{.} \Gamma\prs{\mrm{Lie}\prs{G}} = G\]
\end{enumerate}

\begin{remark}
בנינו התאמה בין חבורות לי של מטריצות לאלגבראות לי של מטריצות.
ניתן להתאים באופן כללי בין חבורות לי לאלגבראות לי (אבסטרקטיות).

קיים משפט שאומר שכל אלגברת לי כזאת איזומורפית לאלגברת לי של מטריצות, ונקבל מדיון זה שחבורת לי כללית היא חבורת כיסוי של חבורות לי של מטריצות.
\end{remark}

\begin{example}
החבורה
$G \leq \mrm{GL}_n\prs{k}$
של מטריצות אלכסוניות הפיכות איזומורפית ל־%
$\prs{k^\times}^n$.

מתקיים
\[\text{.} \mrm{Lie}\prs{G} = \set{X \in M_n}{\substack{\exists \gamma \colon \prs{-\eps,\eps} \to G \\ \gamma\prs{0} = I \\ \gamma'\prs{0} = X} } \text.\]
במקרה זה
$\mrm{Lie}\prs{G}$
אלגברת המטריצות האלכסוניות, שהאיזומורפית ל־%
$k^n$.

זה נכון כי לכל
$X = \mrm{diag}\prs{x_1, \ldots, x_n}$
נוכל להביט במסילה
$\gamma\prs{t} = \mrm{diag}\prs{1 + t x_1, \ldots, 1+ t x_n} \in G$.
מתקיים כאן
\[\text{.} \brs{X,Y} = 0\]
\end{example}

\begin{definition}
עבור
$\mfrak{g} \leq M$
אלגברת לי, נגדיר
\[\text{.} \Gamma\prs{\mfrak{g}} \ceq \set{\exp\prs{X_1} \cdot \ldots \cdot \exp\prs{X_t}}{\prs{X_i}_{i \in [t]} \in \mfrak{g}} \leq \mrm{GL}_n\prs{k}\]
\end{definition}

\begin{example}
אם
$\mfrak{g}$
האלגברה של מטריצות אלכסוניות נקבל
\[ \Gamma\prs{g} = \set{\mrm{diag}\prs{e^{\lambda_1}, \ldots, e^{\lambda_n}}}{\prs{\lambda_i}_{i \in [n]} \subseteq k}\]
כי
$e^\lambda e^\mu = e^{\lambda + \mu}$.

אם
$k = \mbb{C} \setminus \set{0}$,
כל
$z$
ניתן לכתיבה כ־%
$e^a$.
עבור
$G = \prs{k^\times}^n$
נקבל
\[\text{.} \Gamma\prs{\mrm{Lie}\prs{G}} = G\]

אם
$k = \mbb{R}$
נקבל כי
$e^a$
מכסה את המספרים החיוביים אז
\[\text{.} \Gamma\prs{\mrm{Lie}\prs{G}} = \set{\mrm{diag}\prs{a_1, \ldots, a_n}}{\forall i \in [n] \colon a_i > 0} \cong \prs{R_{>0}}^n\]
אם נחשוב על הטופולוגיה של
$G$
נגלה שהיא לא קשירה וש־%
$\Gamma\prs{\mrm{Lie}\prs{G}}$
רכיב קשירות של
$G$.

נסמן
$\mfrak{g} = \mrm{Lie}\prs{G}$.
ו־%
$k \in \set{\mbb{R}, \mbb{C}}$.
אז
$\exp \colon \mfrak{g} \to G$
הוא הומומורפיזם בין
$\prs{\mfrak{g}, +} \cong \prs{k^n, +}$
ל־%
$G$.
כאשר
$k = \mbb{R}$
זה איזומורפיזם
\[\text{.} \exp \colon \mfrak{g} \riso \Gamma\prs{\mrm{Lie}\prs{G}}\]
אם
$k = \mbb{C}$,
יהיה גרעין, כי
$e^{2 \pi k i} = 1$
עבור
$k$
שלם.
עבור
$n = 1$
למשל, נקבל העתקת כיסוי
\[\text{.} \exp \colon \prs{\mbb{C}, +} \to \prs{\mbb{C}^\times, \cdot}\]
\end{example}

\begin{example}
נחשוב האם
$\prs{k^n, +}$
היא חבורת מטריצות, ואם כן מה אלגברת לי שלה.

התשובה היא כן, מתקיים
\[\text{.} G = \set{\pmat{1 & \cdots & 0 & \cdots & x_1 \\ \vdots & & & & \\ 0 & & \ddots & & \\ \vdots & & & & x_n \\ 0 & \cdots & 0 & \cdots & 1}}{\prs{x_i}_{i \in [n]} \subseteq k} \leq \mrm{GL}_{n+1}\prs{k}\]
\end{example}

\begin{example}
מתקיים
\[\mrm{Lie}\prs{G} = \set{\pmat{0 & \cdots & 0 & \cdots & x_1 \\ \vdots & & & & \\ 0 & & \ddots & & \\ \vdots & & & & x_n \\ 0 & \cdots & 0 & \cdots & 0}}{\prs{x_i}_{i \in [n]} \subseteq k} \leq \mrm{GL}_{n+1}\prs{k}\]
ו־%
$\exp \colon \prs{\mrm{Lie}\prs{G}, +} \riso G$
איזומורפיזם.
\end{example}

נשים לב כי מתקיים
\[\det\prs{\exp\prs{X}} = e^{\tr X}\]
וניתן דוגמאות למשפחות של חבורות לי.

\begin{examples*}[דוגמאות למשפחות של חבורות לי]
\enumthm
\begin{enumerate}
\item \[\mrm{SL}_n\prs{k} = \set{g \in \mrm{GL}_n\prs{k}}{\det\prs{g} = 1}\]
עם אלגברת לי
\[* \mfrak{sl}_n\prs{k} = \set{X \in M_n}{\tr\prs{X} = 0}\]
כפי ששנראה.


תהי
$X \in \mfrak{sl}_n$.
תהי
$\gamma\prs{t} = \exp\prs{tX} \in \mrm{SL}_n\prs{k}$.
אז
$\gamma\prs{0} = I$
ו־%
$\gamma'\prs{0} = X$.
מתקיים
\[\mfrak{sl}_n \subseteq \mrm{Lie}\prs{\mrm{SL}_n\prs{k}}\]
ולמעשה יש שוויון:
אם
$\gamma\prs{t} \in \mrm{SL}_n\prs{k}$
וגם
$\gamma\prs{0} = I, \gamma'\prs{0} = X$
מתקיים
\[0 \underset{\brs{\gamma\prs{t} \in \mrm{SL}_n}}{=} \frac{\diff}{\diff t} \left. \prs{\det\prs{\gamma\prs{t}}}\right._{t=0} = \tr\prs{\gamma'\prs{0}} = \tr\prs{X}\]
ולכן באמת
$\mfrak{sl}_n = \mrm{Lie}\prs{\mrm{SL}_n\prs{k}}$.

\item \[\mcal{O}\prs{n} = \set{g \in \mrm{GL}_n\prs{\mbb{R}}}{g^t \cdot g = I}\]
\item \[SO\prs{n} = \mcal{O}\prs{n} \cap \mrm{SL}_n\prs{\mbb{R}}\]
עם אלגברת לי
\[\text{.} \mfrak{so}_n = \set{X \in M_n\prs{\mbb{R}}}{X^t = - X}\]
\item \[U\prs{n} = \set{g \in \mrm{GL}_n\prs{\mbb{C}}}{\bar{g}^t \cdot g = I}\]
עם אלגברת לי
\[\text{.} \mfrak{u}_n = \set{X \in M_n\prs{\mbb{C}}}{\bar{X}^t = - X}\]
\item \[\mbb{SU}\prs{n} = U\prs{n} \cap \mrm{SL}_n\prs{\mbb{C}}\]
עם אלגברת לי
\[\text{.} \mfrak{su}_n = \mfrak{u}_n \cap \mfrak{sl}_n\prs{\mbb{C}}\]
\end{enumerate}
\end{examples*}

\begin{proposition}
\begin{enumerate}
\item%1
 מתקיים
\[\text{.} \mfrak{so}_n = \mrm{Lie}\prs{\mrm{SO}\prs{n}} = \mrm{Lie}\prs{\mcal{O}\prs{n}}\]
\item%2
מתקיים
\[\text{.} \mfrak{su}_n = \mrm{Lie}\prs{\mrm{SU}\prs{n}}\]
\end{enumerate}
\end{proposition}

\begin{proof}
\begin{enumerate}
\item%1
תהי
\[\text{.} \gamma \colon \prs{-\eps, \eps} \to \mrm{SO}\prs{n}\]
אז
\[\text{.}\forall s \in \prs{-\eps,\eps} \colon \prs{\gamma\prs{s}}^t \gamma\prs{s} = I\]
נגזור את המשווה ונקבל
\[\text{.} \frac{\diff}{\diff s} \prs{\gamma\prs{s}^t}_{s = 0} \cdot \cancelto{I}{\gamma\prs{0}} + \cancelto{I}{\gamma\prs{0}^t} \cdot \gamma'\prs{0} = 0\]
שחלוף מתחלף עם נגזרת, לכן נקבל
\[\text{.} \gamma'\prs{0}^t + \gamma'\prs{0} = 0\]
לכן
$\gamma'\prs{0} \in \mfrak{so}_n$.

להיפך, אם
\begin{align*}
\exp\prs{sX}^t &= \sum_{n \in \mbb{N}} \frac{1}{n!} s^n \prs{X^t}^n
\\&= \exp\prs{sX^t}
\\&= \exp\prs{-sX}
\\\text{.} \hphantom{\exp\prs{sX}^t} &= \exp\prs{sX}^{-1}
\end{align*}
לכן
$\exp\prs{sX} \in \mcal{O}\prs{n}$.
ראינו שאם
$X \in \mfrak{sl}_n$
אז
$\exp\prs{sX} \in \mrm{SL}_n\prs{\mbb{R}}$,
לכן
$\gamma\prs{s} = \exp\prs{sX} \in \mrm{SO}\prs{n}$.
אכן מתקיים
$\gamma\prs{0} = I, \gamma'\prs{0} = X$.

\item%2
ההוכחה דומה.
\end{enumerate}
\end{proof}

%9.11.2020

\section{דוגמאות במימדים נמוכים}

\begin{examples*}
\enumthm
\begin{enumerate}
\item $\mcal{O}\prs{1} = \set{\pm 1} \cong \mbb{Z}/2\mbb{Z}$.
\item \[\mrm{SO}\prs{2} = \set{\pmat{a & -b \\ b & a}}{a^2 + b^2 = 1} = \set{\pmat{\cos \theta & - \sin \theta \\ \sin \theta & \cos \theta}}{\theta \in \mbb{R}} \cong S^1\]
אלו כל הסיבובים בזווית
$\theta$.
\item 
מתקיים
\[\text{.}\mcal{O}\prs{2} = \mrm{SO}\prs{2} \sqcup \pmat{0 & 1 \\ 1 & 0} \cdot \mrm{SO}\prs{2}\]
\item מתקיים
$\mbb{C}^\times \cong \mrm{GL}_1\prs{\mbb{C}}$.
כמרחב ממשי
\[\text{.} \mbb{C} = \spn\set{1,i}\]
כל
$z \in \mbb{C} \setminus \set{0}$
נותן אופרטור הפיך על
$C \cong \mbb{R}^2$
על ידי מכפלה. אם נביט במטריצה המייצגת של אופרטור זה ביחס לבסיס
$\prs{1,i}$
נקבל ש־%
$z = a + ib$
מיוצג על ידי
$\pmat{a & b \\ - b & a}$.

נקבל שיכון
\[\text{.} \iota \colon \mrm{GL}_1\prs{\mbb{C}} \rmono \mrm{GL}_2\prs{\mbb{R}}\]
מתקיים
\[\text{.} \mrm{U}\prs{1} \eq \set{z \in \mbb{C}}{z \bar{z} = 1} = S^1\]
אז
\[\text{.} \iota\prs{\mrm{U}\prs{1}} = \mrm{SO}\prs{2}\]
זאת מתרחבת להעתקה לינארית
$\mbb{C} \to M_2\prs{\mbb{R}}$.
לכן
$\mrm{U}\prs{1} \cong \mrm{SO}\prs{2}$.
\item
מתקיים
$\mrm{Lie}\prs{\mrm{GL}_1\prs{\mbb{C}}} = \mbb{C}$.
אם ניקח
$z \in \mbb{C}$
נקבל
\[\text{.} \iota\prs{\exp\prs{z}} = \exp\prs{\iota\prs{z}}\]
אז מתקיים
\begin{align*}
\mrm{Lie}\prs{i\prs{\mrm{U}\prs{1}}} &= \mrm{Lie}\prs{\mrm{SO}\prs{2}}
\\&=
\mfrak{so}_2
\\&=
\set{\pmat{0 & -\alpha \\ \alpha & 0}}{\alpha \in \mbb{R}}
\\&=
i \mbb{R}
\\\text{.}\hphantom{\mrm{Lie}\prs{i\prs{\mrm{U}\prs{1}}}}&\leq
\mbb{C}
\end{align*}
עבור
$\pmat{0 & - \alpha \\ \alpha & 0} \in \mfrak{so}_2$
אכן ניתן להגדיר
\[\gamma_\alpha\prs{t} \ceq \pmat{\cos\prs{\alpha t} & -\sin\prs{\alpha t} \\ \sin\prs{\alpha t} & \cos\prs{\alpha t}} \in \mrm{SO}\prs{2}\]
שמקיימת
$\gamma_\alpha\prs{0} = I, \gamma_{\alpha}'= \pmat{0 & - \alpha \\ \alpha & 0}$.
\end{enumerate}
\end{examples*}

\begin{theorem}[משפט הסיבוב של אוילר]
כל איבר
$g \in \mrm{SO}\prs{3}$
הוא סיבוב סביב ציר נתון במרחב.

כלומר, קיימת מטריצה
$T \in \mrm{SO}\prs{3}$
עבורה
\[\text{.} TgT^{-1} = \pmat{\cos \theta & -\sin\theta & 0 \\ \sin \theta & \cos \theta & 0 \\ 0 & 0 & 1}\]
\end{theorem}

\begin{proof}
מתקיים
\begin{align*}
\det\prs{g - I} &=
\det\prs{g^t - I}
\\&=
\det\prs{g^1 - I}
\\&=
\det\prs{g^{-1}\prs{I-g}}
\\&=
\det\prs{g^{-1}} \det\prs{I - g}
\\&=
\det\prs{I - g}
\\&=
\prs{-1}^3 \det\prs{g-I}
\\&=
-\det\prs{g-I}
\end{align*}
לכן
$\det\prs{g-I} = 0$.
לכן
$1$
שורש של הפולינום האופייני של
$g$
לכן יש ל־%
$g$
וקטור עצמי
$e_3 \in \mbb{R}^3$
עם ערך עצמי
$1$.
נסמן
$W \ceq \set{e_3}^\perp \leq \mbb{R}^3$.
אז
\begin{align*}
\trs{g \cdot W, e_3} &=
\trs{W, g^t \cdot e_3}
\\&=
\trs{W, g^{-1} e_3}
\\&=
\trs{W, e_3}
\\&=
\set{0}
\end{align*}
לכן
$W$
נשמר על ידי
$g$.
לכן קיים בסיס אורתונורמלי
$B$
שבו
\[\brs{g}_B = \pmat{h & \\ & 1}\]
עבור
$h \in M_2\prs{\mbb{R}}$
וקיימת
$T \in \mrm{SO}\prs{2}$
עבורה
\[\text{.} T g T^{-1} = \brs{g}_B = \pmat{h & \\ & 1}\]
לכן
$h = \pmat{\cos \theta & - \sin\theta \\ \sin \theta & \cos\theta}$
עבור
$\theta \in \mbb{R}$
כלשהי.
\end{proof}

\begin{remark}
נכתוב
\[\pmat{h & \\ & 1} = \exp\prs{Y}\]
עבור
\[\text{.} Y = \pmat{0 & \alpha & 0 \\ -\alpha & 0 & 0 \\ 0 & 0 & 0} \in \mfrak{so}_3\]
אז
\[g = \Ad\prs{T^{-1}}\prs{\exp\prs{Y}} = \exp\prs{\Ad\prs{T^{-1}}\prs{Y}}\]
ומתקיים
\[\text{.} \Ad\prs{T^{-1}}\prs{Y} \in \mfrak{so}_3\]
אז
\[\exp \colon \mfrak{so}_3\to\mrm{SO}\prs{3}\]
הוא על.
\end{remark}

נזכיר כי מתקיים
\[\text{.} \mfrak{so}_3 = \set{X \in M_3\prs{\mbb{R}}}{X^t = -X}\]
נבחר בסיס
\begin{align*}
E_1 &= \pmat{0 & 0 & 0 \\ 0 & 0 & -1 \\ 0 & 1 & 0} \\
E_2 &= \pmat{0 & 0 & 1 \\ 0 & 0 & 0 \\ -1 & 0 & 0} \\
\text{.} E_3 &= \pmat{0 & -1 & 0 \\ 1 & 0 & 0 \\ 0 & 0 & 0}
\end{align*}
ל־%
$\mfrak{so}_3$.
נסמן את האיזומורפיזם
\begin{align*}
\mfrak{so}_3 &\to \mbb{R}^3 \\
\alpha_1 E_2 + \alpha_2 E_2 + \alpha_3 E_3 &\mapsto \pmat{\alpha_1 \\ \alpha_2 \\ \alpha_3}
\end{align*}
על ידי
$X \mapsto \vec{X}$.

\begin{lemma}
\begin{enumerate}
\item מתקיים
\[\text{.} \forall X,Y \in \mfrak{so}_3 \colon \overrightarrow{AB}{\brs{X,Y}} = X\prs{\vec{Y}}\]
\item
מתקיים
\[\text{.} \overrightarrow{\Ad\prs{a}\prs{X}} = a\prs{\vec{X}}\]
\end{enumerate}
\end{lemma}

\begin{proof}
\begin{enumerate}
\item
מספיק לבדוק על איברי בסיס
\[\text{.} \overrightarrow{\brs{E_i, E_j}} = E_i \prs{\vec{E_j}}\]
\item $\exp$
הוא על.
נקבל
\begin{align*}
\overrightarrow{\Ad\prs{a} \prs{Y}}
&=
\overrightarrow{\exp\prs{\ad X}\prs{Y}}
\\&=
\overrightarrow{\sum_{n \in \mbb{N}} \frac{1}{n!} \prs{\ad X}^n\prs{Y}}
\\&=
\sum_{n \in \mbb{N}} \frac{1}{n!} \overrightarrow{\prs{\ad\prs{X}}^n\prs{Y}}
\\&=
\sum_{n \in \mbb{N}} \frac{1}{n!} X^n\prs{\vec{Y}}
\\&=
\exp\prs{X} \prs{\vec{Y}}
\\\text{.} \hphantom{\overrightarrow{\Ad\prs{a} \prs{Y}}} &=
a\prs{\vec{Y}}
\end{align*}
\end{enumerate}
\end{proof}

\begin{remark}
הפעולה של
$\mrm{SO}\prs{3}$
על
$\mrm{Lie}\prs{\mrm{SO}\prs{3}}$
על ידי
$a \cdot X = \Ad\prs{a}\prs{X}$
איזומורפית לפעולה הסטנדרטית של
$\mrm{SO}\prs{3}$
על
$\mbb{R}^3$.
\end{remark}

\begin{proposition}
\begin{enumerate}
\item עבור
$X \in \mfrak{so}_3$
המטריצה
$\exp\prs{X}$
היא סיבוב סביב
$\vec{X} \in \mbb{R}^3$
בכיוון יד ימין בזווית
$\norm{\vec{x}}$.
\item מתקיים
$\exp\prs{X} = \exp\prs{Y}$
אם ורק אם
$X = c \cdot Y$
עבור
$c \in \mbb{R}$
וגם
\[\norm{\vec{X} - \vec{Y}} = 2 \pi k\]
עבור
$k \in \mbb{Z}$.
\end{enumerate}
\end{proposition}

\begin{proof}
\begin{enumerate}
\item יהי
$X \in \mfrak{so}_3$.
אז יש
$a \in \mrm{SO}\prs{3}$
ו־%
$\alpha = \norm{\vec{X}}$
עבורם
\begin{align*}
\vec{X} &= a \cdot \prs{\vec{E_3}}
\\&=
\overrightarrow{AB}{\alpha \cdot \Ad\prs{a} E_3}
\end{align*}
ולכן
\[\text{.} X = \alpha \Ad\prs{a} \prs{E_3} = \Ad\prs{a} \pmat{\pmat{0 & \alpha & 0 \\ - \alpha & 0 & 0 \\ 0 & 0 & 0}}\]
אז
\begin{align*}
\exp\prs{X} &=
\exp\prs{\Ad\prs{a} \pmat{\pmat{0 & \alpha & 0 \\ - \alpha & 0 & 0 \\ 0 & 0 & 0}}}
\\
\text{.} \hphantom{\exp\prs{X}} &= \Ad\prs{a} \pmat{\cos\alpha & - \sin \alpha & 0 \\ \sin \alpha & \cos \alpha & 0 \\ 0 & 0 & 1}
\end{align*}

אז
$\exp\prs{X}$
סיבוב סביב הציר
\[a \cdot \pmat{0 \\ 0 \\ 1} = a\prs{\vec{E_3}} = \frac{1}{\alpha} \vec{X}\]
או סביב הציר
$\vec{X}$.
הזווית היא
$\alpha = \norm{\vec{X}}$.

\item
תרגיל.
\end{enumerate}
\end{proof}

\begin{exercise}
חישבו על
$\mrm{SO}\prs{2}$
כמרחב טופולוגי כפי שמתקבל מהטענה.
$\exp \colon \mfrak{so}_3 \to \mrm{SO}\prs{3}$
העתקה רציפה ועל.
\end{exercise}

\begin{example}
מתקיים
\[\text{.} \mrm{SU}\prs{2} = \set{A \in \mrm{GL}_2\prs{\mbb{C}}}{\bar{A}^t = A^{-1}} = \set{\pmat{\alpha &- \bar{\beta} \\ \beta & \bar{\alpha}}}{\substack{\abs{\alpha}^2 + \abs{\beta}^2 = 1}}\]
טופולוגית נוכל לזהות זאת עם
$S^3$.

מהמשפט הספקטרלי, לכל
$a \in \mrm{SU}\prs{2}$
יש
$g \in \mrm{SU}\prs{2}$
עבורה
\[\text{.} g a g^{-1} = \pmat{z & 0 \\ 0 & z^{-1}}\]
ידוע
$z^{-1} = \bar{z}$
לכן
$z = e^{i\theta}$
כלומר
\[g a g^{-1} = \exp \pmat{i\theta & 0 \\ 0 & - i\theta}\]
ואז
\[\text{.} a = \Ad\prs{g^{-1}}\exp\pmat{i\theta & \\ & -\theta} = \exp\prs{\Ad\prs{g^{-1}} \pmat{i\theta & \\ & -i\theta}}\]
לכן
\[\exp \colon \mfrak{su}_2 \to \mrm{SU}\prs{2}\]
על.
מתקיים
\[\mfrak{su}_2 = \set{\pmat{i \zeta_3 & \zeta_1 - i \zeta_2 \\ \zeta_1 + i\zeta_2 & -i\zeta_3}}{\zeta_i \in \mbb{R}}\]
ואז
$\mfrak{su}_2 \cong \mbb{R}^3$
כשמטריצה כנ"ל מתאימה לוקטור
$\pmat{\zeta_1 \\ \zeta_2 \\ \zeta_3}$.
במעבר ל־%
$\mbb{R}^3$
מתקבלת המכפלה הפנימית הסטנדרטית.

אם
$a \in \mrm{SU}\prs{2}$
ו־%
$X,Y \in \mfrak{su}_2$
נקבל
\begin{align*}
\trs{\Ad\prs{a}\prs{X}, \Ad\prs{a}\prs{Y}} &=
\tr\prs{\overline{aX a^{-1}}^t a Y a^{-1}}
\\&=
\tr\prs{a \bar{X}^t Y a^{-1}}
\\\text{.} \hphantom{\trs{\Ad\prs{a}\prs{X}, \Ad\prs{a}\prs{Y}}} &= \trs{X,Y}
\end{align*}

אז
\[\Ad \colon \mrm{SU}\prs{2} \to \mrm{GL}\prs{\mfrak{su}_2} \cong \mrm{GL}_3\prs{\mbb{R}}\]
והאיזומורפיזם נותן
\[\text{.} \Im \Ad \subseteq \mcal{O}\prs{2}\]
למעשה גם
\[\text{.} \Im \Ad \subseteq \mrm{SO}\prs{2}\]
לכן אפשר במקרה זה להסתכל על
\[\text{.} \Ad \colon \mrm{SU}\prs{2} \to \mrm{SO}\prs{3}\]

אם
$Z \in \mfrak{su}_2$
מתקיים
\[\text{.} \forall X,Y \in \mfrak{su}_2 \colon \trs{\ad\prs{Z}\prs{X}, Y} = \trs{X, -\ad\prs{Z}\prs{Y}}\]
אז
\[\ad \colon \mfrak{su}_2 \to \endo\prs{\mfrak{su}_2} \cong M_3\prs{\mbb{R}}\]
והמשוואה נותנת בזיהוי זה ש־%
$\ad\prs{Z}^t = - \ad\prs{Z}$.
לכן נוכל לחשוב על
$\ad$
כהעתקה
\[\text{.} \ad \colon \mfrak{su}_2 \to \mfrak{so}_2\]
\end{example}

\begin{proposition}
\begin{enumerate}
\item
\[\ad \colon \mfrak{su}_2 \to \mfrak{so}_3\]
איזומורפיזם של אלגבראות לי.
\item
\[\Ad \colon \mrm{SU}\prs{2} \to \mrm{SO}\prs{3}\]
היא על ומתקיים
\[\text{.} \ker \Ad = \set{\pm 1}\]
\end{enumerate}
\end{proposition}

\begin{proof}
\begin{enumerate}
\item משוויון מימדים, מספיק להראות שמתקיים
\[\text{.} \ker \prs{\ad} = \set{0}\]
נניח כי
$X \in \ker \prs{\ad}$
כלומר
\[\text{.} \forall Y \in \mfrak{su}_2 \colon \brs{X,Y} = 0\]
תהי
$Z \in M_2\prs{\mbb{C}}$.
אפשר לכתוב
$Z = U + iV + \alpha I$
עבור סקלר
$\alpha \in \mbb{C}$
ועבור
$U,V \in \mfrak{su}_2$.
נקבל
\begin{otherlanguage}{english}
\[\text{.} \brs{X, Z} = \brs{X, U} + i \brs{X,V} + \brs{X, \alpha I} = 0+0+0 = 0\]
\end{otherlanguage}

לכן
$\brs{X,Z} = 0$
ולכן
$X$
מתחלף עם
$M_2\prs{\mbb{C}}$.
לכן יש
$\beta \in \mbb{C}$
עבורה
$X = \beta I$.
ידוע
$\tr X = 0$
לכן
$\beta = 0$
ולכן
$X = 0$.

\item
חישוב
$\ker \Ad$
דומה מאוד לסעיף הקודם.
$g \in \ker \Ad$
מתחלפת עם מטריצות כי
$g a g^{-1} = a$.
כדי להראות ש־%
$\Ad$
יהיה על נשים לב כי
$\Ad \circ \exp = \exp \circ \ad$
כאשר
$\exp, \ad$
שתיהן על.
\end{enumerate}
\end{proof}

%12.11.2020

\begin{theorem}
לכל חבורה
$G \leq \mrm{GL}_n\prs{k}$
מתקיים
$\Gamma\prs{\mrm{Lie}\prs{G}} \leq G$.

במילים אחרות, לכל
$G \leq \mrm{GL}_n\prs{k}$
ולכל
$X \in \mrm{Lie}\prs{G}$
מתקיים
$\exp\prs{X} \in G$.
\end{theorem}

\begin{proof}
תהי
$G \leq \mrm{GL}_n\prs{k}$
ותהי
$X \in \mrm{Lie}\prs{G}$.
נוכיח שמתקיים
$\exp\prs{tX} \in G$
עבור
$\abs{t}$
מספיק קטן.
אם זה מספיק עבור
$\abs{t} < \eps$,
עבור
$k \in \mbb{Z}$
עבורו
$\frac{1}{k}$
נקבל
\[\text{.} \exp\prs{\frac{1}{k} X} \in G \implies \exp\prs{X} = \exp\prs{\frac{1}{k}X}^k \in G\]

תהי
$\mfrak{g} = \mrm{Lie}\prs{G}$
ונבחר בסיס
$\prs{X_i}_{i \in [k]}$
עבור
$\mfrak{g}$
כמרחב וקטורי.
קיימות מסילות
$a_i\prs{t} \in G$
כך שמתקיים
$a_i\prs{0} = I$
וגם
$a_i'\prs{0} = X_i$.
נגדיר
\[g \colon \mfrak{g} \to G\]
על ידי
\[\text{.} g\prs{t_1 X_1 + \ldots + t_k X_k} = a_1\prs{t_1} \cdot \ldots \cdot a_k\prs{t_k}\]
יהי
$S$
תת־מרחב משלים ל־%
$\mfrak{g}$
בתוך
$M_n$.

תהי
\begin{align*}
h \colon S &\to M \\
\text{,} Y &\mapsto I + Y
\end{align*}
כך שמתקיים
$f = g\cdot h$.
אז
\begin{align*}
\prs{\diff h}_0\prs{Y} &= Y \\
\text{.} \prs{\diff g}_0\prs{X} &= X
\end{align*}

נגדיר
\[\text{.} f = g \cdot h \colon M_n \to M_n\]
אז
\[\prs{\diff f}_0 = \id, \quad f\prs{0} = I\]
ובפרט
$\prs{\diff f}_0$
הפיכה.
לכן, לפי משפט הפונקציה ההפוכה
$f$
הפיכה מקומית ב־%
$0$.
כלומר, קיימות סביבה
$W_1$
של
$I$,
סביבה
$W_2$
של
$0$,
והעתקה
\[f^{-1} \colon W_1 \to W_2 \subseteq M_n = \mfrak{g} \oplus S\]
שהפכית ל־%
$f$.
כיוון שהעתקה לסכום ישר היא סכום העתקות (כי סכום ישר הוא קו־מכפלה) נוכל לכתוב
$f^{-1} = U+V$
כאשר
$U,V$
העתקות מ־%
$W_1$
ל־%
$\mfrak{g}, S$
בהתאמה.

נרצה להראות שמתקיים
$V\prs{\exp\prs{tX}} = 0$
כיוון שאז נקבל
\[f^{-1}\prs{\exp\prs{tX}} = g\]
%TODO fill in
מספיק להראות
\[\frac{\diff}{\diff t} \prs{V \prs{\exp\prs{tX}}} = 0\]
וכיוון שמתקיים
$a\prs{t} = \exp\prs{tX}$
צריך לשם כך להראות
\[\diff V_{a\prs{t}} \cdot X \cdot a\prs{t} = \prs{\diff V}_{a\prs{t}} \cdot a'\prs{t} = \frac{\diff}{\diff t} \prs{V \prs{\exp \prs{tX}}} = 0 \text{.}\]
נרצה לכן להראות
\[\prs{\diff V}_a \cdot X \cdot a = 0\]
לכל מטריצה בסביבת
$I$
ולכל
$X \in \mfrak{g}$.
מתקיים
$V \colon M_n \to S$
ולכן
$\diff V_a \colon M_n \to S$.
כעת
\begin{align*}
a &= f\prs{X_a, Y_a}
\\&=
g\prs{X_a} \cdot h\prs{Y_a}
\end{align*}
כאשר
$X_a \in \mfrak{g}, Y_a \in S$.
אז
\begin{align*}
V\prs{f\prs{X_a + X, Y_a}} = Y_a
\end{align*}
לכל
$X \in \mfrak{g}$.
כלומר,
\[\text{.} \prs{\diff V}_a \cdot \prs{\diff f}_{\prs{X_a, Y_a}} \prs{X} = 0\]
אבל
\begin{align*}
\prs{\diff f}_{\prs{X_a, Y_a}}\prs{X} &= \prs{\diff g}_{X_a} \prs{X} \cdot h\prs{Y_a}
\\\text{.} \hphantom{\prs{\diff f}_{\prs{X_a, Y_a}}\prs{X}}&=
\prs{\diff g}_{X_a} \cdot X \cdot g\prs{X_a}^{-1} \cdot a  
\end{align*}
קיבלנו
\[\prs{\diff V}_a \cdot \prs{\diff g}_{X_a} \cdot X \cdot g\prs{X_a}^{-1} \cdot a = 0\]
לכל
$a$
בסביבת
$I$
ולכל
$X \in \mfrak{g}$.
אבל, נרצה
\[\text{.}\prs{\diff V}_a \cdot X \cdot a = 0\]
נסמן
$X' \ceq \prs{\diff g}_{X_a} \cdot X \cdot g\prs{X_a}^{-1}$
ונשים לב שמתקיים
$X' = \gamma'\prs{0}$.
אז
\[\gamma\prs{t} = g\prs{X_a - tX} \cdot g\prs{X_a}^{-1} \in G\]
וגם
$\gamma\prs{0} = I$.
לכן מהגדרת
$\mfrak{g}$
נקבל
$X'$.
לכן ההעתקה
$X \mapsto X'$
היא העתקה לינארית
$\mfrak{g} \to \mfrak{g}$.
אם
$a = I$
ו־%
$f\prs{0,0} = I$
נקבל
\[\prs{\diff g}_0 = \id, \quad X_a = 0\]
ואז
\[A_I = \id\]
אופרטור הפיך.
לכן אם נרחיק את
$a$
ממטריצת היחידה קצת, עדיין
$A_a$
הפיך, ולכן על.
\end{proof}

\chapter{חבורות מטריצות}

\subsection{חבורות מטריצות כחבורות לי}

\section{טופולוגיה על חבורות מטריצות}

\begin{definition}
ראינו שלכל חבורה
$G \leq \mrm{GL}_n\prs{k}$
אפשר לדבר על
$\exp\prs{X} \in G$
לכל
$X \in \mrm{Lie}\prs{G}$.

לכל
$G \leq \mrm{GL}_n\prs{k}$
נגדיר טופולוגיה חדשה באופן הבא.

נאמר שקבוצה
$U \subseteq G$
פתוחה אם לכל
$g \in U$
קיים
$\eps > 0$
כך שמתקיים
$g \cdot \exp\prs{X} \in U$
לכל
$X \in \mrm{Lie}\prs{G}$
המקיים
$\norm{X} < \eps$.

פורמלית, הקבוצות
\[B_{g,\eps} = \set{g \cdot \exp\prs{X}}{\substack{X \in \mrm{Lie}\prs{G} \\ \norm{X} < \eps}}\]
נותנות בסיס לטופולוגיה.
\end{definition}

\begin{exercise}
באופן דומה אפשר לקחת בסיס אחר
$\set{\exp\prs{X} \cdot g}$
אבל הוא נותן את אותה הטופולוגיה.
\end{exercise}

\begin{exercise}
ההעתקות
\begin{align*}
\exp \colon \mrm{Lie}\prs{G} &\to G \\
\cdot \colon G \times G &\to G \\
\prs{}^{-1} \colon G &\to G
\end{align*}
רציפות בטופולוגיה הנ"ל.
\end{exercise}

\begin{observation}
קיבלנו שלכל
$g \in G$
אם נבחר
$\eps > 0$
מספיק קטן יש סביבה
$B_{g,\eps}$
של
$g$
שהומיאומורפית לקבוצה פתוחה במרחב אוקלידי.
לכן
$G$
יריעה טופולוגית.
ההעתקות המעבר (בין הקבוצות במרחבים האוקלידיים) יוצאות חלקות ולכן
$G$
\emph{יריעה חלקה}.

אכן, אם
\[h = g_1 \exp\prs{X_1} = g_2 \cdot \exp\prs{X_2}\]
נקודה בחיתוך של שתי קבוצות פתוחות, ניתן לכתוב
\[X_1 = \log\prs{g_1^{-1} g_2 \exp\prs{X_2}} \coloneqq \psi\prs{X_2}\]
ולכן
$X_1$
כפונקציה של
$X_2$
היא פונקציה חלקה
$\mbb{R}^n \to \mbb{R}^n$.
ההופכית שלה חלקה באותו אופן, ולכן זה דיפאומורפיזם.
\end{observation}

\begin{remark}
במבנה של
$G$
כיריעה חלקה, ההעתקות
\begin{align*}
\exp \colon \mrm{Lie}\prs{G} &\to G \\
\cdot \colon G \times G &\to G \\
\prs{}^{-1} \colon G &\to G
\end{align*}
חלקות.
\end{remark}

\begin{remark}
לפעמים הטופולוגיה שהגדרנו מתלכדת עם הטופולוגיה על
$G$
כתת־קבוצה של
$M_n$,
אבל זה לא נכון תמיד.
\end{remark}

\begin{example}
תהי
$G \ceq \mrm{GL}_n\prs{\mbb{Q}}$.
כל מסילה חלקה
$\gamma \colon \prs{-\eps, \eps} \to G$
קבועה.
לכן
$\gamma'\prs{0} = 0$
ולכן
$\mrm{Lie}\prs{G} = \set{0}$.

אז
$B_{g,\eps} = \set{g}$
ונקבל את הטופולוגיה הדיסקרטית על
$G$.
זאת אינה הטופולוגיה היחסית.
\end{example}

\begin{example}[ישרים על הטורוס]
תהי
\[\mbb{T}^2 \ceq \set{\pmat{a & 0 \\ 0 & b}}{\abs{a} = \abs{b} = 1} \leq \mrm{GL}_2\prs{\mbb{C}}\]
מתקיים
$\mbb{T}^2 \cong S^1 \times S^1$
וגם
\begin{align*}
\text{.} \mrm{Lie}\prs{\mbb{T}^2} = \set{\pmat{2 \pi i \theta_1 & 0 \\ 0 & 2 \pi i \theta_2}}{\theta_1, \theta_2 \in \mbb{R}}
\end{align*}
ניתן לבדוק שהטופולוגיה שעל
$\mbb{T}^2$
שהגדרנו בבניית המבנה של היריעה החלקה מתלכדת עם הטופולוגיה המושרית ועם הטופולוגיה על
$S^1 \times S^1$.

לכל
$X,Y \in \mfrak{g} \ceq \mrm{Lie}\prs{\mbb{T}^2}$
מתקיים
$\brs{X,Y} = 0$.
לכן כל תת־מרחב של
$\mfrak{g}$
הוא אלגברת לי.
כאן
\[\exp \colon \prs{\mfrak{g},+} \to \mbb{T}^2\]
הוא הומומורפיזם של חבורות על ידי
\[\mbb{R}^2 \to \quot{\mbb{R}^2}{2 \pi \mbb{Z}^2} \cong \mbb{T}^2\]
כשההעתקה הראשונה היא העתקת המנה.

עבור כל
$\alpha \in \mbb{R}$
ניקח תת־אלגברה
\[\text{.} \mfrak{g}_\alpha = \set{\mrm{diag}\prs{2 \pi i \alpha \theta, 2 \pi i \theta}}{\theta \in \mbb{R}}\]
אז
\[\mfrak{g}_\alpha = \mrm{Lie}\prs{G_\alpha}\]
כאשר
\[\text{.} G_\alpha \ceq \set{\mrm{diag}\prs{e^{2 \pi i \alpha \theta}, e^{2 \pi i \theta}}}{\theta \in \mbb{R}}\]
נגדיר
\[\gamma_\alpha\prs{\theta} = \mrm{diag}\prs{e^{2 \pi i \alpha \theta}, e^{2 \pi i \theta}}\]
ואם
$\alpha = \frac{m}{n} \in \mbb{Q}$
מתקיים
$\gamma_\alpha\prs{n} = 1$.
במקרה זה
$G_\alpha$
לולאה סגורה הומיאומורפית ל־%
$S^1$.
אם
$\alpha \notin \mbb{Q}$
נקבל כי
$G_\alpha$
לולאה פתוחה (וצפופה) שהומיאומורפית ל־%
$\mbb{R}$.
אפשר להשתכנע שהטופולוגיה שהגדרנו על
$G_\alpha$
אכן מתאימה לזאת של
$\mbb{R}$,
ובמקרה זה הטופולוגיה המושרית על
$G_\alpha$
בתוך
$\mbb{T}^2$
היא אחרת.

נוכיח שלכל
$\eps > 0$
יש
$\theta$
גדולה כרצוננו כך שמתקיים
$\abs{\gamma\prs{\alpha}\prs{\theta} - I} < \eps$.
נביט בחבורה
\[S^1 = \set{\mrm{diag}\prs{z,1}}{\abs{z} = 1}\]
ובחבורה
\[\text{.} H \ceq G_\alpha \cap S^1 \leq S^1\]
אם
$H$
סופית, קיים
$d \in \mbb{N}$
עבורו
\[\gamma_\alpha\prs{d} = \gamma_\alpha\prs{1}^d = 1\]
שזאת סטירה לאי־רציונליות של
$\alpha$.
לכן
$H$
אינסופית.
מקומפקטיות של
$S^1$
נקבל
$k_1, k_2 \in \mbb{Z}$
עבורם
\[\text{.} \abs{\gamma_\alpha\prs{k_1 - k_2} - I} = \abs{\gamma_\alpha\prs{k_1} - \gamma_\alpha\prs{k_2}} < \eps\]
\end{example}

עבור כל חבורת מטריצות
$G$
החבורה
$G^\circ = \Gamma\prs{\mrm{Lie}\prs{G}}$
היא תת־החבורה של
$G$
הנוצרת על ידי איברים מהצורה
$\exp\prs{X}$.

\begin{proposition}
\begin{enumerate}
\item $G^\circ$
תת־חבורה נורמלית וקשירה ב־%
$G$.
\item אם
$G$
קשירה מתקיים
$G = G^\circ$.
\end{enumerate}
\end{proposition}

\begin{proof}
\begin{enumerate}
\item עבור
\[a = \exp\prs{X_1} \cdot \ldots \cdot \exp\prs{X_k}\]
המסילה
\[\gamma\prs{t} = \exp\prs{t X_1} \cdot \ldots \cdot \exp\prs{t X_k}\]
מסילה רציפה שמקשרת בין
$a,I$.
מתקיים גם
\[\forall g \in G \colon \mrm{Ad}\prs{g}\prs{a} = \exp\prs{\mrm{Ad}\prs{a}\prs{X_1}} \cdot \ldots \cdot \exp\prs{\mrm{Ad}\prs{a} \prs{X_k}} \in \mrm{Lie}\prs{G}\]
לכן
$G^\circ$
נורמלית.
\item
$G^\circ$
מכילה סביבה פתוחה
$U$
של
$I$,
למשל
$U = B_{I,\eps}$.
מתקיים
\[G^\circ = \bigcup_{g \in G^\circ} g \cdot U\]
וזה איחוד של קבוצות פתוחות. לכן
$G^\circ$
פתוחה.
תת־חבורה פתוחה של חבורה טופולוגית היא גם סגורה, כי הקוסטים שלה ב־%
$G$
פתוחים.
לכן
$G$
פתוחה וסגורה, ולכן קשירה. לכן
$G = G^\circ$.
באופן דומה, איברי
$G/G^\circ$
פתוחים וסגורים ולכן הינם רכיבי הקשירות של
$G$.
\end{enumerate}
\end{proof}

\begin{corollary}
$G^\circ$
רכיב הקשירות של
$I \in G$.
\end{corollary}

\begin{corollary}
$G \leq \mrm{GL}_n$
קשירה בטופולוגיה שהגדרנו אם ורק אם
$\Gamma\prs{\mrm{Lie}\prs{G}} = G$.
\end{corollary}

\begin{remark}
בעתיד נוכיח שאם
$G \leq \mrm{GL}_n$
היא סגורה בטופולוגיה המושרית, הטופולוגיה שהגדרנו מתלכדת עם הטופולוגיה היחסית.
\end{remark}

\begin{example}
לכל
$X \in M_n\prs{\mbb{R}}$
מתקיים
\[\text{.} \det\prs{\exp\prs{X}} = e^{\tr\prs{X}} > 0\]
אז
\[\text{.}\mrm{GL}_n\prs{\mbb{R}}^\circ \subseteq \mrm{GL}_n\prs{\mbb{R}}^+ \ceq \set{A \in \mrm{GL}_n\prs{\mbb{R}}}{\det\prs{A} > 0}\]
למעשה, ראינו בתרגיל ש־%
$\mrm{GL}_n\prs{\mbb{R}}^+$
קשירה.
לכן יש שוויון
\[\text{.} \mrm{GL}_n\prs{\mbb{R}}^\circ = \mrm{GL}_n\prs{\mbb{R}}^+\]
\end{example}

\begin{example}
ראינו שקיימת
$g \in \mcal{O}\prs{n}$
עבורה
\[\text{.} \mcal{O}\prs{n} = \mrm{SO}\prs{n} \cup g \cdot \mrm{SO}\prs{n}\]
ראינו גם שמתקיים
\[\mfrak{so}_3 = \mrm{Lie}\prs{\mrm{SO}\prs{3}}\]
ושמתקיים
\[\Gamma\prs{\mfrak{so}_3} = \mrm{SO}\prs{3}\]
לכן
\[\text{.} \mrm{SO}\prs{3} = \mrm{SO}\prs{3}^\circ = \mcal{O}\prs{3}^\circ\]
\end{example}

\begin{example}
תהי
\[\text{.} \mrm{O}\prs{1,1} \ceq \set{A \in \mrm{GL}_n\prs{\mbb{R}}}{A \pmat{1 & 0 \\ 0 & - 1} A^t = \pmat{1 & 0 \\ 0 & - 1}}\]
עבור
$A \in \mrm{O}\prs{1,1}$
מתקיים
\[\det\prs{A} \cdot \det \pmat{1 & \\ & -1} \cdot \det\prs{A^t} = \det\pmat{1 & \\ & -1}\]
לכן
$\det\prs{A}^2 = 1$
לכן
$\det\prs{A} = \pm 1$.
מתקיים
$\pmat{1 & 0 \\ 0 & - 1} \in \mrm{O}\prs{1,1}$
וגם
\[\mrm{O}\prs{1,1} = \mrm{SO}\prs{1,1} \cup \pmat{1 & 0 \\ 0 & - 1} \mrm{SO}\prs{1,1}\]
עבור
\[\text{.} \mrm{SO}\prs{1,1} \ceq \mrm{O}\prs{1,1} \cap \mrm{SL}_n\prs{\mbb{R}}\]

החבורה
$\mrm{SO}\prs{1,1}$
אינה קשירה.
מתקיים
\[\mfrak{g} = \mrm{Lie}\prs{\mrm{O}\prs{1,1}}\]
ואם נגזור את
$\gamma \colon \prs{-\eps, \eps} \to \mrm{O}\prs{1,1}$
עם
$\gamma\prs{0} = I$
נקבל
\[\text{.}\gamma'\prs{0} \pmat{1 & 0 \\ 0 & - 1} + \pmat{1 & 0 \\ 0 & - 1} \gamma'\prs{0}^t = 0\]
אז
\begin{align*}
\text{.} \mfrak{g} = \set{X \in M_2\prs{\mbb{R}}}{A \pmat{1 & 0 \\ 0 & - 1} + \pmat{1 & 0 \\ 0 & - 1} A^t = 0} = \set{\pmat{0 & a \\ a & 0}}{a \in \mbb{R}}
\end{align*}
זאת חבורה חד־מימדית.
נקבל
\[\mrm{O}\prs{1,1}^\circ = \mrm{SO}\prs{1,1}^\circ = \set{\exp\pmat{0 & a \\ a & 0}}{a \in \mbb{R}}\]
והאיברים בחבורה זאת נקראים סיבובים היפרבוליים.

נסמן
\[\text{.} C \ceq \frac{1}{2} \pmat{1 & - 1 \\ 1 & 1} \in \mrm{SO}\prs{2}\]
אז
\begin{otherlanguage}{english}
\begin{align*}
\exp\pmat{0 & a \\ a & 0} &=
\exp\prs{C \pmat{2a & \\ & -2a} C^{-1}}
\\&=
C \pmat{e^{2 a} & \\ & e^{-2a}} C^{-1}
\\ \text{.} \hphantom{\exp\pmat{0 & a \\ a & 0}} &=
\pmat{\cosh\prs{2a} & \sinh\prs{2a} \\ \sinh\prs{2a} & \cosh\prs{2a}}
\end{align*}
\end{otherlanguage}

בבסיס המלכסן אלו העתקות מהצורה
$\pmat{t & 0 \\ 0 & t^{-1}}$
עבור
$t > 0$.
העתקות אלה מקיימות
\[xy = \prs{tx}\prs{t^{-1}y}\]
ולכן משמרות את ההיפרבולה.

מתקיים למשל
\begin{align*}
-I \in \mrm{SO}\prs{1,1} \setminus \mrm{SO}\prs{1,1}^\circ
\end{align*}
ולמעשה
\begin{align*}
\brs{\mrm{SO}\prs{1,1} : \mrm{SO}\prs{1,1}^\circ} &= 2 \\
\text{.}\brs{\mrm{O}\prs{1,1} : \mrm{O}\prs{1,1}^\circ} &= 4
\end{align*}
בדקו זאת וחשבו בנוסף מהי החבורה הסופית
$\quot{\mrm{O}\prs{1,1}}{\mrm{O}\prs{1,1}^\circ}$.
\end{example}

\begin{definition}
לכל
$G$,
החבורה
$\quot{G}{G^\circ}$
נקראת ה־%
\textenglish{component group}
של
$G$.
\end{definition}

\begin{example}
עבור
$G \ceq \mrm{GL}_n\prs{\mbb{Q}}$
מתקיים
$\quot{G}{G^\circ} \cong G$.
\end{example}

\subsection{חבורות לי אבליות}

\begin{proposition}
תהי
$G \leq \mrm{GL}_n\prs{k}$
קשירה.
$G$
אבלית אם ורק אם
$\mrm{Lie}\prs{G}$
אבלית (כלומר
$\brs{X,Y} = 0$
לכל
$X,Y \in \mrm{Lie}\prs{G}$).
\end{proposition}

\begin{proof}
נניח כי
$G$
אבלית ותהיינה
$X,Y \in \mrm{Lie}\prs{G}$.
מתקיים
\begin{align*}
\text{.} \gamma\prs{s,t} \ceq \Ad\prs{\exp\prs{tX}} \prs{\exp\prs{sY}} = \exp\prs{sY}
\end{align*}
נגזור ב-%
$0$
לפי
$s$
ונקבל
\begin{align*}
\text{.} \exp\prs{tX} \cdot Y \cdot \exp\prs{-tX} = \Ad\prs{\exp\prs{tX}}\prs{Y} = Y
\end{align*}
נגזור ב־%
$t = 0$
ונקבל
$XY - YX = 0$.

בכיוון השני, נניח כי
$\mrm{Lie}\prs{G}$
אבלית וכי
$G$.
אז
\[\forall X,Y \in \mrm{Lie}\prs{G} \colon \exp\prs{X} \exp\prs{Y} = \exp\prs{Y}\exp\prs{X}\]
אז
$\Gamma^\circ =\Gamma\prs{\mrm{Lie}}\prs{G}$
אבלית, ומתקיים
$G^\circ = G$
כי
$G$
קשירה.
\end{proof}

\begin{remark}
נניח כי
$G$
קשירה ואבלית. אז
$\exp \colon \mfrak{g} \to G$
הומומורפיזם מ־%
$\prs{\mfrak{g}, +}$
ל־%
$\prs{G, \cdot}$.
אז אפשר לכתוב
\[G = \quot{\prs{\mfrak{g}, +}}{L}\]
כאשר
\[\text{.} L \ceq \ker \exp = \set{x \in \mfrak{g}}{\exp\prs{X} = 1}\]

לפעמים
$G$
כזאת נקראת טורוס (מוכלל).
\end{remark}

%19.11.2020

\begin{remark}
כל חבורה סופית
$G$
אפשר לשכן בתוך
$S_n$.
יש גם שיכון
\[i \colon S_n \to \mrm{GL}_n\prs{\mbb{R}}\]
על ידי מטריצות פרמוטציה על הבסיס הסטנדרטי.

באופן זה אפשר לחשוב על כל חבורה סופית כחבורת מטריצות. ל־%
$G$
סופית נקבל
$\mrm{Lie}\prs{G} = \set{0}$.
לכן הטופולוגיה של
$G$
כחבורת לי היא דיסקרטית.

נשים לב ש־%
$\mrm{Lie}\prs{G}$
אבלית, בעוד
$G$
לאו דווקא אבלית. זה מתאפשר כי
$G^\circ = \set{\id} \subsetneq G$.
על כל חבורה סופית אפשר לחשוב כחבורת לי לא־קשירה עם
$G^\circ$
טריוויאלי.
\end{remark}

\begin{definition}[טורוס
$n$%
־מימדי]
נגדיר
\[\text{.} \mbb{T}^n \ceq \set{\mrm{diag}\prs{e^{i \theta_1}, \ldots, e^{i \theta_n}}}{\prs{\theta_i}_{i \in [n]} \subseteq \mbb{R}} \leq \mrm{GL}_n\prs{\mbb{C}}\]
\end{definition}

\begin{remark}
$\prs{S^1}^n \cong \mbb{T}^n$
כחבורות לי, וההעתקה
\begin{align*}
\exp \colon \mbb{R}^n \cong \mrm{Lie}\prs{\mbb{T}^n} &\to \mbb{T}^n \\
\end{align*}
היא הומומורפיזם על.
אז
\[\text{.}\mbb{T}^n \cong \quot{\mbb{R}^n}{\ker\prs{\exp}} = \quot{\mbb{R}^n}{\mbb{Z}^n}\]
\end{remark}

\begin{example}
$\prs{\mbb{R}^n, +}$
זאת חבורת מטריצות אבלית קשירה על ידי שיכון כמטריצות
$\pmat{& & & * \\ & I_n & & \vdots \\ & & & * \\ & 0 & & 1}$.
לחילופין, מתקיים
\begin{align*}
\prs{\prs{\mbb{R}^\times}^n}^\circ &\cong \set{\mrm{diag}\prs{e^{\theta_1}, \ldots, e^{\theta_n}}}{\prs{\theta_i}_{i \in [n]} \subseteq \mbb{R}} \\
\text{.} \prs{\mbb{R}^\times}^n &\cong \set{\mrm{diag}\prs{x_1, \ldots, x_n}}{\forall i \in [n] \colon x_i \neq 0}
\end{align*}
אז
\begin{align*}
\mbb{R} &\cong \prs{\mbb{R}^\times}^\circ \\
x &\mapsto e^x \\
\log\prs{y} &\mapsfrom y
\end{align*}
איזומורפיזם.
נקבל כי
$\exp \colon \mbb{R}^n \to \prs{\prs{\mbb{R}^\times}^n}^\circ$
איזומורפיזם.
\end{example}

\begin{proposition}
תהי
$G$
חבורה קשירה אבלית. אז
\[\exp \colon \mrm{Lie}\prs{G} \to G\]
הוא הומומורפיזם על ומתקיים
\[G \cong \quot{\mrm{Lie}\prs{G}}{\ker\prs{\exp}}\]
כאשר
$\ker\prs{\exp}$
תת־חבורה דיסקרטית בתוך
$\prs{\mrm{Lie}\prs{G}, +}$.
\end{proposition}

\begin{theorem}\label{theorem:quotient_by_discrete}
עבור
$V$
מרחב וקטורי ממשי ו־%
$L \leq V$
תת־חבורה דיסקרטית יש
$k,n$
עבורם
\[\text{.} \quot{V}{L} \cong \mbb{T}^k \times \mbb{R}^n\]
\end{theorem}

\begin{corollary}
כל חבורת מטריצות אבלית קשירה היא מהצורה
$\mbb{T}^k \times \mbb{R}^n$.
\end{corollary}

\begin{lemma}\label{lemma:groups_are_lattices}
תהי
$L \leq V$
תת־חבורה דיסקרטית של מרחב וקטורי ממשי. אז קיימים וקטורים בלתי־תלויים לינארית
$\prs{u_i}_{i \in \brs{n}}$
עבורם
\[\text{.} L = \spn_{\mbb{Z}}\prs{u_i}_{i \in \brs{n}}\]
\end{lemma}

\begin{proof}
נבנה באינדוקציה וקטורים
$\prs{u_i}_{i \in \brs{r}}$
בלתי־תלויים לינארית ב־%
$V$
עבורם
\[\text{.} L \cap \spn_{\mbb{R}}\prs{u_1, \ldots, u_r} = \spn_{\mbb{Z}} \prs{u_1, \ldots, u_r}\]
אם
$L \subseteq \spn_{\mbb{R}} \prs{u_1, \ldots, u_r}$
סיימנו.
אחרת קיים
$u \in \mbb{L}$
בלתי־תלוי לינארית ב־%
$\prs{u_1, \ldots, u_r}$.
תהי
\[P = \set{\sum_{i \in \brs{r}} \alpha_i u_i + \beta u}{\substack{\prs{\alpha_i}_{i \in [r]} \subseteq \brs{0,1} \\ \beta \in \brs{0,1}}} \subseteq V\]
שהינה קומפקטית.
אז
\[u \in P \cap L \cong \set{0}\]
וזאת קבוצה סופית כחיתוך של קומפקטית ודיסקרטית.
תהי
$v \in P \cap L$
עם
$\beta \neq 0$
מינימלי.
אז לכל
$x \in L \cap \spn_{\mbb{R}}\prs{u_1, \ldots, u_r, u}$
קיים
$n \in \mbb{Z}$
עבורו
\[\text{.} x - nv \in L \cap \spn_{\mbb{R}}\prs{u_1, \ldots, u_R} = \spn_{\mbb{Z}}\prs{u_1, \ldots, u_r}\]
(אחרת היינו מוצאים
$n \in \mbb{Z}$
שיתן סתירה למינימליות של
$\beta$).
\end{proof}

\begin{proof}[\ref{theorem:quotient_by_discrete}]
מ־%
\ref{lemma:groups_are_lattices}
יש ל־%
$L$
בסיס
$\prs{u_i}_{i \in \brs{m}}$.
נשלים אותו לבסיס
$\prs{u_i}_{i \in \brs{k}}$
של
$V$
ואז
$V \cong \mbb{R}^k$
לפי הבסיס. נקבל
\[L \cong \mbb{Z}^m \times \set{0} \leq \mbb{R}^m \times \mbb{R}^{k-m}\]
ואז
\[\text{.} \quot{V}{L} \cong \quot{\mbb{R}^m}{\mbb{Z}^m} \times \mbb{R}^{k-m}\]
\end{proof}

\begin{remark}
לפעמים, בהקשר של חבורות אלגבריות, חבורות אלגבריות קשירות נקראות טורוסים.
\end{remark}

\subsection{חזרה להתאמת לי}

נרצה שלכל אלגברת לי נתונה
$\mfrak{g} \leq M_n$
יתקיים
\[\text{.} \mfrak{g} = \mrm{Lie}\prs{\Gamma\prs{\mfrak{g}}}\]
לכל
$X \in \mfrak{g}$
אנו יודעים
$\exp\prs{tX} \in \Gamma\prs{\mfrak{g}}$.
לכן לפי ההגדרה נקבל
\[X \in \mrm{Lie}\prs{\Gamma\prs{\mfrak{g}}}\]
ולכן
\[\text{.} \mfrak{g} \subseteq \mrm{Lie}\prs{\Gamma\prs{\mfrak{g}}}\]
נרצה לתאר את
$\Gamma\prs{\mfrak{g}}$
באופן יותר קונקרטי ונרצה להראות במובן מסוים שהיא לא גדולה מדי.

מכך ש־%
$\exp\prs{X} = \exp\prs{\frac{1}{k} X}^k$
נובע שלכל סביבה
$U \in \mfrak{g}$
של
$0$
מתקיים
\[\text{.} \Gamma\prs{\mfrak{g}} = \bigcup_{k \in \mbb{N}_+} \exp\prs{U}^k = \bigcup_{k \in \mbb{N}_+} \exp\prs{\bar{U}}^k\]
נרצה להיפטר מהחזקות ולשם כך נוכל לכתוב
\[\text{.} \Gamma\prs{\mfrak{g}} = \bigcup_{g \in \Gamma\prs{\mfrak{g}}} g \cdot \exp\prs{\bar{U}}\]

נרצה להיות מסוגלים לכתוב את
$\Gamma\prs{\mfrak{g}}$
כאיחוד בן מניה באופן דומה.
זה מתאפשר, וינבע מהמהות של נוסחת
\textenglish{Baker-Cambell-Hausdorff},
מזה שקיימת העתקה
\[C \colon \mfrak{g} \times \mfrak{g} \to \mfrak{g}\]
מוגדרת בסביבת אפס.

%LECTURE 10 

ראינו כי עבור
$G$
חבורת מטריצות קשירה מתקיים
\[\text{.} G = \Gamma\prs{\mrm{Lie}\prs{G}}\]
נרצה בעצם תיאור כלשהו של כל
$G$
על ידי
$\exp$.
ראינו שמתקיים
\[\text{.} \Gamma\prs{\mfrak{g}} = \bigcup_{k \in \mbb{N}_+} \exp\prs{U}^k\]
עבור
$U \subseteq \mfrak{g}$
סביבה פתוחה של
$0$.

ניזכר שעבור
$X,Y \in \mfrak{g}$
מתקיים
\[C\prs{X,Y} = \log\prs{\exp\prs{X} \exp\prs{Y}} \in \mfrak{g}\]
כאשר זה מוגדר.
נגדיר
\[I_X\prs{Y} \ceq C\prs{X,Y}\]
ואז
$I_0\prs{Y} = Y$.
לכן מתקיים
\begin{align*}
\text{.} \prs{\diff\prs{I_0}}_0 = \id
\end{align*}
מרציפות נקבל כי
$\prs{\diff I_X}_0$
הפיכה עבור
$X$
מספיק קטן.
ממשפט הפונקציה ההפוכה קיימת סביבה
$U \subseteq \mfrak{g}$
של
$0$
כך שעבור
$X$
מספיק קטן הקבוצה
\[I_X\prs{U} = C\prs{X,U}\]
פתוחה (בדקו למה
$U$
אחידה לכל
$X$
מספיק קטן).

תהי
\[V \ceq C\prs{U,U}\]
ותהי
\[\text{.} V' = C\prs{\bar{U}, \bar{U}}\]
אז
$V'$
קומפקטית כתמונה רציפה של קבוצה קומפקטית
$\bar{U} \times \bar{U}$.
מתקיים
\[V' \subseteq \bigcup_{X \in V'} C\prs{X,U}\]
אם בוחרים
$V'$
מספיק קטנה, וזה איחוד של קבוצות פתוחות ולכן פתוח.
מקומפקטיות, קיימת קבוצה סופית
$\set{X_i}{i \in [n]} \subseteq V'$
כך שמתקיים
\[\text{.} V' \subseteq \bigcup_{i \in [n]} C\prs{X_i, U}\]
לכל
$i \in [n]$
נסמן
$a_i = \exp\prs{X_i}$.
על ידי לקיחת אקספוננט נקבל
\[\text{.} \exp\prs{\bar{U}} \exp\prs{\bar{U}} \subseteq \bigcup_{i \in [n]} a_i \cdot \exp\prs{\bar{U}}\]

אם נמשיך באינדוקציה נקבל
\[\bigcup_{k=1}^{\infty} \exp\prs{\bar{U}}^k \subseteq \bigcup_b b \cdot \exp\prs{\bar{U}}\]
כאשר
$b$
רץ על מילים באותיות
$\set{a_i}{i \in [n]}$.

\begin{theorem}
לכל אלגברת לי
$\mfrak{g}$
של מטריצות מתקיים
$\mfrak{g} = \mrm{Lie}\prs{\Gamma\prs{\mfrak{g}}}$.
\end{theorem}

\begin{proof}
עבור
$X \in \mfrak{g}$
יש מסילה
\[\gamma\prs{t} = \exp\prs{tx} \in \Gamma\prs{\mfrak{g}}\]
עם
\[\text{.} X = \gamma'\prs{0} \in \mrm{Lie}\prs{\Gamma\prs{\mfrak{g}}}\]
לכן
$\mfrak{g} \subseteq \mrm{Lie}\prs{\Gamma\prs{\mfrak{g}}}$.

להכלה ההפוכה, נראה שקיימת סביבה פתוחה
$U \subseteq \mfrak{g}$
של
$0$
וקיים
$c \in \Gamma\prs{\mfrak{g}}$
כך שהקבוצה
$c \exp\prs{U}$
פתוחה בטופולוגיה של
$\Gamma\prs{\mfrak{g}}$.
אכן,
$\Gamma\prs{\mfrak{g}}$
מרחב טופולוגי קומפקטי מקומית והאוסדורף שהוא איחוד בן מנייה של קבוצות סגורות מהצורה
$c \cdot \exp\prs{\bar{U}}$.
לפי משפט הקטגוריה של בייר, אחת הקבוצות
$c \cdot \exp\prs{\bar{U}}$
עם פנים לא ריק.
אפשר להניח (בדקו!) ש־%
$c \cdot \exp\prs{U}$
צפופה ופתוחה בתוך
$c \cdot \exp\prs{\bar{U}}$
ומכאן
$c \cdot \exp\prs{U}$
פתוחה בעצמה ב־%
$\Gamma\prs{\mfrak{g}}$.

אפשר להניח ש־%
$c \cdot \exp\prs{U} \subseteq B\prs{c, \eps}$
עבור
$c,\eps$
כך שמתקיים שהאקספוננט הומאומורפיזם לתמונה.
אז
$U \subseteq \mrm{Lie}\prs{\Gamma\prs{\mfrak{g}}}$
פתוחה כי
$\exp$
הומיאומורפיזם, אבל אם
$\mfrak{g}$
תת־מרחב ממימד קטן יותר זה לא יכול לקרות (כי תת־מרחב לא יכול להכיל קבוצה פתוחה).
\end{proof}

\begin{theorem}
בהינתן חבורת מטריצות
$G$
קיימת התאמה חד־חד ערכית בין אוסף כל התת־חבורות הקשירות
$H \leq G$
לבין אוסף כל התת־אלגבראות לי
$\mfrak{h} \leq \mrm{Lie}\prs{G}$.
\end{theorem}

\begin{proof}
זה נובע מכך שההתאמה שהראנו בין אלגבראות לי לחבורות שומרת על סדר הכלה.
\end{proof}

\begin{proposition}
תהי
$G \leq \mrm{GL}_n\prs{k}$
ונניח שמתקיים
$G = G^\circ$.
יש התאמ בין תת־אלגבראות לי
$\mfrak{h} \leq \mrm{Lie}\prs{G}$
שמקיימות
\[\brs{\mrm{Lie}\prs{G}, \mfrak{h}} \subseteq \mfrak{h}\]
(ונקראות אידאלים), לבין תת-חבורות נורמליות קשירות
$H \ideal G$.
\end{proposition}

\begin{proof}
נניח כי
$H$
ו־%
$\mfrak{h}$
קשורות בהתאמת לי.

נניח כי
$H$
ת"ח נורמלית. אז
\begin{align*}
\forall X \in \mrm{Lie}\prs{G} \forall Y \in \mfrak{h} \colon \brs{X,Y} &= \frac{\diff}{\diff t} \rest{\prs{\Ad \prs{\exp\prs{tx}}\prs{Y}}}{t=0}
\end{align*}
כי
\begin{align*}
\frac{\diff}{\diff t} \prs{\exp\prs{tx} Y \exp\prs{-tX}} = X \exp\prs{tX} Y \exp\prs{-tX} + \exp\prs{tX} Y \exp\prs{-tX} \prs{-X}
\end{align*}
וב־%
$t = 0$
מקבלים
$XY - YX$.
מתקיים
\[\Ad\prs{\exp\prs{tX}} \colon H \to H\]
וגם
\[\Ad\prs{\exp\prs{tX}}\prs{\mfrak{h}} \subseteq \mfrak{h}\]
לכן מהנ"ל
$\brs{X,Y} \in \mfrak{h}$.

בכיוון השני, אם
$\mfrak{h}$
אידאל, מספיק לבדוק שמתקיים
\[\text{.}\forall X \in \mrm{Lie}\prs{G} \forall Y \in \mfrak{h} \colon \Ad\prs{\exp\prs{X}} \prs{\exp\prs{Y}} \in H\]
מתקיים
$G = G^\circ, H = H^\circ$
ולכן
$G,H$
נוצרות על ידי
$\exp$.
לכן
\begin{align*}
\Ad\prs{\exp\prs{X}} \prs{\exp\prs{Y}} &=
\exp\prs{\Ad\prs{\exp\prs{X}}\prs{Y}} \\&=
\exp\prs{\exp\prs{\ad\prs{X}}\prs{Y}}
\end{align*}
אבל כיוון שמתקיים
$\ad\prs{X} \prs{Y} \in \mfrak{h}$
נובע
$\exp \prs{\ad\prs{X}}\prs{Y} \in \mfrak{h}$,
ולכן
\[\text{.} \exp\prs{\exp\prs{\ad\prs{X}}\prs{Y}} \in H\]
\end{proof}

\section{הומומורפיזמים}

\subsection{הגדרה}

בהינתן הומומורפיזם גזיר
$\phi \colon G \to H$
בין חבורות מטריצות, הנגזרת
\[\prs{\diff \phi}_I\]
היא העתקה לינארית
\[\text{.} \mrm{Lie}\prs{G} \to \mrm{Lie}\prs{H}\]
ביתר פירוט, אם
\[\gamma \colon \prs{-\eps, \eps} \to G\]
מקיימת
\begin{align*}
\gamma\prs{0} &= I \\
\gamma'\prs{0} &= X \in \mrm{Lie}\prs{G}
\end{align*}
אז
\[\phi\prs{\gamma\prs{0}} = \phi\prs{I} = I \in H\]
ומתקיים
\[\text{.} \prs{\phi \circ \gamma}' \prs{0} = \prs{\diff \phi}_I \prs{\gamma'\prs{0}} = \prs{\diff \phi}_I \prs{X} \in \mrm{Lie}\prs{H} \text{,}\]
אז
$\phi$
מעבירה וקטורים משיקים ב־%
$G$
לוקטורים משיקים ב־%
$H$.
נסמן העתקה זאת בין המשיקים
\[ \text{.} \diff \phi \colon \mrm{Lie}\prs{G} \to \mrm{Lie}\prs{H}\]

\begin{proposition}
$\diff \phi$
הוא הומומורפיזם של אלגבראות לי. כלומר, מתקיים
\[\forall X,Y \in \mrm{Lie}\prs{G} \colon \brs{\diff \phi\prs{X}, \diff \phi\prs{Y}} = \diff\phi\prs{\brs{X,Y}}\]
וגם
\[\text{.} \forall X \in \mrm{Lie}\prs{G} \colon \phi\prs{\exp\prs{X}} = \exp\prs{\diff \phi\prs{X}}\]

את המשוואה השנייה נציג על ידי הדיאגרמה הקומוטטיבית הבאה.

\begin{otherlanguage}{english}
\[
\begin{tikzcd}
G \arrow[r, "\phi"] & H \\
\mrm{Lie}\prs{G} \arrow[u, "\exp"] \arrow[r, swap, "\diff \phi"] & \mrm{Lie}\prs{H} \arrow[u, swap, "\exp"]
\end{tikzcd}
\]
\end{otherlanguage}

\end{proposition}

\begin{proof}
לכל
$X \in \mrm{Lie}\prs{G}$
מתקיים
\begin{align*}
\frac{\diff}{\diff t} \prs{\phi\prs{\exp\prs{tX}}} &= \frac{\diff}{\diff s}\left.\prs{\phi\prs{\exp\prs{\prs{t+s}X}}}\right._{s=0}
\\&=
\frac{\diff}{\diff s} \left.\prs{\phi\prs{\exp\prs{tX}} \phi\prs{\exp\prs{sX}}}\right._{s=0}
\\&= \phi\prs{\exp\prs{tX}} \cdot \frac{\diff}{\diff s} \left. \prs{\phi\prs{\exp\prs{sX}}} \right._{s=0}
\\&= \phi\prs{\exp\prs{tX}} \cdot \diff \phi\prs{X}
\end{align*}
ולפי טענת המד"ר בתחילת הקורס נובע
\[\text{.} \phi\prs{\exp\prs{tX}} = \exp\prs{t \cdot \diff \phi\prs{X}}\]
כאשר
$t = 1$
אנו מקבלים את המשוואה.

נראה כעת כי
$\diff \phi$
הוא הומומורפיזם.
לכל
$X,Y \in \mrm{Lie}\prs{G}$
מתקיים
\begin{align*}
\phi\prs{\Ad\prs{\exp\prs{tX}} \prs{\exp\prs{sY}}}
&= \Ad\prs{\phi\prs{\exp\prs{tX}}}\prs{\phi\prs{\exp\prs{sY}}}
\\&= \Ad\prs{\exp\prs{t \diff \phi\prs{X}}} \prs{\exp\prs{s \diff \phi\prs{Y}}}
\end{align*}
ולאחר גזירה ב־%
$s=0$
נקבל
\[\text{.} \diff \phi\prs{\Ad\prs{\exp\prs{tX}} \prs{Y}} = \Ad\prs{\exp\prs{t \diff \phi\prs{X}}} \prs{\diff \phi\prs{Y}}\]
נגזור כעת ב־%
$t = 0$
ונקבל שאגף שמאל שווה
\[\text{.}\diff \phi \prs{\frac{\diff}{\diff t} \prs{\Ad\prs{\exp\prs{tX}}\prs{Y}}_{t=0}} = \diff \phi\prs{\brs{X,Y}}\]
אגף ימין שווה
$\brs{\diff \phi\prs{X}, \diff \phi\prs{Y}}$
באותו טיעון, ולכן נקבל את התוצאה.
\end{proof}

\begin{remark}
למעשה, קיבלנו ש־%
\[\mrm{Lie} \colon \catname{Mat-Grp} \to \catname{Lie-Alg}\]
הוא פנקטור מהקטגוריה של חבורות לקטגוריה של אלגבראות לי
כשעל מורפיזמים הוא מוגדר
$\mrm{Lie}\prs{\phi} = \diff \phi$.
העובדה
\[\mrm{Lie}\prs{\phi_1 \circ \phi_2} = \mrm{Lie}\prs{\phi_1} \circ \mrm{Lie}\prs{\phi_2}\]
היא כלל השרשרת.
\end{remark}

נרצה לדעת מתי
\[\diff \colon \hom\prs{G,H} \to \hom\prs{\Lie\prs{G}, \Lie\prs{H}}\]
הוא חד־חד ערכי, ומתי הוא על. אם הוא חד־חד ערכי ועל, הדבר קרוב להיות שקילות בין קטגוריות.

כדי לבדוק מתי
$\diff$
חד־חד ערכי, נבדוק מתי
$\phi$
נקבע ביחידות על ידי
$\diff \phi$.
מתקיים
\[\phi\prs{\exp\prs{X}} = \exp\prs{\diff \phi\prs{X}}\]
ולכן
$\phi$
נקבע ביחידות על
$G^\circ = \Gamma\prs{\Lie\prs{G}}$.
לכן אם
$G$
קשירה נובע ש־%
$\diff$
חד־חד ערכי.
בפרט, אם נביט על תת־קטגוריה של חבורות מטריצות קשירות נקבל שהפנקטור
$\Lie$
חד־חד ערכי על מורפיזמים, כלומר
\emph{נאמן}.

נשאל כעת מתי
$\diff$
על. זאת השאלה מתי אפשר להרים הומומורפיזם של אלגבראות לי. אם
$\psi \colon \Lie\prs{G} \to \Lie\prs{H}$,
מתי קיים
$\phi \colon G \to H$
כך ש־%
$\diff \phi = \psi$.
לזה יש תשובה טופולוגית, והתשובה היא כן בפרט כאשר
$G$
פשוטת קשר.

\begin{remark}
תהי
$G$
חבורת מטריצות ונסמן
$\mfrak{g} = \Lie\prs{G}$.
אם
$g \in G$
אפשר לחשוב על
$\Ad\prs{g} \colon G \to G$
כאוטומורפיזם של
$G$.
נקבל העתקה
\[ \text{.} \Ad \colon G \to \mrm{Aut}\prs{G}\]
נוכל להרכיב ולקבל
\[\text{.} G \xrightarrow[\Ad]{} \aut\prs{G} \xrightarrow[\diff]{} \aut\prs{\mfrak{g}} \leq \mrm{GL}\prs{\mfrak{g}}\]
נסמן הרכבה זאת (\textenglish{with abuse of notation})
\[\text{.} \Ad\prs{g} = \prs{\diff \circ \Ad}\prs{g} \in \aut\prs{\mfrak{g}}\]
אם נסמן זאת
$\Ad_{\mfrak{g}}$
נקבל
\[\text{.} \Ad_{\mfrak{g}} \colon G \to \aut\prs{\mfrak{g}}\]
נוכל גם לגזור ולקבל
\[\text{.}\Ad_{\mfrak{g}}\prs{g} = \diff\prs{\Ad_G\prs{g}} \colon \mfrak{g} \to \mfrak{g}\]

כעת,
$\aut\prs{\mfrak{g}}$
חבורת מטריצות ומתקיים
\[\text{.}\diff \prs{\Ad_{\mfrak{g}}} \colon \mfrak{g} \to \Lie\prs{\aut\prs{\mfrak{g}}} \leq \endo\prs{\mfrak{g}}\]
\emph{%
העתקה זאת מסומנת
$\ad$}
ומתקיים
\[\Ad_{\mfrak{g}}\prs{\exp\prs{X}} = \exp\prs{\ad\prs{X}}\]
לכל
$X \in \mfrak{g}$.
\end{remark}

קיבלנו חבורת מטריצות חדשה
\[\text{.} \quot{G}{\ker\prs{\Ad_{\mfrak{g}}}} \cong \Ad_{\mfrak{g}}\prs{G} \leq \mrm{GL}\prs{\mfrak{g}}\]
את הגרעין הזה אנחנו לא מבינים ישר. מתקיים
\[\ker\prs{\Ad_G} = Z\prs{G}\]
וגם
$\Ad_{\mfrak{g}} = \diff \circ \Ad_G$
ולכן אנחנו כן יודעים
$Z\prs{G} \leq \ker\prs{\Ad_{\mfrak{g}}}$.
על פניו, מלחתכילה לא ידוע לנו ש־%
$\quot{G}{Z\prs{G}}$
חבורת מטריצות, אך נראה זאת.

%LECTURE 11 (partly above)

\begin{remark}

דיברנו על גזירות וחלקות של הומומורפיזמים
$\phi \colon G \to H$
כאשר
$G,H$
חבורות מטריצות, והגדרנו את
$\diff \phi$
כנגזרת של
$\phi$
ב־%
$I$.
הכוונה היא שאם ניקח סביבה
$B_{I,\eps} \subseteq G$
של
$I$
שמוגדרת על ידי
\[\text{,}\set{\exp\prs{X}}{\substack{X \in \Lie\prs{G} \\ \norm{X} < \eps}}\]
וניקח סביבה
$B_{I,\eps}' \subseteq H$
של
$I$,
נקבל שההעתקת המעבר
$\rest{\exp}{V}^{-1} \circ \phi \circ \rest{\exp}{U}$
גזירה/חלקה ו־%
$\diff \phi$
הנגזרת של הרכבה זאת.
בפרט,
$\diff \phi$
לא תלויה במימוש של
$G$
כחבורת מטריצות.
למעשה גם הטופולוגיה שהגדרנו על
$G$,
והחלקות של
$\phi$
אינן תלויות במימוש.

בהמשך נגדיר חבורות לי כלליות שאינן בהכרח חבורות מטריצות, באופן שאינו תלוי במימוש.
\end{remark}

%LECTURE 12

עולות לנו כמה שאלות על ההתאמה
\[\text{.}\Lie \colon \hom\prs{G,H} \to \hom\prs{\Lie\prs{G}, \Lie\prs{H}}\]

\begin{question}
מה אומרות התכונות של
$\diff \phi$
על אלו של
$\phi$?
אם
$\diff \phi$
איזומורפיזם, מה זה אומר על
$\phi$?
\end{question}

\begin{question}
בהינתן
$f \in \hom\prs{\Lie\prs{G}, \Lie\prs{H}}$,
האם אפשר להרים אותו להומומורפיזם
$\phi \colon G \to H$
שמקיים
$\diff \phi = f$?
\end{question}

\begin{example}
נסתכל על
$G = \prs{\mbb{C}, +}, H = \prs{\mbb{C}^\times, \cdot}$.
אז
\begin{align*}
G &\cong \set{\pmat{1 & a \\ 0 & 1}}{a \in \mbb{C}} \leq \mrm{GL}_2\prs{\mbb{C}}
\text{.} H &\cong \mrm{GL}_1\prs{\mbb{C}}
\end{align*}
נקבל
\begin{align*}
\Lie\prs{G} &= \set{\pmat{0 & a \\ 0 & 0}}{a \in \mbb{C}} \\
\Lie\prs{H} &= \mbb{C}
\end{align*}
ואכן מתקיים
\[\text{.}\exp\pmat{0 & a \\ 0 & 0} = \pmat{1 & 0 \\ 0 & 1} + \pmat{0 & a \\ 0 & 0} + 0 = \pmat{1 & a \\ 0 & 1}\]
תהי
\begin{align*}
f \colon \Lie\prs{G} &\riso \Lie\prs{G_2} \\
\pmat{0 & a \\ 0 & 0} &\mapsto \prs{a}
\end{align*}
ונחפש הרמה
\[\text{.} \phi \colon G \to H\]
נדרוש
\[\text{.} \phi\prs{\exp_{G_1} \pmat{0 & a \\ 0 & 0}} = \exp_{G_2}\prs{f\pmat{0 & a \\ 0 & 0}}\]
אז
\[\text{.} \phi\prs{\pmat{1 & a \\ 0 & 1}} = e^a\]

נרצה לדעת האם אפשר להרים את
$f^{-1}$.
התשובה היא לא, כי
$\log$
לא מוגדר על כל
$\mbb{C}^\times$.
עדיין,
$\log$
מוגדר מקומית.
\end{example}

\subsection{הצגות של טורוסים}

נסתכל על טורוס
$n$%
־מימדי
\[\text{.} \mbb{T}^n \ceq \set{\mrm{diag}\prs{z_1, \ldots, z_n}}{\abs{z_i} = 1} \cong \prs{S^1}^n\]
נניח כי
\[\text{.} \phi \colon \mbb{T}^n \to \mbb{C}^\times = \mrm{GL}_1\prs{\mbb{C}}\]
נסמן
$\mfrak{t}^n \ceq \Lie\prs{\mbb{T}^n}$
ומתקיים
\[\text{.} \mfrak{t}^n = \set{\mrm{diag}\prs{2 \pi i \theta_1, \ldots, 2 \pi i \theta_n}}{\theta_i \in \mbb{R}}\]
מתקיים כי
$\mfrak{t}^n$
אלגברה קומוטטיבית ו־%
$\diff \phi$
הומומורפיזם כללי.
נכתוב
\[\text{.} \forall X \in \mfrak{t}^n \colon \phi\prs{\exp\prs{X}} = e^{\diff \phi\prs{X}}\]
אם
$X = \prs{2 \pi i \theta_1, \ldots, 2 \pi i \theta_n}$
מתקיים
\[\diff \phi\prs{X} = 2 \pi i \prs{\ell_1 \theta_1 + \ldots + \ell_n \theta_n}\]
עבור
$\ell_1,\ldots, \ell_n \in \mbb{C}$.
ככה נראים פונקציונלית
$\mbb{R}$%
־לינאריים
$\mbb{R}^n \to \mbb{C}$.

אם
$\theta_1,\ldots, \theta_N \in \mbb{Z}$
אז
$\exp\prs{X} = I$.
לכן במקרה זה
$e^{\diff \phi\prs{X}} = 1$
ואז
$\diff \phi\prs{X} = 2 \pi i k$
עבור
$k \in \mbb{Z}$.
בפרט, אם ניקח
$X = 2 \pi i e_j$
נקבל
\[\diff \phi\prs{X} = 2 \pi i \ell_j \in 2 \pi i \mbb{Z}\]
ואז
$\ell_j \in \mbb{Z}$.
אז
\[\phi\prs{\mrm{diag}\prs{1, \ldots, 1, e^{2 \pi i \theta}, 1, \ldots, 1}} = e^{2 \pi i \ell_j \theta}\]
כאשר ה־%
$e^{2 \pi i \theta}$
מופיע במקום ה־%
$j$.
אז
\[\text{.} \phi\prs{\mrm{diag}\prs{1, \ldots, z, \ldots, 1}} = z^{\ell_j}\]

נצרף את כל ערכי
$j$
ונקבל
\[\phi\prs{\mrm{diag}\prs{z_1, \ldots, z_n}} = z_1^{\ell_1} \cdot \ldots \cdot z_n^{\ell_n}\]
עבור
$\ell_i \in \mbb{Z}$.

אכן, כל בחירה של שלמים
$\ell_1, \ldots, \ell_n$
מגדירה הומומורפיזם
$\mbb{T}^n \to \mbb{C}^\times$
לפי הנוסחה הזאת.

יתרה מכך, אם
\[\phi \colon \mbb{T}^n \to \mrm{GL}\prs{V}\]
הצגה מרוכבת של טורוס (כלומר, הומומורפיזם, כאשר
$V$
מ"ו מעל
$\mbb{C}$%
)
אז
\[\diff \phi \colon \mfrak{t}^n \to \endo\prs{V}\]
העתקה לינארית. נסמן
\[E_j = \diff \phi\prs{2 \pi i e_j} \in \endo\prs{V}\]
ונקבל
$\exp E_j = I$.
$E_j$
כזאת חייבת להיות לכסינה עם ע"ע שהם כפולות שלמות של
$2 \pi i$.
בבסיס
$B$
ל־%
$V$
בו
$E_j$
אלסונית מתקיים
\[\brs{\phi\prs{1, \ldots, z, \ldots, 1}}_B = \exp\brs{E_j}_B = \mrm{diag}\prs{z^{\ell_{j,1}}, \ldots, z^{\ell_{j,n}}}\]
כאשר
$n = \dim\prs{V}$
וכאשר
\[\text{.} \brs{E_j} = \mrm{diag}\prs{2 \pi i \ell_{j,1}, \ldots, 2 \pi i \ell_{j,n}}\]

מתקיים
\[\brs{E_{j,1}, E_{j_2}} = \diff \phi\prs{\brs{t_1, t_2}} = \diff \phi\prs{0} = 0\]
עבור
$t_j = 2 \pi i e_j$
וכיוון ש־%
$\mfrak{t}^n$
אלגברה קומוטטיבית.
לכן
$E_1, \ldots, E_n$
מטריצות מתחלפות בזוגות. במצב כזה, קיים בסיס
$B$
ל־%
$V$
שבו כל
$E_1, \ldots, E_n$
אלכסוניות.
כלומר
\[\brs{\phi\prs{z_1, \ldots, z_n}}_B = \mrm{diag}\prs{z_1^{\ell_{1,1}} \cdot \ldots \cdot z_n^{\ell_{n,1}}, \ldots, z_1^{\ell_{1,m}} \cdot \ldots z_n^{\ell_{n,m}}}\]
ןאכם כך בלחרה של שלמים
$\set{\ell_{r,s}}_{\substack{r \in [n] \\ s \in [m]}}$
תיתן הומומורפיזם כזה.

במילים אחרות, פירקנו את
$V$
לסכום ישר
\[V = \bigoplus_{i \in [m]} V_i\]
כל שכל
$V_i$
הוא מרחב חד־מימדי של וקטורים עצמיים ל־%
$\phi\prs{\mbb{T}^n}$.

\subsection{טורי פורייה}

פורייה ניסה לפתור מד"ח (משוואת החום) עם תנאי התחלה מחזוריים.
תנאי ההתחלה הוא פונקציה רציפה
$f \colon S^1 \to \mbb{C}$.
הוא שם לב שקל יותר לפתור את הבעיה אם מפרקים את
$f$
לטור
\[\text{.} f = \sum_{n \in \mbb{Z}} a_n z^n\]
בשפה מודרנית,
$f$
היא וקטור בתוך מרחב הפונקציות על
$\mbb{T} \cong S^1$,
שהיא חבורה.
נסמן ב־%
$\mcal{C}\prs{\mbb{T}}$
את הפונקציות הרציפות
$\mbb{T} \to \mbb{C}$.
יש הומומורפיזם טבעי
\[\text{.} \phi \colon \mbb{T} \to \mbb{GL}\prs{\mcal{C}\prs{\mbb{T}}}\]
זאת נקראת ההצגה הרגולרית של
$\mbb{T}$
ומוגדרת על ידי
\[\text{.} \phi\prs{g}\prs{f} = \phi\prs{g}\prs{f}\prs{z} = f\prs{g^{-1} z}\]
אם
$\mcal{C}\prs{\mbb{T}}$
מתפרק (כמו במקרה הסוף מימדי) לסכום ישר של מרחבים עצמיים חד־מימדיים זה בדיוק יתן פירוק לטור פורייה, כי הפונקציות
$z^n$
הן בדיוק הוקטורים העצמיים של
$\phi\prs{\mbb{T}}$.

\subsection{גרעין ותמונה}

\begin{proposition}
יהי
$\phi \colon G \to H$
הומומורפיזם של חבורות מטריצות.
\begin{enumerate}
\item $\Lie\prs{\ker\prs{\phi}} = \ker \prs{\diff \phi}$.
\item $\Lie\prs{\im \phi} = \im\prs{\diff \phi}$
אם
$\quot{G}{G^\circ}$
בת־מניה.
\end{enumerate}
\end{proposition}

\begin{proof}
\begin{enumerate}
\item מתקיים
$X \in \Lie\prs{\ker \phi}$
אם ורק אם
$\exp\prs{tX} \in \ker \phi$
לכל
$t$
מספיק קטן, אם ורק אם
\[\exp\prs{t \diff \phi\prs{X}} = \phi\prs{\exp\prs{tX}} = I_H\]
לכל
$t$
מספיק קטן, אם ורק אם
$\diff \phi\prs{X} = 0$
אם ורק אם
$X \in \ker \prs{\diff \phi}$.
\item תהי
$X \in \Lie\prs{G}$.
אז
\[\exp\prs{t \diff \phi\prs{X}} = \phi\prs{\exp \prs{tX}} \in \im \phi\]
ולכן
$\diff \phi\prs{X} \in \Lie\prs{\im \phi}$.
אז
$\im\prs{\diff \phi} \subseteq \Lie\prs{\im \phi}$.

בכיוון ההפוך, מתקיים
\[G^\circ = \Gamma\prs{\Lie\prs{G}} = \bigcup_{k \in \mbb{Z}} a_k \cdot \exp\prs{\bar{U}}\]
עבור ערכים
$a_k \in G^\circ$.
תהי
$U \subseteq \Lie\prs{G}$
סביבה פתוחה של
$0$.
אז
\[G = \bigcup a_k \exp\prs{\bar{U}}\]
איחוד בן מניה עבור ערכים
$a_k \in G$.
אז
\[\text{.}\im \phi = \bigcup \phi\prs{a_k} \phi\prs{\exp\prs{[\bar{U}}}\]
$\bar{U}$
קומפקטית לכן
$\phi\prs{\exp\prs{\bar{U}}}$
קומפקטית ובפרט סגורה.
כמו בהוכחה מקודם, אם נפעיל את משפט בייר
\[\phi\prs{a_k} \phi\prs{\exp\prs{\bar{U}}} = \phi\prs{a_k} \exp\prs{\diff \phi\prs{\bar{U}}}\]
עם פנים פתוח.
כמו בהוכחה קודמת, נקבל ש־%
$\diff \phi\prs{\bar{U}}$
מכילה קבוצה פתוחה ב־%
$\Lie\prs{\im \phi}$.
לכן
\[\text{.} \diff \phi\prs{\Lie\prs{G}} = \Lie\prs{\im \phi}\]
\end{enumerate}
\end{proof}

\begin{corollary}
אם
$\phi$
חד־חד ערכית / על,
$\diff \phi$
חד־חד ערכית / על, בהתאמה.
\end{corollary}

תהי
$G$
חבורת מטריצות ותהי
$\mfrak{g} = \Lie\prs{G}$.
ראינו שמתקיים
$\diff\prs{\Ad_G\prs{a}} = \Ad_{\mfrak{g}}\prs{a}$
וגם
\[\text{.}\ker \Ad_G\prs{a} = Z\prs{a}\]
כעת נקבל מהטענה שמתקיים
\[\Lie\prs{Z\prs{a}} = \ker\prs{\diff\prs{\Ad_G\prs{a}}} = \ker\prs{\Ad_{\mfrak{g}}\prs{a}} = \set{X \in \mfrak{g}}{aX = Xa}\]
ונסמן את הביטוי האחרון
$\mfrak{z}\prs{a}$.
אז
\begin{align*}
a \in \ker\prs{\Ad_\mfrak{g}} &\iff \mfrak{z}\prs{a} = \mfrak{g}
\\&\iff \Lie\prs{Z\prs{a}} = \mfrak{g}
\\&\iff Z\prs{a}^\circ = G^\circ
\\&\iff G^\circ \leq Z\prs{a}
\end{align*}
ואם
$G$
קשירה נקבל
\[\text{.} \ker\prs{\Ad_{\mfrak{g}}} = Z\prs{G}\]

נקבל באופן כללי יותר
\[\text{.} Z\prs{G} \leq \ker\prs{\Ad_{\mfrak{g}}} = Z_G\prs{G^\circ} = \set{a \in G}{\forall g \in G^\circ \colon ag = ga}\]
אז
\[\Lie\prs{\ker \Ad_{\mfrak{g}}} = \ker\prs{\diff\prs{\Ad_{\mfrak{g}}}} = \ker\prs{\ad} = \set{X \in \mfrak{g}}{\forall Y \in \mfrak{g} \colon \brs{X,Y} = 0} = \mfrak{g}\prs{\mfrak{g}}\]
ואם
$G$
קשירה נקבל
\[\text{.} \mfrak{z}\prs{\mfrak{g}} = \Lie\prs{Z\prs{G}}\]
במקרה זה נקבל גם
\[\text{.} \Ad_{\mfrak{g}}\prs{G} \cong \quot{G}{\ker\prs{\Ad_{\mfrak{g}}}} = \quot{G}{Z\prs{G}}\]

\begin{fact}
אם
$f \colon \mfrak{g} \to \mfrak{h}$
הומומורפיזם של אלגבראות לי, אז
\[\im f \cong \quot{\mfrak{g}}{\ker f}\]
כאלגבראות לי.
\end{fact}

לפי העובדה, מתקיים
\[\Lie\prs{\Ad_{\mfrak{g}}\prs{G}} = \Im \prs{\ad_{\mfrak{g}}} \cong \quot{\mfrak{g}}{\ker \prs{\Ad_{\mfrak{g}}}} = \quot{\mfrak{g}}{\mfrak{z}\prs{\mfrak{g}}}\]
כאשר
$G$
קשירה. במקרה זה גם
$\Ad_{\mfrak{g}}\prs{G}$
קשירה ולכן
\[\text{.} \Ad_{\mfrak{g}}\prs{G} = \Gamma\prs{\quot{\mfrak{g}}{\mfrak{z}\prs{\mfrak{g}}}}\]
זה לא תלוי ב־%
$G$
אל רק ב־%
$\mfrak{g}$.

\begin{definition}[אלגברת לי]
\emph{אלגברת לי}
היא מרחב וקטורי
$\mfrak{g}$
מעל שדה
$\mbb{F}$
יחד עם פעולה
$\brs{\cdot,\cdot} \colon \mfrak{g} \times \mfrak{g} \to \mfrak{g}$
בילינארית, אסוציאטיבית, אנטיאסוציאטיבית (כלומר $\brs{x,y} = -\brs{y,x}$) ושמקיימת את זהו יעקובי:
\[\text{.} \brs{x, \brs{y,z}} + \brs{z, \brs{x,y}} + \brs{Y,\brs{z,x}}\]
\end{definition}

עם הגדרה זאת לפעמים יותר קל לעבוד. למשל, אפשר להגדיר כך מנה של אלגברת לי. אם
$\mfrak{h} \ideal \mfrak{g}$
אידאל (תת־מרחב שמקיים
$\brs{\mfrak{h}, \mfrak{g}} \subseteq \mfrak{h}$%
), נוכל להגדיר
$\quot{\mfrak{g}}{\mfrak{h}}$
ביחד עם
\[\text{.} \brs{X + \mfrak{h}, Y + \mfrak{h}} \ceq \brs{X,Y} + \mfrak{h}\]

בחזרה לדיון הקודם, מתקיים
\[\ad\prs{\mfrak{g}} \cong \quot{\mfrak{g}}{\ker\prs{\ad}} = \quot{\mfrak{g}}{z\mfrak{z} prs{\mfrak{g}}}\]
כאשר
\[\text{.} \mfrak{z}\prs{\mfrak{g}}\prs{X \in \mfrak{g}}{\forall Y \in \mfrak{g} \colon \brs{X,Y} = 0}\]
זה איזומורפיזם של אלגבראות לי, ומתקיים גם
\[\text{.} \Lie\prs{\quot{G}{Z\prs{G}}} \quot{\mfrak{g}}{\mfrak{z}\prs{\mfrak{g}}}\]

\begin{remark}
גם אם
$\mfrak{g}$
מראש הייתה נלקחת כאלגברה אבסטרקטית אז עדיין
$\ad\prs{\mfrak{g}}$
אלגברה של מטריצות עבור
\[\text{.} \ad \colon \mfrak{g} \to \endo\prs{\mfrak{g}}\]
כלומר,
$\quot{\mfrak{g}}{\mfrak{z}\prs{\mfrak{g}}}$
אלגברת מטריצות גם עבור
$\mfrak{g}$
אבסטרקטית.

ממשפט אדו, למעשה
$\mfrak{g}$
עצמה גם בהכרח ניתנת לשיכון כמטריצות.

קיבלנו גם שעבור
$G$
חבורת מטריצות קשירה,
$\quot{G}{Z\prs{G}}$
גם היא חבורת מטריצות.
עבור
$H \ideal G$
תת־חבורה נורמלית סגורה,
$\quot{G}{H}$
לא בהכרח חבורת מטריצות.
זאת אחת המוטיבציות להגדיר חבורות לי אסטרקטיות.
\end{remark}

\begin{example}
מתקיים
$\mrm{PGL}_n\prs{\mbb{C}} = \Ad_{\mfrak{g}}\prs{\mrm{GL}_n\prs{\mbb{C}}}$.
ידוע כי
$\mrm{GL}_n\prs{\mbb{C}}$
קשירה. אפשר לראות זאת על ידי צורת ז'ורדן.
הראו כתרגיל שאפשר לחבר כל איבר במסילה רציפה ליחידה.

מתקיים גם
\begin{align*}
Z\prs{\mrm{GL}_n\prs{\mbb{C}}} &\cong \mbb{C}^\times \\
\mrm{PGL}_n\prs{\mbb{C}} &\cong \quot{\mrm{GL}_n\prs{\mbb{C}}}{\mbb{C}^\times} \\
\text{.} \Lie\prs{\mrm{PGL}_n\prs{\mbb{C}}} &\cong \quot{M_n\prs{\mbb{C}}}{\mbb{C}}
\end{align*}
גם
\begin{align*}
\phi \colon \mfrak{sl}_n\prs{\mbb{C}} &\to \Lie\prs{\mrm{PGL}_n\prs{\mbb{C}}} \\
X &\to X
\end{align*}
איזומורפיזם.

כל
$X \in M_n\prs{\mbb{C}}$
ניתן לכתיבה כ־%
\[X = X_0 + \alpha I\]
עבור
$X_0 \in \mfrak{sl}_n\prs{\mbb{C}}$
ו־%
$\alpha = \frac{\tr X}{n}$.
ב־%
$\quot{M_n\prs{\mbb{C}}}{\mbb{C}}$
מתקיים
$X = X_0$
ולכן
$\phi$
על.

נראה ש־%
$\phi$
חד־חד ערכית. יהי
$X \in \ker \phi$.
אז
$X = \alpha I$
ומתקיים
$n \alpha = \tr X = 0$
לכן
$X = 0$.

מתקיים גם
\[ \mrm{PSL}_n\prs{\mbb{C}} = \Ad_{\mfrak{g}}\prs{\mrm{SL}_n\prs{\mbb{C}}} = \quot{\mrm{SL}_n\prs{\mbb{C}}}{\mu_n \cdot I}\]
כאשר
\[\text{.} \mu_n \ceq \set{z \in \mbb{C}}{z^n = 1}\]
גם
$\mrm{SL}_n\prs{\mbb{C}}$
קשירה מאותן סיבות. מתקיים
\[\mfrak{z}\prs{\mfrak{sl}_n\prs{\mbb{C}}} = \set{0}\]
וגם
\[\text{.} \ad\prs{\mfrak{sl}_n\prs{\mbb{C}}} = \mfrak{sl}_n\prs{\mbb{C}}\]
אז
\[\text{.} \mrm{PSL}_n\prs{\mbb{C}} \cong \mrm{PGL}_n\prs{\mbb{C}}\]
\end{example}

\subsection{מרחבי כיסוי}

\begin{definition}[הומאומורפיזם לוקלי]
נאמר שהומומורפיזם
\[\pi \colon \tilde{G} \to G\]
הוא
\emph{הומאומורפיזם לוקלי}
אם לכל
$x \in \tilde{G}$
יש סביבה פתוחה
$U$
כך ש־%
$\pi\prs{U}$
פתוחה וש־%
$\rest{\pi}{U}$
הומאומורפיזם לתמונה.
\end{definition}

\begin{definition}[העתקת כיסוי]
אם
$\tilde{G}, G$
קשירות, נקרא להומאומורפיזם לוקלי
$\pi \colon \tilde{G} \to G$
\emph{העתקת כיסוי}.
\end{definition}

\begin{example}
נסתכל על
$\phi \colon \mbb{R} \to \mbb{T}$
המוגדרת על ידי
$\theta \mapsto e^{2 \pi i \theta}$.
אז
$\ker \phi = \mbb{Z}$.
נגדיר
$\phi_n \colon \mbb{T} \to \mbb{T}$
על ידי
$z \mapsto z^n$
ואז
$\phi_n$
גם העתקות כיסוי, וגם
$\phi_n \circ \phi$
כאלה כהרכבות של כיסויים. מתקיים
$\ker \prs{\phi_n \circ \phi} = \frac{1}{n} \mbb{Z}$.
\end{example}

\begin{proposition}
\begin{enumerate}
\item תהי
\[\pi \colon \tilde{G} \to G\]
העתקת כיסוי. אז
$\pi$
על כי יש סביבה פתוחה
$\pi\prs{U}$
של
$I$.
$G^\circ$
נוצרת תמיד על ידי סביבה פתוחה של
$I$ (תרגיל)
ומתקיים
\[\text{.} G = G^\circ \subseteq \im \pi\]

\item
$\ker \pi \leq \tilde{G}$
חבורה דיסקרטית. אם
$e \in \ker \pi$
יש סביבה
$U$
של
$e$
כך שמתקיים
\[\text{.} U \cap \ker \pi = \set{e}\]

\item מתקיים
$\ker \pi \leq Z\prs{\tilde{G}}$.
יהי
$e \in \ker \pi$.
 עבור
 $a \in \tilde{G}$
 ניקח מסילה רציפה
$\tilde{a}\prs{t}$
עם
$\tilde{a}\prs{0} = I$
ועם
$\tilde{a} \prs{1} = a$.
תהי
\[\text{.} \gamma\prs{t} \ceq \tilde{a}\prs{t} \cdot e \cdot \tilde{a}\prs{t}^{-1} \in \ker \pi\]
אז
\[\gamma\prs{0} = e, \quad \gamma\prs{1} = aea^{-1}\]
וזאת מסילה בתוך
$\ker \pi$
דיסקרטית.
אז
\[\text{.} e = \gamma\prs{0} = \gamma\prs{1} = aea^{-1}\]

\item $G$
היא מנה של
$\tilde{G}$
בחבורה דיסקרטית במרכז.
\end{enumerate}
\end{proposition}

\begin{lemma}
יהי
$\phi \colon G \to H$
הומומורפיזם.

$\phi$
העתקת כיסוי אם ורק אם
$\diff \phi$
איזומורפיזם.
\end{lemma}

\end{document}