\documentclass[10pt, twoside]{book}

\usepackage{hyperref}

%%%%%%%%%%%%
% Geometry %
%%%%%%%%%%%%

\RequirePackage{geometry}
\geometry{
	 a4paper,
	 inner=10mm,
	 outer=30mm,
	 marginparwidth = 20mm,
	 top = 20mm,
	 bottom = 20mm
	}

%%%%%%%%%
% Babel %
%%%%%%%%%

\usepackage[nil,bidi=basic-r]{babel}
\babelprovide[import=he,main]{hebrew}
\babelprovide[import=en-GB]{english}

% For some reason Babel’s `\babelfont` doesn’t work
\babelfont[hebrew]{rm}{Open Sans Hebrew}

\newcommand{\texthebrew}[1]{\foreignlanguage{hebrew}{#1}}
\newcommand{\nikud}[1]{$\mbox{\H{#1}}$}
\newcommand{\textenglish}[1]{\foreignlanguage{english}{#1}}
\newcommand{\LR}[1]{{‏\textdir TLT #1}}

%%%%%%%%%
% Maths %
%%%%%%%%%

\usepackage{math-fonts}
\usepackage{math-graphics}
\usepackage{math-symbols}
\usepackage{math-theorems-heb}

%%%%%%%%%
% Title %
%%%%%%%%%

\title{רשימות הרצאה לחבורות לי \\ \large{חורף 2020, הטכניון}}
\author{הרצאותיו של מקס גורביץ' \\ \large \small{הוקלדו על ידי אלעד צורני}}
\date{\today}

\begin{document}

\maketitle
\tableofcontents

\chapter{מבוא}

\section{מבוא היסטורי}

\subsection{חבורות לי}

בנורבגיה, סביב שנת 1870, מתמטיקאי בשם סופוס לי, שחקר משוואות דיפרנציאליות, שם לב להופעה של מה שנקרא
\textenglish{\emph{``transformation groups''}}.

\begin{description}
\item[באופן לוקלי,]
נניח כי נתון שדה וקטורי חלק
\[\text{.} x \colon \mbb{R}^n \to \mbb{R}^n\]
נחפש פתרון למשוואה
\begin{align*}
y'\prs{t} &= x\prs{y\prs{t}} \\
y\prs{0} &= y_0
\end{align*}
עבור
$y \colon \mbb{R} \to \mbb{R}^n$.
לפי משפט קיום ויחידות של מד"ר, קיים פתרון יחיד
$y\prs{t}$
שמוגדר עבור
$t \in \prs{-\eps, \eps}$
עבור
$\eps > 0$
כלשהו.

נגדיר
$\phi_x\prs{t} \prs{y_0} \ceq y\prs{t}$
לכל
$y_0 \in \mbb{R}^n$.
אז
\[\phi_X\prs{t} \colon \mbb{R}^n \to \mbb{R}^n\]
אוטומורפיזם של
$\mbb{R}^n$.
זאת משפחה של אופרטורים שמשתנה באופן חלק כתלות ב־%
$t$.

מתקיים
\[\phi_x\prs{0} = \id\]
ומיחידות הפתרונות,
\[ \text{.} \phi_x\prs{t} \circ \phi_x\prs{s} = \phi_x\prs{t+s}\]
התמונה
$\im\phi_x$
נקראת
\emph{חבורה חד־פרמטרית}.
אז
\[\phi_x \colon \mbb{R} \to \aut\prs{\mbb{R}^n}\]
הומומורפיזם של חבורות.
זה לא מדויק, $\phi_x$ אולי לא מוגדר תמיד.

\item[באופן גלובלי,]
קיימות משוואות שהפתרונות שלהן אינווריאנטיים לפעולה של חבורה כלשהי. למשל
\[\text{.} \mcal{O}_n\prs{\mbb{R}} \ceq \set{A \in M_n\prs{\mbb{R}}}{A^t A = I}\]

נסתכל על הלפלסיאן
\[\text{.} \Delta = \sum_{i \in [n]} \del_{x_i, x_i}\]
אם
$y \colon \mbb{R}^n \to \mbb{R}$
מקיימת
$\Delta\prs{y} = 0$
אז
$y \circ g = y$
לכל
$g \in \mcal{O}_n\prs{\mbb{R}}$.
\end{description}

סופוס לי חקר חבורות חלקות כאלו, ולמעשה הגדיר חשבון דיפרנציאלי של חבורות.
כדי לחקור פונקציה חלקה
$f \colon \mbb{R}^n \to \mbb{R}^m$
אנו מסתכלים על הנגזרת של
$f$
בנקודה, שהיא העתקה לינארית
\[\text{.} T \colon \mbb{R}^n \to \mbb{R}^m\]
האנלוג לחבורות הוא
\emph{חבורת לי}
שהיא חבורה שאותה אפשר
.``לגזור''
פורמלית, זאת חבורה עם מבנה של יריעה חלקה, או במילים אחרות, אוביקט חבורה בקטגוריה של יריעות חלקות.

``נגזרת''
של חבורת לי נקראת
\emph{אלגברת לי}
$\mrm{Lie}\prs{G}$.
זהו מרחב וקטורי עם מבנה מסוים.
תורת לי בסיסית היא הבנת ההתאמה בין חבורות לי לאלגבראות לי שלהן.

\begin{examples*}
\enumthm
\begin{enumerate}
\item $\mrm{Lie}\prs{\mrm{GL}_n\prs{\mbb{R}}} = M_n\prs{\mbb{R}}$.
\item $\mrm{Lie}\prs{\mrm{GL}_n\prs{\mbb{C}}} = M_n\prs{\mbb{C}}$.
\item $\mrm{Lie}\prs{\mcal{O}_n\prs{k}} = \mfrak{so}_n\prs{k} \ceq \set{A \in M_n\prs{k}}{A^t = -A}$.
\item $\mrm{Lie}\prs{\mrm{SL}_n\prs{k}} = \mfrak{sl}_n\prs{k} \ceq \set{A \in M_n\prs{k}}{\tr\prs{A} = 0}$.
\item נגדיר
\[\mrm{Sp}_{2n}\prs{k} = \set{A \in M_{2n}\prs{k}}{A^t J A = J}\]
עבור
\[J = \pmat{0 & I_n \\ -I_n & 0} \text{.}\]
זאת נקראת
\emph{החבורה הסימפלקטית}.
אז
\[\mrm{Lie}\prs{\mrm{Sp}_{2n}\prs{k}} = \mfrak{sp}_{2n}\prs{k} \ceq \set{A \in M_{2n}\prs{k}}{A^tJ = -J A} \text{.}\]
\end{enumerate}
\end{examples*}

באופן כללי, אם ניקח חבורת לי
$G \leq \mrm{GL}_n\prs{k}$
אז
$\mrm{Lie}\prs{G} \leq M_n\prs{k}$
וזאת תהיה אלגברה יחד עם הפעולה של הקומוטטור
\[\text{.} \brs{A,B} = AB - BA\]

בקורס זה נעשה "קירוב" של התאמת לי על ידי לקיחת
\emph{כל}
תת־חבורה
$G \leq \mrm{GL}_n\prs{\mbb{R}}$
ובניית
$\mrm{Lie}\prs{G} \leq M_n\prs{\mbb{R}}$.

\subsection{סיווג של חבורות לי}

חבורות לי מופיעות במגוון מקומות בטבע ובמתמטיקה.
הרבה חבורות סימטריה בטבע הינן חבורות לי. אנו רוצים להבין בין השאר מסיבות אלו את המבנה של חבורות לי ולנסות לסווג אותן.

תחילה נבין את הקשר בין
$G_1, G_2$
כך ש־%
$\mrm{Lie}\prs{G_1} \cong \mrm{Lie}\prs{G_2}$.
התשובה לשאלה זאת תיעזר בחבורות כיסוי.
נקבל מכך שמספיק להשיג סיווג של אלגבראות לי. למעשה, מספיק לסווג אלגבראות לי מעל
$\mbb{C}$.

נצמצם לפעמים את הדיון לחבורות לי
\emph{פשוטות}
שהן חבורות לי קשירות ללא תת־חבורות נורמליות קשירות לא טריוויאליות.
דוגמאות לחבורות לי פשוטות הן
$\mrm{SO}_n = \mrm{SL}_n \cap \mcal{O}_n, \mrm{Sp}_{2n}, \mrm{SL}_n$.
מהן ניתן, במובן מסוים, לבנות כל חבורת לי.
עבור
$G$
חבורת לי פשוטה,
$\mrm{Lie}\prs{G}$
היא אלגברת לי פשוטה.
לכן, מספיק לסווג אלגבראות לי פשוטות.

סביב שנת 1890, \textenglish{Cartan} ו־%
\textenglish{Killing}
סיווגו באופן מלא אלגבראות לי פשוטות מעל
$\mbb{C}$,
שהן "אבני הבניין של הסימטריה של הטבע".

%date 26.10.2020

יש משפחות של אלגבראות לי פשוטות, שנקראות אלגבראות קלאסיות. הן
$\mfrak{sl}_n\prs{\mbb{C}}, \mfrak{so}_n\prs{\mbb{C}}, \mfrak{sp}_{2n}\prs{\mbb{C}}$
שהגדרנו מקודם.
חבורות לי שאלו הן אלגבראות לי שלהן נקראות חבורות קלאסיות.
התברר שלמעט המשפחות הקלאסיות יש עוד בדיוק 5 אלגבראות לי פשוטות, שנקראות מיוחדות
\textenglish{exceptional}.
הן מסומנות
$G_2, F_4, E_6, E_7, E_8$.
הכי קטנה מהן היא
$G_2$
בעלת מימד
$14$,
והכי גדולה היא
$E_8$
בעלת מימד
$248$.
לצורך הבנת התוצאה, נדרשת התעמקות אלגברית במבנה של אלגבראות לי, מה שדורש זמן.

\subsection{חבורות קומפקטיות}

חבורות לי הן חבורות שהן יריעות חלקות.
מתברר שאם הן קומפקטיות, יש הרבה מה להגיד עליהן.
חבורות לי קומפקטיות הן למשל
\begin{align*}
\mcal{O}\prs{n} &\equiv \mcal{O}_n\prs{\mbb{R}} \ceq \set{A \in \mrm{GL}_n\prs{\mbb{R}}}{A^t A = I} \\
\text{.} U\prs{n} &\ceq \set{A \in \mrm{GL}_n\prs{\mbb{C}}}{\bar{A}^t A = I}
\end{align*}
כאשר המבנה שלהן מושרה מזה של
$M_n\prs{k} \cong k^{n^2}$.

\subsection{תורת ההצגות}

בתורת לי הגיעו להבנה שניתן ללמוד הרבה על חבורה ע"י הבנת הפעולה שלה על מרחבים וקטוריים. זוהי תורת ההצגות. עבור חבורות קומפקטיות יש מבנה נחמד לתורת ההצגות.

\subsection{חבורות רדוקטיביות ואלגבריות}

משפט שנחתור אליו הוא שכל חבורת לי קומפקטית קשירה היא חבורה אלגברית של מטריצות ממשיות.
מסקנה ממשפט זה תהיה שלכל חבורת לי קומקפטית ניתן להגדיר קומפלקסיפיקציה על ידי הרחבת סקלרים. זאת מוגדרת על ידי השורשים המרוכבים של אותם פולינומים.
חבורות שנקבל מקומפלקסיפיקציה נקראות
\emph{רדוקטיביות}
ומהוות משפחה גדולה.

בשנת 1920 פיתח הרמן וייל
\textenglish{(Weyl)}
את המבנה וההצגות של חבורות לי רדוקטיביות על סמך חבורות קומפקטיות.

בשנת 1940 קלוד שיבליי
\textenglish{(Chevalley)}
הראה שניתן לתאר כל חבורת לי פשוטה כחבורה אלגברית.
מסקנה של כך היא ש"לא צריך" אנליזה כדי לחקור תורת לי.
אפשר גם להגדיר אנלוגית של חבורות לי מעל כל שדה.

\subsection{התפתחות מודרנית של התחום}

חבורות לי התפתחו לנושאים מאוחרים שמופיעים בהרבה תחומים.

\begin{itemize}
\item
אחד התחומים הוא חבורות אלגבריות. תחום אחר הוא חבורות סופיות מסוג לי
\textenglish{(Lie Type)}
שהן חבורות אלגבריות מעל שדה סופי. אלו אבני יסוד בסיווג של חבורות סופיות פשוטות.

\item
נושא אחר הוא מרחבים סימטריים. אלו מרחבים טבעיים עם פעולה של חבורת לי. למשל
$S^3$
הוא מרחב סימטרי עם הפעולה של
$\mcal{O}\prs{3}$.

\item
חבורות לי גם הפכו לכלי מרכזי בחקר של תבניות אוטומורפיות ותורת המספרים. הן מתקשרות למשפט
\textenglish{Harish-Chandra}%
ולתוכנית
\textenglish{Laglands}.

\item
חבורות לי מתקשרות גם לאנליזה הרמונית.

\item נחקרת היום גם תורת הצגות אינסוף־מימדיות של חבורות לי ופעולות של חבורות לי על מרחבי פונקציות.
בין השאר נחקרות פעולות על מרחבי פונקציות שמקודדים תוכן של תורת מספרים.

\item
נחקרות גם היום, משנת 1985, חבורות קוונטיות, שהן דפורמציות של חבורות לי.

\item נחקרות גם חבורות ואלגבראות אינסוף־מימדיות שנקראות
אלגבראות
\textenglish{Kac-Moody}.

\item ישנן תורת לי קטגורית ותורת לי גיאומטרית. הראשונה עוסקת למשל בקטגוריות שמוגדרות באופן דומה לחבורות לי.

\end{itemize}

\section{חבורות לי}

\subsection{אקספוננט של מטריצות}

נעבוד מעל שדה
$k \in \set{\mbb{R}, \mbb{C}}$.

נסמן ב־%
\[M \equiv M_n \equiv M_n\prs{k}\]
מטריצות
$n \times n$
מעל
$k$
עם הנורמה האוקלידית המושרית מ־%
$k^{n^2}$.

\begin{proposition}[עקרון ההצבה]
יהי
\[F\prs{z} = \sum_{n=0}^{\infty} a_n z^n\]
עבור
$a_n \in k$
טור חזקות עם רדיוס התכנסות
$r$.

לכל
$A \in M_n$
עבורה
$\norm{A} < r$
הטור
$F\prs{A} = \sum_{n=0}^{\infty} a_n A^n$
מגדיר טור מטריצות מתכנס בנורמה.
\end{proposition}

\begin{proof}
הנורמה מקיימת
\begin{align*}
\norm{X+Y} &\leq \norm{X} + \norm{Y} \\
\text{.} \norm{X \cdot Y} &\leq \norm{X} \cdot \norm{Y}
\end{align*}
לכן
\begin{align*}
\norm{\sum_{n=k}^\ell a_n A^n} &\leq \sum_{n=k}^\ell \abs{a_n}\norm{A^n}
\\&\leq
\sum_{n=k}^\ell \abs{a_n}\norm{A}^n
\\&<
\sum_{n=k}^\ell \abs{a_n}\prs{r-\eps}^n
\\&\xrightarrow[k,\ell \to \infty]{} 0
\end{align*}
ולפי קריטריון קושי נקבל שיש לסדרה
$\norm{\sum_{n=1}^N a_n A^n}$
גבול.
\end{proof}

אם
$F\prs{z}$
טור חזקות עם רדיוס התכנסות
$\rho$
ו־%
$G\prs{z}$
עם רדיוס התכנסות
$\sigma$,
אז
\begin{align*}
\prs{F+G}\prs{A} = F\prs{A} + G\prs{A} \\
\prs{FG}\prs{A} = F\prs{A} \cdot G\prs{A}
\end{align*}
עבור
$A$
עם
$\norm{A} < \min\set{\rho,\sigma}$.
אם
$G\prs{0} = 0$
אז
$F \circ G$
הוא טור חזקות, ועבור מטריצה
$A$
כך ש־%
$\norm{A} < \sigma$
וגם
$\norm{G\prs{A}} < \rho$
הטור
\[\prs{F \circ G}\prs{A} = F\prs{G\prs{A}}\]
מתכנס.



\begin{definition}[אקספוננט של מטריצה]
עבור
$z \in k$
נגדיר
\[\exp\prs{z} \ceq \sum_{n = 0}^{\infty} \frac{1}{n!} \cdot z^n\]
ועבור
$z \in k$
עם
$\abs{z} < 1$
נגדיר
\[\text{.} \log\prs{1+z} = \sum_{n=0}^{\infty} \frac{\prs{-1}^{n-1}}{n} \cdot z^n\]

מכאן, לכל מטריצה
$X \in M$
ניתן להגדיר
\[\exp\prs{X} = \sum_{n=0}^{\infty} \frac{1}{n!} X\]
ועבור
$X \in M$
עבורה
$\norm{X - I} < 1$
נגדיר
\[\text{.} \log\prs{x} \ceq \log\prs{1+\prs{X-I}}\]
\end{definition}

\begin{corollary}
\begin{enumerate}
\item לכל מטריצה
$X \in M$
מתקיים
\[\exp\prs{X} \cdot \exp\prs{-X} = I\]
וגם
$\exp\prs{X} \in \mrm{GL}_n\prs{k}$.

\item כאשר
$\norm{X - I} < 1$
מתקיים
\[\text{.} \exp\prs{\log\prs{X}} = X\]

\item עבור
$X \in M$
המקיימת
$\norm{X} < \log 2$
מתקיים
\[\text{.} \log\prs{\exp\prs{X}} = X\]

אכן
מתקיים
\begin{align*}
\norm{\exp\prs{X} - I} &= \norm{\sum_{n=1}^{\infty} \frac{1}{n!} X^n}
\\&\leq
\sum_{n=1}^{\infty} \frac{1}{n!} \norm{X}^n
\\&= \exp\prs{\norm{X}} - 1
\\&< 2 - 1
\\\text{.} \hphantom{\norm{\exp\prs{X} - I}} &= 1
\end{align*}
\end{enumerate}
\end{corollary}

\begin{exercise}
כאשר
$XY = YX$
מתקיים
\begin{align*}
\text{.} \exp\prs{X} \cdot \exp\prs{Y} = \sum_{n=0}^{\infty} \frac{1}{n!} \prs{X+Y}^n = \exp\prs{X+Y}
\end{align*}
בפרט, עבור
$t,s \in k$
מתקיים
\[\text{.} \exp\prs{tX} \cdot \exp\prs{sX} = \exp\prs{\prs{t+s}X}\]
לכן
\begin{align*}
a_X \colon k &\to \mrm{GL}_n\prs{k} \\
t &\mapsto \exp\prs{tX}
\end{align*}
הומומורפיזם של חבורות.
\end{exercise}

\begin{proposition}
\begin{enumerate}
\item $a_x$ הוא הפתרון היחיד למשוואה
\begin{align*}
a'\prs{t} = a\prs{t} \cdot X \\
a\prs{0} = I
\end{align*}
או למשוואה
\begin{align*}
a'\prs{t} = X \cdot a\prs{t} \\
a\prs{0} = I
\end{align*}
עבור
\[ \text{.} a \colon k \to \mrm{GL}_n\prs{k}\]

\item 
$a_X$
הוא ההומומורפיזם החלק היחיד
\[a \colon k \to \mrm{GL}_n\prs{k}\]
שמקיים
$a'\prs{0} = X$.
\end{enumerate}
\end{proposition}

\begin{proof}
\begin{enumerate}
\item $a_X$
פתרון למשוואה. מתקיים
\begin{align*}
\frac{\diff}{\diff t}\prs{a_X\prs{t}} &= \sum_{n=1}^\infty \frac{\diff}{\diff t} \prs{\frac{1}{n!} \prs{tX}^n}
\\&=
\sum_{n=1}^{\infty} \frac{t^{n-1}}{\prs{n-1}!} X^n
\\ &=
\exp\prs{tX} \cdot X
\\ \text{.} \hphantom{\frac{\diff}{\diff t}\prs{a_X\prs{t}}} &=
X \cdot \exp\prs{tX}
\end{align*}
יחידות נובעת ממשפט קיום ויחידות למד"ר.

\item אם
$a$
הוא הומומורפיזם כלשהו, כמו שנתון, מתקיים
\begin{align*}
a'\prs{t} &= \left. \frac{\diff}{\diff s} \prs{a\prs{t+s}} \right|_{s=0}
\\&=
\left. \frac{\diff}{\diff s}\prs{a \prs{t} \cdot a\prs{s}} \right|_{s=0}
\\&=
a\prs{t} \cdot a'\prs{0}
\\&=
a\prs{t} \cdot X
\end{align*}
ומ־%
$1$
נובע
$a = a_X$.
\end{enumerate}
\end{proof}

%TODO fill in last part of 2nd lecture

%29.10.2020

\begin{example}
נחשב אקספוננט של מטריצה אלכסונית. נקבל
\begin{otherlanguage}{english}
\begin{align*}
\exp\pmat{\lambda_1 & & \\ & \ddots & \\ & & \lambda_n}
&=
\sum_{k = 0}^{\infty} \frac{1}{k!} \pmat{\lambda_1 & & \\ & \ddots & \\ & & \lambda_n}^k
\\&=
\sum_{k=0}^{\infty} \frac{1}{k!} \pmat{\lambda_1^k & & \\ & \ddots & \\ & & \lambda_n^k}
\\\text{.}\hphantom{\exp\pmat{\lambda_1 & & \\ & \ddots & \\ & & \lambda_n}}&=
\pmat{\exp\prs{\lambda_1} & & \\ & \ddots & \\ & & \exp\prs{\lambda_n}}
\end{align*}
\end{otherlanguage}
\end{example}

\begin{definition}[אופרטור ההצמדה]
בהינתן
$a \in \mrm{GL}_n\prs{k}$
נגדיר את
\emph{אופרטור ההצמדה ב־%
$a$}
על ידי
\begin{align*}
\mrm{Ad}\prs{a} \colon M &\to M \\
\text{.} \hphantom{\mrm{Ad}\prs{a} \colon} X &\mapsto aXa^{-1}
\end{align*}
\end{definition}

\begin{notation}
עבור מרחב וקטורי
$V$
נסמן ב־%
$\endo\prs{V}$
את מרחב ההעתקות הלינאריות מ־%
$V$
לעצמו.
זהו מרחב וקטורי איזומורפי ל־%
$M_{n^2}$.
\end{notation}

\begin{notation}
עבור מרחב וקטורי
$V$
נסמן ב־%
$\mrm{GL}\prs{V} \subseteq \endo\prs{V}$
את האנדומורפיזמים ההפיכים של
$V$.
לעתים מסמנים זאת
$\mrm{Aut}\prs{V}$.
\end{notation}

\begin{remark}
מתקיים
$\mrm{Ad}\prs{a} \in \mrm{GL}\prs{M}$
כיוון ש־%
\[\text{.} \mrm{Ad}\prs{a^{-1}} \prs{\mrm{Ad}\prs{a}\prs{X}} = X\]
כמו כן,
$\mrm{Ad}$
הוא הומומורפיזם של חבורות
$\mrm{GL}_n\prs{k} \to \mrm{GL}\prs{M}$
כיוון שמתקיים
\[\text{.} \mrm{Ad}\prs{ab}\prs{X} =  ab X \prs{ab}^{-1} = a \prs{b X b^{-1}} a^{-1} = \mrm{Ad}\prs{a} \circ \mrm{Ad}\prs{b} \prs{X}\]
\end{remark}

\begin{exercise}
אם
$f\prs{z} = \sum_{k=0}^{\infty} a_k z^k$
טור חזקות, ו־%
$X \in M_n$
עבורה
$f\prs{X}$
מוגדרת, מתקיים
\[\text{.} \mrm{Ad}\prs{a} \prs{f\prs{X}} = f\prs{\mrm{Ad}\prs{a}\prs{X}}\]
בפרט, מתקיים
\[\text{.} \mrm{Ad}\prs{a} \prs{\exp\prs{X}} = \exp\prs{\mrm{Ad}\prs{a}\prs{X}}\]
\end{exercise}

\begin{corollary}
נניח כי
$V$
מרחב וקטורי עם בסיס
$B$.
יש זיהוי
\begin{align*}
\endo\prs{V} &\cong M_n \\
\text{.} \hphantom{\endo\prs{.}} T &\leftrightarrow \brs{T}_B
\end{align*}
לכן, עבור
$X \in \endo\prs{V}$
אפשר להגדיר
$\exp\prs{X} \in \mrm{GL}\prs{V}$
כאופרטור שמקיים
\[\brs{\exp\prs{X}}_B = \exp\prs{\brs{X}_B}\]
ובנייה זאת לא תלויה בבחירת הבסיס
$B$.
\end{corollary}

\begin{definition}[הקומוטטור]
עבור
$X,Y \in M$
נגדיר
\[\text{.} \brs{X,Y} = XY - YX\]
\end{definition}

\begin{definition}
עבור
$X \in M$
נגדיר
\begin{align*}
\mrm{ad}\prs{X} \colon M &\to M \\
\text{.} \hphantom{\mrm{ad}\prs{.} aaa} Y &\mapsto \brs{X,Y}
\end{align*}
\end{definition}

\begin{proposition}
עבור
$X \in M$
מתקיים
\[\text{.} \mrm{Ad}\prs{\exp\prs{X}} = \exp\prs{\mrm{ad}\prs{x}}\]
\end{proposition}

\begin{proof}
מתקיים
\[\mrm{Ad}\prs{\exp\prs{X}}\prs{Y} = \exp\prs{X} \cdot Y \cdot \exp\prs{-X}\]
ומצד שני
\begin{align*}
\text{.} \exp\prs{\mrm{ad}\prs{X}}\prs{Y} &= \sum_{k=0}^{\infty} \frac{1}{k!} \prs{\mrm{ad}\prs{X}}^k \prs{Y}
\end{align*}
נגדיר
\[\text{.} A\prs{t} \ceq \mrm{Ad}\prs{\exp\prs{tX}}\]
נחפש את
$A\prs{t}$.
מתקיים
\[A\prs{0} = \mrm{Ad}\prs{I} = \id\]
וגם
\begin{align*}
\frac{\diff}{\diff t}\prs{A\prs{t}\prs{Y}}
&=
\frac{\diff}{\diff t} \prs{\exp\prs{tX} \cdot Y \cdot \exp\prs{-tX}}
\\&=
\frac{\diff}{\diff t}\prs{\exp\prs{tX}} Y \exp\prs{-tX} + \exp\prs{tX}Y \frac{\diff}{\diff t} \prs{\exp\prs{-tX}}
\\&=
X \cdot A\prs{t}\prs{Y} + A\prs{t}\prs{Y} \cdot \prs{-X}
\\ \text{.} \hphantom{\frac{\diff}{\diff t}\prs{A\prs{t}\prs{Y}}} &= \mrm{ad}\prs{X} \prs{A\prs{t} \prs{Y}}
\end{align*}
לכן
\[\text{.} A'\prs{t} = \mrm{ad}\prs{X} \cdot A\prs{t}\]
ראינו שזה מחייב
\[\text{.} A\prs{t} = \mrm{exp}\prs{t \mrm{ad}\prs{X}}\]
לכן
$A\prs{1}$
מה שרצינו.
\end{proof}

\subsection{נוסחת
\textenglish{Campbell-Baker-Hausdorff}}

ראינו שקיימות סביבה
$U \subseteq M$
של
$0$
וסביבה
$V \subseteq \mrm{GL}_n\prs{k}$
של
$I$
כך ש־%
$\exp\colon U \to V$
הוא הומאומורפיזם.

עבור
$A,B \in V$
מספיק קרובות ל־%
$I$
מתקיים
$AB \in V$.
כלומר, אם
$A = \exp\prs{X}$
ו־%
$B = \exp\prs{Y}$
יש
$Z \in V$
עבורה
\[\text{.} \exp\prs{Z} = \exp\prs{X} \cdot \exp\prs{Y}\]
נחשוב על ההתאמה הזאת כפונקציה,
$Z = C\prs{X,Y}$,
שמוגדרת לפחות עבור
$X,Y$
קרובים מספיק ל־%
$0$.

נרצה נוסחה עבור
$C\prs{X,Y}$,
או לפחות תיאור כלשהו של
$C$.
ראינו שאם
$XY = YX$
אז
$C\prs{X,Y} = X+Y$.

\begin{proposition}[נוסחת
\textenglish{Campbell-Baker-Hausdorff}]
באופן כללי מתקיים
\[C\prs{X,Y} = x + Y + \frac{1}{2} \brs{X,Y} + \frac{1}{12} \prs{\brs{X, \brs{X,Y}} + \brs{Y,\brs{Y,X}}} + \ldots\]
כאשר שאר הביטויים הם קומוטטורים ארוכים יותר ב־%
$X,Y$.
\end{proposition}

\begin{remark}
מתקיים
\[\text{.} C\prs{tx, ty} = t\prs{x+Y} + \frac{t^2}{2} \brs{X,Y} + o\prs{t^2}\]
\end{remark}

\begin{definition}[אלגברת לי של מטריצות]
תת מרחב
$\mfrak{g} \leq M$
יקרא
\emph{אלגברת לי של מטריצות}
אם הוא סגור תחת הקומוטטור.
\end{definition}

\begin{corollary}
אם
$X,Y \in \mfrak{g}$
ו־%
$C\prs{X,Y}$
מוגדרת, אז
$C\prs{X,Y} \in \mfrak{g}$.
\end{corollary}

%2.11.2020

\begin{exercise}
לכל מטריצה
$X \in M_n$
מתקיים
\[\text{.} \det\prs{\exp\prs{X}} = e^{\tr\prs{X}}\]
\end{exercise}

נביט בטורי החזקות הבאים.
\begin{align*}
F\prs{z} &= \sum_{n \in \mbb{N}} \frac{\prs{-1}^n}{\prs{n+1}!} z^n = \left\{ \mat{ \frac{1-\exp\prs{-z}}{z} & z \neq 0 \\ 1 & z = 0 } \right. \\
G\prs{z} &= \sum_{n \in \mbb{N}} \frac{\prs{-1}^n}{n+1} z^n = \left\{ \mat{\frac{\log\prs{1+z}}{z} & z \neq 0 \\ 1 & z = 0} \right.
\end{align*}
כאשר
$G$
מוגדרת כשמתקיים
$\abs{z} < 1$.
אז
\begin{align*}
G\prs{\exp\prs{z} - 1} \cdot \exp\prs{z} \cdot F\prs{z} &=
\frac{z}{\exp\prs{z} - 1} \cdot \exp\prs{z} \cdot \frac{1 - \exp\prs{-z}}{z}
= 1
\end{align*}
למשל כאשר
$\abs{z} < \log\prs{2}$.

\begin{proposition}[נוסחת \textenglish{Duhamel}]
תהי
$x \colon \mbb{R} \to M_n$
מסילה חלקה.
מתקיים
\[\text{.} \frac{\diff}{\diff t} \exp\prs{x\prs{t}} = \exp\prs{x\prs{t}} F\prs{\ad\prs{x\prs{t}} \prs{x'\prs{t}}}\]
\end{proposition}

\begin{proof}
נגדיר
\[\text{.} Y\prs{s,t} \ceq \exp\prs{-s x\prs{t}} \cdot \frac{\diff}{\diff t} \prs{\exp\prs{s x\prs{t}}}\]
אז
\[\frac{\diff}{\diff t} \prs{\exp\prs{x\prs{t}}} = \exp\prs{x\prs{t}} Y\prs{1,t}\]
וגם
\[\text{.} Y\prs{0,t} = 0\]
אז
\begin{align*}
\frac{\del Y}{\del s} &= \exp\prs{-s x\prs{t}} \cdot \prs{-x\prs{t}} \cdot \frac{\diff}{\diff t}\prs{\exp\prs{sx\prs{t}}} + \exp\prs{-s x\prs{t}} \cdot \frac{\diff}{\diff t} \prs{x\prs{t} \exp\prs{s x\prs{t}}}
\\&=
\exp\prs{-sx\prs{t}} \cdot x'\prs{t} \cdot \exp\prs{sx\prs{t}}
\\&=
\Ad\prs{\exp\prs{-sx\prs{t}}}\prs{x'\prs{t}}
\\&=
\exp\prs{\ad\prs{-sx\prs{t}}} \prs{x'\prs{t}}
\\&=
\exp\prs{-s \ad\prs{x\prs{t}}} \prs{x'\prs{t}}
\end{align*}
ולכן
\begin{align*}
Y\prs{1,t} &=
\int_0^1 \frac{\del Y}{\del s}\prs{s,t} \diff s
\\&=
\int_0^1 \sum_{n \in \mbb{N}} \frac{\prs{-1}^n s^n}{n!} \prs{\ad\prs{x\prs{t}}}^n \prs{x'\prs{t}} \diff s
\\&=
\sum_{n \in \mbb{N}} \int_0^1 \frac{\prs{-1}^n s^n}{n!} \prs{\ad\prs{x\prs{t}}}^n \prs{x'\prs{t}} \diff s
\\&=
\sum_{n \in \mbb{N}} \frac{\prs{-1}^n}{\prs{n+1}!} \prs{\ad \prs{x\prs{t}}}^n \prs{x'\prs{t}}
\\&=
F\prs{\ad\prs{x\prs{t}}}\prs{x'\prs{t}}
\end{align*}
כנדרש.
\end{proof}

\begin{notation}
נסמן
\[\text{.} \brs{X_1, X_2, \ldots, X_k} \ceq \brs{X_1, \ldots , \brs{X_{k+2}, \brs{X_{k+1, X_k}}}}\]
\end{notation}

\begin{exercise}
אם
$C\prs{X,Y}$
מוגדר, גם
$C\prs{tX, tY}$
מוגדר לכל
$t \in \prs{0,1}$.
\end{exercise}

\begin{proposition}[נוסחת
\textenglish{Dynkin} (1947)]
אם
$C\prs{X,Y}$
מוגדר אז
\[\text{.} C\prs{X,Y} = \sum_{k \in \mbb{N}_+} \frac{\prs{-1}^{k-1}}{k} \sum_{\substack{i_1, \ldots, i_k \in \mbb{N} \\ j_1, \ldots, j_k \in \mbb{N} \\ \forall n \in \mbb{N} \colon i_n + j_n > 0}} \frac{1}{\prs{i_1 + j_1} \cdot \ldots \cdot \prs{i_k + j_k} \cdot i_1! j_1! \cdot \ldots \cdot i_k! j_k!} [\underset{i_1}{\underbrace{X, \ldots, X}}, \underset{j_1}{\underbrace{Y, \ldots, Y}}, \ldots, \underset{i_k}{\underbrace{X, \ldots, X}}, \underset{j_k}{\underbrace{Y, \ldots, Y}}]\]
\end{proposition}

\begin{proof}
נגדיר
\[Z\prs{t} = C\prs{tX, tY}\]
עבור
$t \in \prs{0,1}$,
ונגדיר
$Z\prs{0} = 0$.
אז
\[\exp Z\prs{t} = \exp\prs{tX} \cdot \exp\prs{tY}\]
ולכן
\begin{align*}
\Ad\prs{\exp Z\prs{t}} &= \Ad\prs{\exp\prs{tX}} \Ad\prs{\exp\prs{tY}} \\
\exp\prs{\ad Z\prs{t}} &= \exp\prs{\ad \prs{tX}} \exp\prs{\ad\prs{tY}}
\\
\text{.} \hphantom{\exp\prs{\ad Z\prs{t}}} &= \exp\prs{t \ad\prs{X}} \exp\prs{t \ad\prs{Y}}
\end{align*}
לכן מתקיים
\begin{align*}
\frac{\diff}{\diff t} \prs{\exp Z\prs{t}} &=
X \exp \prs{Z\prs{t}} + \exp\prs{Z\prs{t}} \cdot Y
\end{align*}
ומצד שני
\begin{align*}
\text{.} \frac{\diff}{\diff t} \prs{\exp Z\prs{t}} &=
\exp\prs{Z\prs{t}} \cdot F\prs{\ad Z\prs{t}} \prs{Z'\prs{t}}
\end{align*}
לכן
\[X \exp \prs{Z\prs{t}} + \exp\prs{Z\prs{t}} \cdot Y = \exp\prs{Z\prs{t}} \cdot F\prs{\ad Z\prs{t}} \prs{Z'\prs{t}}\]
 ולאחר פישוט נקבל
 \begin{align*}
 F\prs{\ad Z\prs{t}} \prs{Z'\prs{t}} &= \Ad\prs{\exp\prs{-Z\prs{t}}}\prs{X} + Y
 \\&=
 \exp\prs{-ad\prs{Z\prs{t}}}\prs{X} + Y
\end{align*}
לכן
\begin{align*}
Z'\prs{t} &= \prs{G\prs{\exp\prs{\ad \prs{Z\prs{t}}} - I} \exp\prs{\ad \prs{Z\prs{t}}}} \prs{\exp\prs{-ad\prs{Z\prs{t}}}\prs{X} + Y}
\\&=
G\prs{\exp\prs{\ad\prs{Z\prs{t}}} - I}\prs{X + \exp\prs{\ad Z\prs{t}}\prs{Y}}
\\&=
\sum_{k \in \mbb{N}} \frac{\prs{-1}^k}{k+1} \prs{\exp\prs{t \ad X} \exp\prs{t \ad Y} - I}^k \prs{X + \exp\prs{\ad \prs{Z\prs{t}}} \prs{Y}}
\end{align*}
כאשר
\begin{align*}
\exp\prs{t \ad \prs{Y}} \prs{Y} &= \sum_{j \in \mbb{N}} \frac{1}{j!} t^j \prs{\ad \prs{Y}}^j \prs{Y} = Y
\end{align*}
ולכן
\begin{align*}
Z'\prs{t} &= \sum_{k \in \mbb{N}} \frac{\prs{-1}^k}{k+1} \prs{\exp\prs{t \ad X} \exp\prs{t \ad Y} - I}^k \prs{X + \exp\prs{t \ad\prs{X}} \prs{Y}}
\\\text{.} \hphantom{Z'\prs{t}} &=
\sum_{k \in \mbb{N}} \frac{\prs{-1}^k}{k+1} \sum_{n \in \mbb{N}} t^n \brs{\text{sums of commutators in $X$ and $Y$}}
\end{align*}
אז
\begin{align*}
\text{.} C\prs{X,Y} & Z\prs{1}
\\&=
\int_0^1 Z'\prs{t} \diff t
\end{align*}
על ידי אינטגרציה איבר איבר בטור של
$Z'$
נגיע לנוסחה עבור
$C\prs{X,Y}$
ונקבל את הנדרש.
\end{proof}

\begin{remark}
מתקיים
\[\text{.} C\prs{X,Y} = \log\prs{\exp\prs{X}, \exp\prs{Y}}\]
\end{remark}

\chapter{אלגבראות לי}

\section{חבורות לי ואלגבראות לי}

\subsection{מאלגבראות לי לחבורות לי}

לכל
$V \leq M_n\prs{\mbb{R}}$
נגדיר חבורת מטריצות
\[\Gamma\prs{V} \ceq \set{\exp\prs{X_1} \cdot \ldots \cdot \exp\prs{X_m}}{x_1, \ldots, x_m \in V} \leq \mrm{GL}_n\prs{k}\]
שהיא תת־חבורת מטריצות של
$\mrm{GL}_n\prs{k}$.

\subsection{מחבורות לי לאלגבראות לי}

בכיוון ההפוך, עבור כל תת־חבורה
$G \leq \mrm{GL}_n\prs{k}$
נגדיר את
\emph{המרחב המשיק ל־%
$G$}
על ידי
\[\text{.} \mfrak{g} \ceq \mrm{Lie}\prs{G} = \set{X \in M_n}{\exists \gamma \colon \prs{-\eps, \eps} \to G \colon \substack{ \gamma \in \mscr{C}^1 \\ \gamma\prs{0} = I \\ \gamma'\prs{0} = X}}\]

\begin{proposition}
\begin{enumerate}
\item $\mfrak{g}$
מרחב וקטורי.
\item עבור
$X \in \mfrak{g}$
ו־%
$a \in G$
מתקיים
\[\text{.} \Ad\prs{a}\prs{X} \in \mfrak{g}\]
\item $\mfrak{g}$
אלגברת לי.
\end{enumerate}
\end{proposition}

\begin{proof}
\begin{enumerate}
\item תהיינה
$X,Y \in \mfrak{g}$
ויהיו
$\alpha,\beta \in \mbb{R}$.
נניח כי
$a,b$
מסילות ב־%
$G$
עבורן
\begin{align*}
a\prs{0} &= b\prs{0} = I \\
a'\prs{0} &= X \\
\text{.} b'\prs{0} &= Y
\end{align*}
נגדיר
\begin{align*}
c\prs{t} \ceq a\prs{\alpha t} \cdot b\prs{\beta\prs{t}} \in G \\
\text{.} c\prs{0} = I
\end{align*}
אז
\[\text{.} c'\prs{0} = \alpha a'\prs{0} b\prs{0} + \beta a\prs{0} b'\prs{0} = \alpha X + \beta Y \in \mfrak{g} \]
\item תהי
$\gamma$
מסילה ב־%
$G$
עם
$\gamma\prs{0} = I$
ו־%
$\gamma'\prs{0} = X$.
לכל
$t$
בתחום ההגדרה של
$\gamma$
מתקיים
\[\text{.} \delta\prs{t} \Ad\prs{a} \prs{\gamma\prs{t}} \in G\]
אז
\begin{align*}
\delta\prs{0} &= \Ad\prs{a}\prs{I} = I \\
\delta'\prs{0} &= \left.\frac{\diff}{\diff t} \prs{a \gamma\prs{t} a^{-1}}\right|_{t=0} 
\\&=
\Ad\prs{a} \prs{\gamma'\prs{0}}
\\&=
\Ad\prs{a} \prs{X}
\end{align*}
כנדרש.

\item
יהיו
$X,Y \in \mfrak{g}$
ו־%
$a,b$
מסילות ב־%
$G$
עבורן
$a'\prs{0} = X, b'\prs{0} = Y$.
יהי
\[I = a\prs{t} a\prs{t}^{-1}\]
ואז
\[\text{.} 0 = a'\prs{0} \cdot a\prs{0}^{-1} + \left. a\prs{0} \frac{\diff}{\diff t}\prs{a\prs{t}^{-1}} \right|_{t=0}\]
לכן
\[\text{.} \left. \frac{\diff}{\diff t}\prs{a\prs{t}^{-1}}\right|_{t=0} = -a'\prs{0} = -X\]
אז
\[\gamma\prs{t} = \Ad\prs{a\prs{t}}\prs{Y} \in \mfrak{g}\]
ומתקיים
\begin{align*}
\gamma'\prs{0} &= a'\prs{0} \cdot Y \cdot a\prs{0}^{-1} + a\prs{0} Y \left. \frac{\diff}{\diff t}\prs{a\prs{t}^{-1}} \right|_{t=0}
\\&=
X \cdot Y \cdot Y + I \cdot Y \cdot \prs{-X}
\\&= \brs{X,Y}
\end{align*}
לכן
\[\text{.} \brs{X,Y} = \gamma'\prs{0} \in \mfrak{g}\]
\end{enumerate}
\end{proof}

נרצה להראות שתחת תנאים מתאימים, ההעתקות
$\mrm{Lie}, \Gamma$
הופכיות אחת לשנייה.
כלומר,
נרצה ליצור התאמה חח"ע בין חבורות מטריצות מסוימות לבין אלגבראות לי.

ליתר דיוק, נוכיח שלושה דברים מרכזיים.

\begin{enumerate}
\item לכל אלגברת לי
$\mfrak{g} \leq M_n$
ממשית מתקיים
\[\text{.} \mfrak{g} = \mrm{Lie}\prs{\Gamma\prs{\mfrak{g}}}\]
\item עבור כל תת־חבורה
$G \leq \mrm{GL}_n\prs{k}$
מתקיים
\[\text{.} \Gamma\prs{\mrm{Lie}\prs{G}} \leq G\]
\item נחפש תנאים על
$G$
עבורם
\[\text{.} \Gamma\prs{\mrm{Lie}\prs{G}} = G\]
חבורה
$G$
כזאת תיקרא בהמשך
\emph{קשירה}.
\\
נוכיח את משפט התת־חבורה הסגורה:
אם
$G \leq \mrm{GL}_n\prs{k}$
תת־חבורה סגורה וקשירה, מתקיים
\[\text{.} \Gamma\prs{\mrm{Lie}\prs{G}} = G\]
\end{enumerate}

\begin{remark}
בנינו התאמה בין חבורות לי של מטריצות לאלגבראות לי של מטריצות.
ניתן להתאים באופן כללי בין חבורות לי לאלגבראות לי (אבסטרקטיות).

קיים משפט שאומר שכל אלגברת לי כזאת איזומורפית לאלגברת לי של מטריצות, ונקבל מדיון זה שחבורת לי כללית היא חבורת כיסוי של חבורות לי של מטריצות.
\end{remark}

\end{document}