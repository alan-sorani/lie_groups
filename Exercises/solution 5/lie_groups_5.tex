\documentclass[10pt]{article}

\usepackage{hyperref}
\usepackage{enumitem}
\usepackage{comment}

%%%%%%%%%
% Maths %
%%%%%%%%%

\usepackage{math-fonts}
\usepackage{math-graphics}
\usepackage{math-symbols}
\usepackage{math-theorems}

%%%%%%%%%
% Title %
%%%%%%%%%

\title{Lie Groups --- Exercise Page \#5}
\author{Elad Tzorani}
\date{\today}

\begin{document}

\maketitle

\setcounter{section}{1}

\begin{exercise}%1
Show that equivalence of categories is an equivalence relation on the collection of categories.
\end{exercise}

\begin{solution}%1
\begin{description}
\item[Reflexivity:]
Let $\mcal{C}$ be a category, there's a functor $\id_{\mcal{C}}$ acting as the identity on all objects and morphisms.
Specifically, $\id_{\mcal{C}}$ is bijective on $\hom$-sets and is essentially surjective, since for all $x \in \ob\prs{\mcal{C}}$ we have $X \cong X = \id_{\mcal{C}}\prs{X}$.
\item[Transitivity:]
Let $\mcal{C}, \mcal{D}, \mcal{E}$ be three categories and let $F \colon \mcal{C} \to \mcal{D}$ and $G \colon \mcal{D} \to \mcal{E}$ be equivalences of categories.

Let $X,Y \in \ob\prs{\mcal{C}}$.
We have
\begin{align*}
\hom_{\mcal{C}}\prs{X,Y} \underset{F_{X,Y}}{\riso} \hom_{\mcal{D}}\prs{F\prs{X}, F\prs{Y}} \underset{G_{F\prs{X}, F\prs{Y}}}{\riso} \hom_{\mcal{E}}\prs{GF\prs{X}, GF\prs{Y}}
\end{align*}
so $\prs{GF}_{X,Y} = G_{F\prs{X},F\prs{Y}} \circ F_{X,Y}$ is a bijection as a composition of bijections since $F,G$ are equivalences of categories. Hence $\prs{GF}_{X,Y}$ is a bijection so $GF$ is fully-faithful.

Let $Z \in \mcal{E}$. $G$ is an equivalence of categories and is therefore essentially-surjective. Hence there's $Y \in \mcal{D}$ such that $G\prs{Y} \cong Z$. $F$ is an equivalence of categories hence there's $X \in \mcal{C}$ such that $F\prs{X} \cong Y$.

\begin{lemma}
A functor sends isomorphisms to isomorphisms.
\end{lemma}
\begin{proof}
Let $\mcal{C},\mcal{D}$ be categories, let $\phi \in \hom_{\mcal{C}}\prs{X,Y}$ be an isomorphism, and let $F \colon \mcal{C} \to \mcal{D}$ be a functor.
We have
\[\id_{F\prs{Y}} = F\prs{\id_{Y}} = F\prs{\phi \phi^{-1}} = F\prs{\phi} \circ F\prs{\phi^{-1}}\]
and similarly $F\prs{\phi^{-1}} \circ F\prs{\phi} = \id_{F\prs{X}}$, so $F\prs{\phi^{-1}} = F\prs{\phi}^{-1}$, so $F\prs{\phi}$ is invertible and thus an isomorphism in $\mcal{D}$.
\end{proof}

We conclude from the lemma that $G\prs{F\prs{X}} \cong G\prs{Y}$ so $\prs{G \circ F}\prs{X} = G\prs{F\prs{X}} \cong Z$, which means $G \circ F$ is essentially surjective.

\item[Symmetry:]
Let $F \colon \mcal{C} \to \mcal{D}$ be an equivalence of categories.
We construct a functor $G \colon \mcal{D} \to \mcal{C}$ by taking $Y \in \ob\prs{\mcal{D}}$ to any $X \in \ob\prs{\mcal{C}}$ such that $F\prs{X} \cong Y$ (which we can do thanks to the axiom of choice). Let $f \in \hom_{\mcal{D}}\prs{Y,Y'}$ and let $X = G\prs{Y}, X' = G\prs{Y'}$. Since $F$ is fully-faithful we have $\hom_{\mcal{C}}\prs{X,X'} \cong \hom_{\mcal{D}}\prs{Y, Y'}$ so there's $\tilde{f} \in \hom_{\mcal{C}}\prs{X,X'}$ such that $f = F\prs{\tilde{f}}$. Define $G\prs{f} = \tilde{f}$.

We have to show $G$ is functorial and an equivalence of categories.

\begin{description}
\item[Functorial:]
Let $Y \in \ob\prs{\mcal{D}}$ and let $X = G\prs{Y}$. We have $\id_Y = F\prs{\id_X}$ hence by definition $G\prs{\id_Y} = \id_X = \id_{G\prs{Y}}$.

Let $f \in \hom_{\mcal{D}}\prs{Y, Y'}$ and $g \in \hom_{\mcal{D}}\prs{Y', Y''}$, we want to show $G\prs{g \circ f} = G\prs{g} \circ G\prs{f}$. By definition, $G\prs{g \circ f}$ is the unique morphism such that $F\prs{G\prs{g \circ f}} = g \circ f$. However,
\[F\prs{G\prs{g} \circ G\prs{f}} = F\prs{G\prs{g}} \circ F\prs{G\prs{f}} = g \circ f\]
by the same property and by functoriality of $F$. Hence
\[F\prs{G\prs{g} \circ G\prs{f}} = F\prs{G\prs{g \circ f}} \text{,}\]
but since $F$ is faithful this implies $G\prs{g} \circ G\prs{f} = G\prs{g \circ f}$.

\item[Fully-Faithful:]
For $Y,Y' \in \ob\prs{\mcal{D}}$ and $X = G\prs{Y}, X' = G\prs{Y'}$ we have by definition $G_{Y,Y'} = F_{X,X'}^{-1}$, hence $G_{Y,Y'} \colon \hom_{\mcal{D}}\prs{Y,Y'} \to \hom_{\mcal{C}}\prs{G\prs{Y}, G\prs{Y'}}$ is bijective.

\item[Essentially-Surjective:]
Let $X \in \ob\prs{\mcal{C}}$. Let $Y = F\prs{X}$ and let $X' = G\prs{Y}$. We want to show $X \cong X'$.
We have a bijection $F_{X,X'} \colon \hom_{\mcal{C}}\prs{X,X'} \riso \hom_{\mcal{D}}\prs{Y,Y}$ hence there's $f \in \hom_{\mcal{C}}\prs{X,X'}$ such that $F\prs{f} = \id_Y$. Similarly, swapping roles between $X,X'$ there's $g \in \hom_{\mcal{C}}\prs{X',X}$ such that $F\prs{g} = \id_Y$. We get
\begin{align*}
F\prs{f \circ g} &= F\prs{f} \circ F\prs{g} = \id_Y \circ \id_Y = \id_Y = F\prs{\id_{X'}} \\
F\prs{g \circ f} &= F\prs{g} \circ F\prs{f} = \id_Y \circ \id_Y = \id_Y = F\prs{\id_{X}}
\end{align*}
and since $F$ is faithful this implies $f \circ g = \id_{X'}$ and $g \circ f = \id_X$, which together implies $g = f^{-1}$, so $f \colon X \riso X'$ is an isomorphism, as required.
\end{description}
\end{description}
\end{solution}

\begin{exercise}[Adjoint Functors]%2
Let $\mcal{C}, \mcal{D}$ be categories. A pair $\prs{L,R}$ with $L \colon \mcal{C} \to \mcal{D}$ and $R \colon \mcal{D} \to \mcal{C}$ is called \emph{adjoint} (where $L$ is called \emph{left-adjoint to $R$} and $R$ \emph{right-adjoint to $L$}) if for any $X \in \ob\prs{\mcal{C}}, Y \in \ob\prs{\mcal{D}}$ there is a bijection
\[\hom_{\mcal{D}} \prs{L\prs{X}, Y} \underset{\Phi_{X,Y}}{\riso} \hom_{\mcal{C}}\prs{X, R\prs{Y}}\]
such that
\begin{align*}
\Phi_{X_1, Y_1}\prs{h \circ L\prs{f}} &= \Phi_{X_2, Y_1}\prs{h} \circ f
\Phi_{X_2, Y_2}\prs{g \circ h} = R\prs{g} \circ \Phi_{X_2, Y_1}\prs{h}
\end{align*}
for all
\begin{align*}
f &\in \hom_{\mcal{C}}\prs{X_1, X_2},\\
g &\in \hom_{\mcal{D}}\prs{Y_1, Y_2},\\
h &\in \hom_{\mcal{D}}\prs{F\prs{X_2}, Y_1} \text{.}
\end{align*}

Show that an equivalence of categories $F \colon \mcal{C} \to \mcal{D}$ always has a right-adjoint and a left-adjoint functor.
\end{exercise}

\begin{solution}%2
Let $F \colon \mcal{C} \to \mcal{D}$ be an equivalence of categories. By the previous exercise there's an equivalence of categories $G \colon \mcal{D} \to \mcal{C}$, which we show is a left-adjoint and a right-adjoint to $F$.

\begin{description}
\item[Right-Adjoint:]
We have to construct a bijection 
\[\Phi_{X,Y} \colon \hom_{\mcal{D}}\prs{F\prs{X}, Y} \riso \hom_{\mcal{C}}\prs{X, G\prs{Y}} \text{.}\]
By definition of $G$ we have $F \circ G = \id_{\mcal{D}}$, so
Define
\[\Phi_{X,Y} = G_{F\prs{X}, Y}\]
which is a bijection because $G$ is an equivalence of categories.

For
\begin{align*}
f &\in \hom_{\mcal{C}}\prs{X_1, X_2},\\
g &\in \hom_{\mcal{D}}\prs{Y_1, Y_2},\\
h &\in \hom_{\mcal{D}}\prs{F\prs{X_2}, Y_1}
\end{align*}
we get
\begin{align*}
\Phi_{X_1, Y_1}\prs{h \circ F\prs{f}} &= G\prs{h \circ F\prs{f}} = G\prs{h} \circ GF\prs{f} = G\prs{h} \circ f = \Phi_{X_2, Y_1}\prs{h} \circ f \\
\Phi_{X_2, Y_2}\prs{g \circ h} &= G\prs{g \circ h} = G\prs{g} \circ G\prs{h} = G\prs{g} \circ \Phi_{X_2, Y_1}\prs{h} \text{,}
\end{align*}
hence $\Phi_{X,Y}$ satisfies the required properties, so $G$ is right-adjoint to $F$.

\item[Left-Adjoint:]
Define $\Phi_{X,Y} = F_{G\prs{X}, Y}$ which is a bijection since $F$ is an equivalence of categories.

Let
\begin{align*}
f &\in \hom_{\mcal{D}}\prs{X_1, X_2},\\
g &\in \hom_{\mcal{C}}\prs{Y_1, Y_2},\\
h &\in \hom_{\mcal{C}}\prs{F\prs{X_2}, Y_1} \text{.}
\end{align*}
We have
\begin{align*}
\Phi_{X_1, Y_1}\prs{h \circ G\prs{f}} &= F\prs{h \circ G\prs{f}} = F\prs{h} \circ FG\prs{f} = F\prs{h} \circ f = \Phi_{X_2, Y_1}\prs{h} \circ f  \\
\Phi_{X_2, Y_2}\prs{g \circ h} &= F\prs{g \circ h} = F\prs{g} \circ F\prs{h} = F\prs{g} \circ \Phi_{X_2, Y_1}\prs{h} \text{,}
\end{align*}
so $\Phi$ satisfies the required properties, so $G$ is a left-adjoint to $F$.
\end{description}
\end{solution}

\newcommand{\LieGrp}{\catname{LieGrp}}
\newcommand{\LieAlg}{\catname{LieAlg}}
\newcommand{\Lie}{\mrm{Lie}}

\begin{exercise}
Show that the pair of functors $\prs{\tilde{\Gamma}, \Lie}$ is an adjoint pair between the categories of Lie groups and of Lie algebras.
You may assume facts that were proven in the case of matrix groups.
\end{exercise}

\begin{solution}%3

We use in the proof some tools from category theory which we illustrate below.

\begin{definition}[Natural Transformation]
Let $F,G \colon \mcal{C} \to \mcal{D}$ be functors. A \emph{natural transformation} $\alpha \colon F \to G$ is the data $\alpha_{X} \colon F\prs{X} \to G\prs{X}$ for all $X \in \ob\prs{\mcal{C}}$ and under the condition that the diagram
\[
\begin{tikzcd}
F\prs{X} \arrow[r, "\alpha_X"] \arrow[d, swap, "F\prs{f}"] & G\prs{X} \arrow[d, "G\prs{f}"] \\
F\prs{Y} \arrow[r, "\alpha_Y"] & G\prs{Y}
\end{tikzcd}
\]
commutes for all $f \in \hom_{\mcal{C}}\prs{X,Y}$.
\end{definition}

\begin{definition}[Natural Isomorphism]
A natural transformation $\alpha$ is a \emph{a natural isomorphism} if $\alpha_X$ is a bijection for all $X \in \ob\prs{\mcal{C}}$.
\end{definition}

\begin{comment}

\begin{definition}[Horizontal Composition of Natural Transformations]
Let $F,G,H \colon \mcal{C} \to \mcal{H}$ and let $\alpha \colon F \to G$ and $\beta \colon G \to H$ be natural transformations. We define their \emph{(horizontal) composition} $\beta \circ \alpha$ by $\prs{\beta \circ \alpha}_X = \beta_X \circ \alpha_X$.
\end{definition}

\begin{lemma}\label{lemma:natural_trans_composition}
Let $F,G,H \colon \mcal{C} \to \mcal{H}$ and let $\alpha \colon F \to G$ and $\beta \colon G \to H$ be natural transformations. Then $\beta \circ \alpha$ is also a natural transformation.
\end{lemma}

\begin{proof}
The large small squares in
\[
\begin{tikzcd}
F\prs{X} \arrow[r, "\alpha_X"] \arrow[d, "F\prs{f}"] &
G\prs{X} \arrow[r, "\beta_X"] \arrow[d, "G\prs{f}"] &
H\prs{X} \arrow[d, "H\prs{f}"] \\
F\prs{Y} \arrow[r, "\alpha_Y"] &
G\prs{Y} \arrow[r, "\beta_Y"] &
H\prs{Y}
\end{tikzcd}
\]
commute for all $f \in \hom_{\mcal{C}}\prs{X,Y}$ since $\alpha,\beta$ are natural transformations.

This implies in general that the large square (it's a rectangle, but people call those squares) commutes for indeed
\begin{align*}
H\prs{f} \circ \beta_X \circ \alpha_X &= \prs{H\prs{f} \circ \beta_X} \circ \alpha
\\&= \beta_Y \circ G\prs{f} \circ \alpha_X
\\&= \beta_Y \circ \prs{G\prs{f} \circ \alpha_X}
\\&= \beta_Y \circ \alpha_Y \circ F\prs{f} \text{.}
\end{align*} 
\end{proof}

\begin{lemma}\label{lemma:natural_iso_composition}
Let $F,G,H \colon \mcal{C} \to \mcal{H}$ and let $\alpha \colon F \to G$ and $\beta \colon G \to H$ be natural isomorphisms. $\beta \circ \alpha$ is also a natural isomorphism.
\end{lemma}

\end{comment}

\begin{proof}
$\alpha,\beta$ are natural isomorphisms, so $\alpha_X, \beta_X$ are bijections for all $X \in \ob\prs{\mcal{C}}$, hence so are $\beta_X \circ \alpha_X$.
$\beta \circ \alpha$ is a natural transformation by \eqref{lemma:natural_trans_composition}, hence this implies $\beta \circ \alpha$ is a natural isomorphism.
\end{proof}

\begin{definition}[Product Category]
Let $\mcal{C}, \mcal{D}$ be categories. We define $\mcal{C} \times \mcal{D}$ with $\ob\prs{\mcal{C} \times \mcal{D}} = \ob\prs{\mcal{C}} \times \ob\prs{\mcal{D}}$ and with
\[\hom_{\mcal{C} \times \mcal{D}}\prs{\prs{X,Y}, \prs{X',Y'}} = \hom_{\mcal{C}}\prs{X, X'} \times \hom_{\mcal{D}}\prs{Y,Y'}\]
and composition $\prs{f_2, g_2} \circ \prs{f_1, g_1} = \prs{f_2 \circ f_1, g_2 \circ g_1}$.
\end{definition}

\begin{definition}[Hom-Functor]
Let $\mcal{C}$ be a category. We define
\[\hom_{\mcal{C}} \colon \mcal{C}^\op \times \mcal{C} \to \catname{Set}\]
where $\catname{Set}$ is the category of sets.
On objects, let $\hom_{\mcal{C}}$ be as defined in class.
On morphisms, for $f^\op \colon X \to Y$ in $\mcal{C}^\op$ and $g \colon X' \to Y'$ in $\mcal{C}$, define
\begin{align*}
\hom_{\mcal{C}}\prs{f^\op,g} \colon \hom\prs{X, X'} &\to \hom\prs{Y,Y'} \\
h &\mapsto g \circ h \circ f \text{,}
\end{align*}
where $f^\op$ is $f$ when viewed as a morphism in the opposite category (so that $f^\op \colon X \to Y$ means $f \colon Y \to X$).
\end{definition}

\begin{lemma}\label{lemma:hom_functor}
$\hom_{\mcal{C}}$ is a functor.
\end{lemma}

\begin{proof}
Let $\prs{f_1^\op, g_1} \colon \prs{X,X'} \to \prs{Y,Y'}$ and $\prs{f_2^\op, g_2} \colon \prs{Y,Y'} \to \prs{Z,Z'}$ be morphisms in $\mcal{C}^\op \times \mcal{C}$.
Let $h \in \hom\prs{X,X'}$, we have
\begin{align*}
\hom_{\mcal{C}}\prs{\prs{f_2^\op, g_2} \circ \prs{f_1^\op, g_1}}\prs{h}
&=
\hom_{\mcal{C}}\prs{f_2^\op \circ f_1^\op, g_2 \circ g_1}\prs{h}
\\&=
\hom_{\mcal{C}}\prs{\prs{f_1 \circ f_2}^\op, g_2 \circ g_1}\prs{h}
\\&= g_2 \circ g_1 \circ h \circ f_1 \circ f_2
\\&= g_2 \circ \prs{g_1 \circ h \circ f_1} \circ f_2
\\&= \hom_{\mcal{C}}\prs{f_2^\op, g_2}\prs{g_1 \circ h \circ f_1}
\\&= \hom_{\mcal{C}}\prs{f_2^\op, g_2}\prs{\hom_{\mcal{C}}\prs{f_1^\op, g_1}\prs{h}}
\\&= \prs{\hom_{\mcal{C}}\prs{f_2^\op, g_2} \circ \hom_{\mcal{C}}\prs{f_1^\op, g_1}}\prs{h} \text{.}
\end{align*}
\end{proof}

\begin{lemma}\label{lemma:hom_functor2}
Let $F \colon \mcal{C} \to \mcal{D}$ and $G \colon \mcal{D} \to \mcal{C}$ be functors. Then so are $\hom_{\mcal{D}}\prs{F\prs{-}, \prs{-}}$ and $\hom_{\mcal{C}}\prs{-,G\prs{-}}$.
\end{lemma}

\begin{proof}
\begin{itemize}
\item Let $f_1^\op, f_2^\op$ be morphisms in $\mcal{D}$ such that $f_2^\op \circ f_1^\op$ is defined and let $g_1, g_2$ be morphisms in $\mcal{C}$ such that $g_2 \circ g_1$ is defined.

We have
\begin{align*}
\hom_{\mcal{D}}\prs{F\prs{f_1^\op \circ f_2^\op}, g_2 \circ g_1}
&=
\hom_{\mcal{D}}\prs{F\prs{f_2 \circ f_1}^\op, g_2 \circ g_1}
\\&=
\hom_{\mcal{D}}\prs{F\prs{f_2 \circ f_1}^\op, g_2 \circ g_1}
\\&=
\hom_{\mcal{D}}\prs{\prs{F\prs{f_2} \circ F\prs{f_1}}^\op, g_2 \circ g_1}\prs{h}
\\&=
\hom_{\mcal{D}}\prs{F\prs{f_2}^\op, g_2} \circ \hom_{\mcal{D}}\prs{F\prs{f_1}^\op, g_2}
\\&=
\hom_{\mcal{D}}\prs{F\prs{f_2^\op}, g_2} \circ \hom_{\mcal{D}}\prs{F\prs{f_1^\op}, g_2}
\end{align*}
where we write $F$ also for the induced map $\mcal{C}^\op \to \mcal{D}^\op$, and where the second-to-last equality is due to \eqref{lemma:hom_functor}.

\item Let $f_1^\op, f_2^\op$ be morphisms in $\mcal{C}$ such that $f_2^\op \circ f_1^\op$ is defined and let $g_1, g_2$ be morphisms in $\mcal{D}$ such that $g_2 \circ g_1$ is defined.

We have
\begin{align*}
\hom_{\mcal{C}}\prs{{f_1^\op \circ f_2^\op}, G\prs{g_2 \circ g_1}}
&=
\hom_{\mcal{C}}\prs{{f_2 \circ f_1}^\op, G\prs{g_2 \circ g_1}}
\\&=
\hom_{\mcal{C}}\prs{{f_2 \circ f_1}^\op, G\prs{g_2 \circ g_1}}
\\&=
\hom_{\mcal{C}}\prs{\prs{\prs{f_2} \circ {f_1}}^\op, G\prs{g_2 \circ g_1}}\prs{h}
\\&=
\hom_{\mcal{C}}\prs{{f_2}^\op, G\prs{g_2}} \circ \hom_{\mcal{C}}\prs{{f_1}^\op, G\prs{g_2}}
\end{align*}
where the last equality is due to \eqref{lemma:hom_functor}.
\end{itemize}
\end{proof}


\begin{lemma}\label{lemma:adjunction_criterion}
Let $L \colon \mcal{C} \to \mcal{D}$ and $R \colon \mcal{D} \to \mcal{C}$ be functors. Then $L$ is left-adjoint to $R$ if and only if there's a natural isomorphism
\[\alpha \colon \hom_{\mcal{D}}\prs{L\prs{-}, -} \riso \hom_{\mcal{C}}\prs{-,R\prs{-}} \text{.}\]
\end{lemma}

\begin{proof}
\begin{itemize}
\item Assume that $L$ is left-adjoint to $R$, and let
\[\Phi_{X,Y} \colon \hom_{\mcal{D}}\prs{L\prs{X}, Y} \riso \hom_{\mcal{C}}\prs{X,R\prs{Y}}\]
be bijections satisfying the conditions in the exercise.
Let $\alpha \colon \hom_{\mcal{D}}\prs{L\prs{-}, -} \to \hom_{\mcal{C}}\prs{-,R\prs{-}}$, given by $\alpha_{X,Y} \ceq \Phi_{X,Y}$ as above, which we show is a natural transformation.

To show $\alpha$ is natural we have to show that
\[
\begin{tikzcd}
\hom_{\mcal{D}}\prs{L\prs{X_1}, Y_1} \arrow[r, "\Phi_{X_1,Y_1}"] \arrow[d, swap, "\hom_{\mcal{D}}\prs{L\prs{f^\op}, g}"] & \hom_{\mcal{C}}\prs{X_1, R\prs{Y_1}} \arrow[d, "\hom_{\mcal{C}}\prs{f^\op, R\prs{g}}"] \\
\hom_{\mcal{D}}\prs{L\prs{X_2}, Y_2} \arrow[r, "\Phi_{X_2,Y_2}"] & \hom_{\mcal{C}}\prs{X_2, R\prs{Y_2}}
\end{tikzcd}
\]
commutes for any $X,X_2 \in \mcal{C}$, $Y,Y_2 \in \mcal{D}$ and $\prs{f^\op, g} \colon \prs{X_1, Y_1} \to \prs{X_2, Y_2}$ in $\mcal{C}^\op \times \mcal{C}$.
Opening the definition of $\hom_{\mcal{D}}, \hom_{\mcal{C}}$ we need to show commutativity of the following.
\[
\begin{tikzcd}
\hom_{\mcal{D}}\prs{L\prs{X_1}, Y_1} \arrow[r, "\Phi_{X_1,Y_1}"] \arrow[d, swap, "g \circ \prs{-} \circ L\prs{f}"] & \hom_{\mcal{C}}\prs{X_1, R\prs{Y_1}} \arrow[d, "R\prs{g} \circ \prs{-} \circ f"] \\
\hom_{\mcal{D}}\prs{L\prs{X_2}, Y_2} \arrow[r, "\Phi_{X_2,Y_2}"] & \hom_{\mcal{C}}\prs{X_2, R\prs{Y_2}}
\end{tikzcd}
\]
%
We can decompose each vertical map so that this is equivalent to commutativity of the large square in the following.
\begin{equation}\label{equation:adjunction}
\begin{tikzcd}
\hom_{\mcal{D}}\prs{L\prs{X_1}, Y_1} \arrow[r, "\Phi_{X_1,Y_1}"] \arrow[d, swap, "g \circ \prs{-}"] & \hom_{\mcal{C}}\prs{X_1, R\prs{Y_1}} \arrow[d, "R\prs{g} \circ \prs{-}"]
\\
\hom_{\mcal{D}}\prs{L\prs{X_1}, Y_2} \arrow[r, "\Phi_{X_1,Y_2}"] \arrow[d, swap, "\prs{-} \circ L\prs{f}"] & \hom_{\mcal{C}}\prs{X_1, R\prs{Y_2}} \arrow[d, "\prs{-} \circ f"]
\\
\hom_{\mcal{D}}\prs{L\prs{X_2}, Y_2} \arrow[r, "\Phi_{X_2,Y_2}"] & \hom_{\mcal{C}}\prs{X_2, R\prs{Y_2}}
\end{tikzcd}
\end{equation}

The smaller squares are both commutative, the bottom one by the first condition on $\Phi_{X,Y}$ and the top by the second condition. Hence the bigger square is commutative, so $\alpha$ is a natural transformation, hence thus a natural isomorphism.

\item Assume There's a natural isomorphism
\[\alpha \colon \hom_{\mcal{D}}\prs{L\prs{-}, Y} \riso \hom_{\mcal{C}}\prs{-,R\prs{-}}\]
and let $\Phi_{X,Y} = \alpha_{X,Y}$.
Since $\alpha$ is natural, we have by the above equivalence that the large square in \eqref{equation:adjunction} commutes for all $f,g$. Taking $Y = Y_1 = Y_2$ and $g = \id_Y$ the large square is exactly the bottom one which therefore commutes.
Taking $X = X_1 = X_2$ and $f = \id_{X}$, the large square is the same as the top one which therefore commutes.
Commutativity of these squares is equivalent to the equation conditions on $\Phi_{X,Y}$, and $\Phi_{X,Y} = \alpha_{X,Y}$ is a bijection for all $X,Y $ since $\alpha$ is an isomorphism.

Hence $\Phi_{X,Y}$ satisfies all the conditions, so $L$ is left-adjoint to $R$.
\end{itemize}
\end{proof}

\begin{lemma}\label{lemma:composition_of_adjoints}
Let $L_i$ be left adjoints to $R_i$ in the following.
\[
\begin{tikzcd}
\mcal{C} \arrow[r, swap, shift right = 3pt, "R_1"] & \mcal{D} \arrow[r, swap, shift right = 3pt, "R_2"] \arrow[l, shift right = 3pt, swap, "L_1"] & \mcal{E} \arrow[l, swap, shift right = 3pt, "L_2"]
\end{tikzcd}
\]
Then $L_1 \circ L_2$ is left-adjoint to $R_2 \circ R_1$.
\end{lemma}

\begin{proof}
%TODO fill in via (co)unit (see adjoint functors on wikipedia)
\end{proof}

Denote $\catname{LieGrp}, \catname{LieAlg}$ the respective categories of Lie groups and algebras, with Lie group homomorphisms and Lie algebra homomorphisms. Denote by $\catname{LieGrp}_{\mrm{sc}}$ the subcategory of simply-connected Lie groups within $\catname{LieGrp}$.

The essential image of $\tilde{\Gamma}$ is contained in $\catname{LieGrp}_{\mrm{sc}}$, hence it factors as $\iota \circ \hat{\Gamma}$ where $\iota \colon \catname{LieGrp}_{\mrm{sc}} \rmono \catname{LieGrp}$ is the embedding, and we've seen that $\hat{\Gamma}$ is an equivalence of categories.
Let $\Lie_0 \colon \catname{LieGrp}_{\mrm{sc}} \to \catname{LieAlg}$ be the restriction of $\Lie$ to $\catname{LieGrp}_{\mrm{sc}}$, which is the inverse to $\hat{\Gamma}$ as seen in class. 

\begin{lemma}

\end{lemma}

\end{solution}

\end{document}
